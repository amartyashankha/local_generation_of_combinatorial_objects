\section{Introduction}
Consider an algorithm performing a computation on a \emph{huge random object} (for example a huge random graph or a ``long'' random walk).
Is it necessary to generate the entire object prior to the computation,
or is it possible to provide local query access to the object and generate it incrementally ``on-the-fly'' (as requested by the algorithm)?
Such an \emph{implementation} would ideally emulate the random object by answering appropriate queries,
in a manner that is consistent with a specific instance of the random object, sampled from the true distribution.
This paradigm is useful when we wish to simulate a sub-linear algorithm on a random object,
since sampling the entire object up front would ruin its efficiency.

In this work, we focus on generating huge random objects in a number of new settings,
including basic random graph models that were not previously considered, Catalan objects, and random colorings of graphs.
One emerging theme that we develop further is to provide access to random objects through more complex yet natural queries.
For example, consider an implementation for Erd\"os-R\'enyi $G(n,p)$ random graphs,
where the simplest query \func{Vertex-Pair}$(u,v)$ would ask about the existence of edge $(u,v)$,
which can be answered trivially by flipping a coin with bias $p$ (thus revealing a single entry in the adjacency matrix).
However, many applications involving non-dense graphs would benefit from \emph{adjacency list} access,
which we provide through \func{Next-Neighbor} and \func{Random-Neighbor} queries\footnote{
\label{foot:graph_queries}
\func{Vertex-Pair}$(u,v)$ returns whether $u$ and $v$ are adjacent, \func{Next-Neighbor}$(v)$ returns a new neighbor of $v$ each time
it is invoked (until none is left), and \func{Random-Neighbor}$(v)$ returns a uniform random neighbor of $v$ (if $v$ is not isolated).}

The problem of sampling partial information about huge random objects was pioneered in \cite{huge,huge_old,huge_journal},
through the implementation of \emph{indistinguishable} pseudo-random objects of exponential size.
Further work in \cite{sparse,reut} considers the implementation of different random graph models.
\cite{reut} introduced the study of $\func{Next-Neighbor}$ queries which provide efficient access to the adjacency list representation.
In addition to supporting \func{Vertex-Pair} and \func{Next-Neighbor},
we also introduce and implement a new \func{Random-Neighbor} query for undirected graphs with independent edge probabilities.

%To support the implementation of a richer class of combinatorial objects, such as random colorings of an input graph $G$,
%we define a new model for implementing random objects which have huge description size (the underlying graph $G$).
%In this setting, our algorithms are unable to read the full input,
%but can still answer queries to a valid sampled object using a sub-linear number of local probes to the input description.

Finally, we define a new model that allows us to generate a richer class of random objects that do not have a succinct description.
Specifically, we study uniformly random \emph{valid} $q$-colorings of an input graph $G$ with max degree $\Delta$.
This is in contrast to prior work in the area, where random objects are defined as a distribution with $\mathcal O(1)$ parameters
(for example, $n$ and $p$ in the $G(n,p)$ model).
The distribution over valid colorings is instead specified via a ``huge'' input (the underlying graph $G$),
that is too large to be read by a sub-linear time algorithm.
Instead, our implementation accesses $G$ using local neighborhood probes.

This new model can be compared to \emph{Local Computation Algorithms}, which also implement query access to a consistent valid solution,
and read their input using local probes.
Inspired by this connection, we extend our model to support \emph{memory-less} implementations.
This allows multiple independent (possibly simultaneous) instantiations to agree on the same random object, without any communication.
We show how to implement local access to color of any given node in a random coloring, using sub-linear resources.
Unlike LCAs which can generate an \emph{arbitrary} valid solution, our model requires a uniformly random one.





\paragraph*{Random Graphs (Section~\ref{sec:undirected} and Section~\ref{sec:small_world})}%
\label{par:random_graphs}
Random graphs are one of the most well studied types of random object.
We consider the problem of implementing local access to a number of fundamental random graph models,
through natural queries, including \func{Vertex-Pair}, \func{Next-Neighbor},
and a newly introduced \func{Random-neighbor} query\footref{foot:graph_queries}, using polylogarithmic resources per query.
Our results on random graphs are summarized in Table~\ref{table:graph_results}.

\subparagraph*{Undirected Random Graphs with Independent Edge Probabilities:}
\label{par:undirected_random_graphs_with_independent_edge_probabilities}
We implement the aforementioned queries for the generic class of \emph{undirected graphs} with \emph{independent edge probabilities}
$\left\{ p_{uv} \right\}_{u,v\in V}$, where $p_{uv}$ denotes the probability that there is an edge between $u$ and $v$.
Under reasonable assumptions on the ability to compute certain values pertaining to consecutive edge probabilities,
our implementations support \func{Vertex-Pair}, \func{Next-Neighbor}, and \func{Random-Neighbor} queries\footref{foot:graph_queries},
using $\mathcal{O}(\poly(\log n))$ time, space, and random bits.
Note that in this setting, $\func{Vertex-Pair}$ queries are trivial by themselves,
since the existence of an edge depends on an independent random variable.
However, maintaining all three types of queries simultaneously is much harder.
As in \cite{reut} (and unlike many of the implementations presented in \cite{huge_old,huge}), our techniques allow unlimited queries,
and the generated random objects are sampled from the true distribution (rather than just being indistinguishable).
In particular, our construction yields local-access implementations for the Erd\"{o}s-R\'{e}nyi $G(n,p)$ model (for \emph{all} values of $p$),
and the Stochastic Block model with random community assignment.

\begin{table}[htpb]
\centering
\renewcommand{\arraystretch}{1}
\begin{tabular}{| >{\centering\arraybackslash}m{120pt} || >{\centering\arraybackslash}m{50pt} | >{\centering\arraybackslash}m{60pt} | >{\centering\arraybackslash}m{70pt} | >{\centering\arraybackslash}m{60pt} |}
    \hline
    \textbf{Model}              & \func{Vertex-Pair} & \func{Next-Neighbor} & \func{Random-Neighbor} & \func{All-Neighbors} \\      \hline \hline
    $G(n,p)$ with $p=\frac{\log^{\mathcal O(1)} n}{n}$
                                & \cite{sparse}      & \cite{sparse}        & \cite{sparse}          & \cite{sparse}        \\[5pt] \cline{1-5}
    $G(n,p)$ for arbitrary $p$  & This paper         & This paper           & This paper             & \UBD                 \\[5pt] \cline{1-5}
    Stochastic Block Model with $\log^{\mathcal O(1)} n$ communities
                                & This paper         & This paper           & This paper             & \UBD                 \\[5pt] \cline{1-5}
    BA Preferential Attachment  & \cite{reut}        & \cite{reut}          & \BD                    & \UBD                 \\[5pt] \cline{1-5}
    Small world model           & This paper         & This paper           & This paper             & This paper           \\[5pt] \cline{1-5}
    Random ordered rooted trees & This paper         & This paper           & This paper             & This paper           \\[5pt] \cline{1-5}
    \end{tabular}
    \vspace{0.7em}
    \caption{\textbf{(Local access implementations for random graphs):}
        All the implementations in this table use polylogarithimc time, additional space and random bits per query.
        A \UBD\ in the \func{All-Neighbors} column indicates that the graphs in this model may have un-bounded degree,
        and it is therefore impossible to sample \func{All-Neighbors} efficiently.
        A \BD\ entry indicates a sampling problem with no known efficient solution.}
    \label{table:graph_results}
\end{table}

\vspace{-1.5em}

While \func{Vertex-Pair} and \func{Next-Neighbor} queries %, as well as \func{All-Neighbors} queries for sparse graphs,
have been considered in prior work \cite{reut, huge_old, sparse},
we provide the first implementations of these in non-sparse random graph models.
Prior results for implementing queries to $G(n,p)$ focused on the sparse case where $p = \log^{\mathcal O(1)} n/n$ \cite{sparse}.
The dense case where $p = \Theta(1)$ is also relatively simple because most of the adjacency matrix is filled,
and neighbor queries can be answered by performing $\Theta(1)$ \func{Vertex-Pair} queries until an edge is found.
The case of general $p$ is more involved, and was not considered previously.
For example, when $p = 1/\sqrt{n}$, each vertex has high degree $\mathcal O(\sqrt{n})$ but most of the adjacency matrix is empty,
thus making it difficult to generate a neighbor efficiently.
\func{Next-Neighbor} queries were introduced in \cite{reut}
in order to access the neighborhoods of vertices in non-sparse graphs in \emph{lexicographic order}.
This query however, does not allow us to efficiently explore graphs in some natural ways, such as via random walks,
since the initially returned neighbors are biased by the lexicographic ordering.
We address this deficiency by introducing a new \func{Random-Neighbor} query,
which would be useful, for instance, in sub-linear algorithms that employ random walk processes.
We provide the first implementation of all three queries in \emph{non-sparse graphs} as follows:

\begin{restatable}{theorem}{UndirectedGrand}
\label{thm:grand}
Given a random graph model defined by the probability matrix $\{ p_{uv}\}_{u,v\in [n]}$,
and assuming that we can compute the the quantities $\prod_{u=a}^b (1-p_{vu})$ and $\sum_{u=a}^b p_{vu}$ in $\mathcal O(\log^{\mathcal O(1)} n)$ time,
there exists an implementation for this model that supports local access through \func{Random-Neighbor}, \func{Vertex-Pair},
and \func{Next-Neighbor} queries using $\mathcal O(\log^{\mathcal O(1)} n)$ running time, additional space, and random bits per query.
\end{restatable}

We show that these assumptions can be realized in Erd\"os-R\'enyi random graphs and the Stochastic Block Model.
SBM presents additional challenges for assigning community labels to vertices (Section~\ref{sec:applications_to_random_graph_models}).

\begin{restatable}{corollary}{ERGrand}
There exists an algorithm that implements local access to a Erd\"{o}s-R\'{e}nyi $G(n,p)$ random graph,
through \func{Vertex-Pair}, \func{Next-Neighbor}, and \func{Random-Neighbor} queries,
while using $\bo(\log^3 n)$ time, $\bo(\log^3 n)$ random bits, and $\bo(\log^2 n)$ additional space per query with high probability.
\end{restatable}

\begin{restatable}{corollary}{SBMGrand}\label{cor:sbm-construct}
There exists an algorithm that implements local access to a random graph from the $n$-vertex Stochastic Block Model
with $r$ randomly-assigned communities,
through \func{Vertex-Pair}, \func{Next-Neighbor}, and \func{Random-Neighbor} queries,
using $O(r\,\poly(\log n))$ time, random bits, and space per query w.h.p.
\end{restatable}

We remark that while we are able to generate Erd\"{o}s-R\'{e}nyi random graphs on-the-fly supporting all three types of queries,
our construction still only requires $\widetilde{\mathcal O}(m+n)$ time and space to generate a complete $G(n,p)$ graph,
which is optimal up to logarithmic factors.

\begin{restatable}{corollary}{EROptimal}
\label{thm:er-optimal}
The final algorithm in Section~\ref{sec:undirected} can generate a complete random graph
from the Erd\"{o}s-R\'{e}nyi $G(n,p)$ model using overall
$\widetilde{\mathcal{O}}(n+m)$ time, random bits and space, which is $\widetilde{\bo}(pn^2)$ in expectation.
This is optimal up to $ \mathcal{O}(\poly(\log n))$ factors.
\end{restatable}




\subparagraph*{Directed Random Graphs - The Small World Model (Section~\ref{sec:small_world}):}
\label{par:directed_random_graphs}
We consider local-access implementations for directed graphs through Kleinberg's Small World model,
where the probabilities are based on distances in a 2-dimensional grid.
This model was introduced in \cite{kleinberg} to capture the geographical structure of real world networks,
in addition to reproducing observed properties of short routing paths and low diameter.
We implement \func{All-Neighbors} queries for the Small World model, using $\mathcal{O}(\poly(\log n))$ time, space and random bits.
Since such graphs are sparse, the other queries follow directly.
\begin{restatable}{theorem}{DirectedGrand}\label{DirectedGrand}
There exists an algorithm that implements \func{All-Neighbors} queries for a random graph from Kleinberg's Small World model,
where probability of including each directed non-grid edge $(u,v)$ in the graph is $c/(\func{dist}(u,v))^2$,
where the range of allowed values of $c$ is $\log^{\pm\mathcal O(1)} n$ and $\func{dist}$ denotes the Manhattan distance,
using $O(\log^2 n)$ time and random words per query with high probability.
\end{restatable}





\paragraph*{Catalan Objects (Section~\ref{sec:catalan_objects})}%
\label{par:intro_catalan_objects}
Catalan objects capture a well studied combinatorial property and admit numerous interpretations that include
well bracketed expressions, rooted trees and binary trees, Dyck languages etc.
In this paper, we focus on the Dyck path interpretation and implement local access to a uniformly random instance of a Dyck path.
Since Dyck paths have natural bijections to other Catalan objects,
we can use our Dyck Path implementation to obtain implementations of these random Catalan objects.

A Dyck path is defined as a sequence of $n$ upward $(+1)$ steps, and $n$ downward $(-1)$ steps,
with the added constraint that the \emph{sum of any prefix of the sequence is non-negative}.
A \emph{random} Dyck path may be viewed as a constrained one dimensional balanced random walk.
A natural query on Dyck paths is $\func{Height}(t)$ which returns the position of the walk after $t$ steps
(equivalently, sum of the first $t$ sequence elements).
$\func{Height}$ queries correspond to $\func{Depth}$ queries on rooted trees and bracketed expressions
(Section~\ref{sec:bijections_to_other_catalan_objects}).

We also introduce and support \func{First-Return} queries, where $\func{First-Return}(t)$ returns
the first time when the random walk returns to the same \emph{height} as it was at time $t$, as long as $\func{Height}(t+1)=\func{Height}(t)+1$
(Section~\ref{sec:bijections_to_other_catalan_objects} presents the rationale for this definition).
This allows us to support more involved queries that are widely used;
for example, \func{First-Return} queries are equivalent to finding the next child of a node in a random rooted tree,
and also to finding a matching closing bracket in random bracketed expressions.

\func{Height} queries for unconstrained random walks follow trivially from the implementation of interval summable functions in \cite{huge, histogram}.
However, the added non-negativity constraint introduces intricate non-local dependencies on the distribution of positions.
We show how to overcome these challenges, and support both queries using $\mathcal O(\poly(\log n))$ resources.

\begin{restatable}{theorem}{CatalanGrand}
\label{thm:catalan_main}
There is an algorithm using $\mathcal O(\log^{\mathcal O(1)} n)$ resources per query that provides access to
a uniformly random Dyck path of length $2n$ by implementing the following queries:
\begin{itemize}
    \item \func{Height}$(t)$ returns the position of the walk after $t$ steps.
    \item \func{First-Return}$(t)$: If $\func{Height}(t+1) > \func{Height}(t)$, then $\func{First-Return}(t)$ returns the smallest index $t'$,
    such that $t'>t$ and $\func{Height}(t') = \func{Height}(t)$ (i.e. the first time the Dyck path returns to the same height).
    Otherwise, if $\func{Height}(t+1) < \func{Height}(t)$, then $\func{First-Return}(t)$ is not defined.
\end{itemize}
\end{restatable}




\paragraph*{Random Coloring of Graphs - A New Model (Section~\ref{sec:random_coloring_of_a_graph}):}%
\label{par:random_coloring_of_graphs}
So far, all the results in this area have focused on random objects with a small description size;
for instance, the $G(n, p)$ model is described with only two parameters $n$ and $p$.
We introduce a new model for implementing random objects with \emph{huge input description};
that is, the distribution is specified as a uniformly random solution to a huge combinatorial problem.
The challenge is that now our algorithms cannot read the entire description in sub-linear time.

In this model, we implement query access to random $q$-colorings of a given huge graph $G$ with maximum degree $\Delta$.
A random coloring is generated by proposing $\mathcal O(n\log n)$ color updates and accepting the ones that do not create a conflict (Glauber dynamics).
This is an inherently sequential process with the acceptance of a particular proposal depending on all preceding neighboring proposals.
Moreover, unlike the previously considered random objects, this one has no succinct representation, and we can only uncover the proper distribution
by probing the underlying graph (in the manner of \emph{local computation algorithms} \cite{LCA, LCA_space_efficient}).
Unlike LCAs which can return an \emph{arbitrary} valid solution, we also have to make sure that we return a solution from the correct distribution.
We are able to construct an efficient oracle that returns the final color of a vertex using only a sub-linear number of probes when $q\ge 9\Delta$.

This implementation also has the feature that multiple independent instances of the algorithm having access to the same random bits,
will respond to queries in a manner consistent with each other; they will generate exactly the same coloring, regardless of the queries asked.
Since these implementations are memory-less, the resulting coloring is \emph{oblivious} to the order of queries,
and only depends on common random bits,

\begin{restatable}{theorem}{ColoringGrand}
\label{thm:coloring_grand}
Given neighborhood query access to a graph $G$ with $n$ nodes, maximum degree $\Delta$, and $q=2\alpha\Delta \ge 9\Delta$ colors,
we can generate the color of any given node from a distribution of color assignments that is $\epsilon$-close (in $L_1$ distance)
to the uniform distribution over all valid colorings of $G$, in a consistent manner,
using only $\mathcal O((n/\epsilon)^{3.06/\alpha}\Delta\log (n/\epsilon))$ time, probes, and public random bits per query.
\end{restatable}




\subsection{Related Work}
\label{sec:related_work}
The problem of computing local information of huge random objects was pioneered in \cite{huge_old,huge}.
Further work of \cite{sparse} considers the implementation of sparse $G(n,p)$ graphs from the Erd\"{o}s-R\'{e}nyi model \cite{er},
with $p = O(\poly(\log n)/n)$, by implementing \func{All-Neighbors} queries.
While these implementations use polylogarithmic resources over their entire execution,
they generate graphs that are only guaranteed to \emph{appear random} to algorithms that inspect a \emph{limited portion} of the generated graph.

In \cite{reut}, the authors construct an oracle for the implementation of recursive trees, and BA preferential attachment graphs.
Unlike prior work, their implementation allows for an arbitrary number of queries.
Although the graphs in this model are generated via a sequential process,
the oracle is able to locally generate arbitrary portions of it and answer queries in polylogarithmic time.
Though preferential attachment graphs are sparse, they contain vertices of high degree,
thus \cite{reut} provides access to the adjacency list through \func{Next-Neighbor} queries.
For additional related work, see Section~\ref{sec:additional_related_work}.



\subsection{Applications}
\label{sec:applications}
One motivation for implementing local access to huge random objects is so that we can simulate sub-linear time algorithms on them.
The standard paradigm of generating the entire (input) object a priori is wasteful,
because sub-linear algorithms only inspect a small fraction of the generated input.
For example, the greedy routing algorithm on Kleinberg's small world networks \cite{kleinberg}
only uses $\mathcal O(\log^2 n)$ probes to the underlying network.
Using our implementation, one can execute this algorithm on a random small world instance in $\mathcal O(\poly(\log n))$ time,
without incurring the $\mathcal O(n)$ prior-sampling overhead, by generating only those parts of the graph that are accessed by the routing algorithm.

Local access implementations may also be used to design parallel generators for random objects.
The model in Definition~\ref{def:local_access_LCA} is particularly suited to parallelization;
different processors/machines can generate parts of the random object independently,
using a \emph{read-only} shared memory containing public random bits.
