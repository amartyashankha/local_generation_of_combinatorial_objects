\begin{abstract}

Consider an algorithm performing a computation on a \emph{huge random object} (for example a random graph or a ``long'' random walk).
Is it necessary to generate the entire object prior to the computation,
or is it possible to provide query access to the object and sample it incrementally ``on-the-fly'' (as requested by the algorithm)?
Such an \emph{implementation} should emulate the random object by answering queries in a manner consistent with
an instance of the random object sampled from the true distribution (or close to it).
This paradigm is useful when the algorithm is sub-linear and thus, sampling the entire object up front would ruin its efficiency.

Our first set of results focus on undirected graphs with independent edge probabilities,
i.e. each edge is chosen as an independent Bernoulli random variable.
We provide a general implementation for this model under certain assumptions.
Then, we use this to obtain the first efficient local implementations for the Erd\"os-R\'enyi $G(n,p)$ model for \emph{all} values of $p$,
and the Stochastic Block model.
As in previous local-access implementations for random graphs, we support \func{Vertex-Pair} and \func{Next-Neighbor} queries.
In addition, we introduce a new \func{Random-Neighbor} query.
Next, we give the first local-access implementation for \func{All-Neighbors} queries in the (sparse and directed) Kleinberg's Small-World model.
Our implementations require no pre-processing time, and answer each query using $ \mathcal{O}(\poly(\log n)) $ time, random bits, and additional space.

Next, we show how to implement random Catalan objects, specifically focusing on Dyck paths
(balanced random walks on the integer line that are always positive).
Here, we support $\func{Height}$ queries to find the location of the walk,
and $\func{First-Return}$ queries to find the time when the walk returns to a specified location.
This in turn can be used to implement $\func{Next-Neighbor}$ queries on random rooted and binary trees,
and $\func{Matching-Bracket}$ queries on random well bracketed expressions (the Dyck language).

Finally, we study random $q$-colorings of graphs with max degree $\Delta$.
In contrast to the prior settings, where random objects are generated according to a single distribution with $\mathcal O(1)$ parameters
(for example, $n$ and $p$ in the $G(n,p)$ model), the distribution here is specified via a ``huge'' input (in this case, the underlying graph).
When $q > \alpha\Delta$ for a small constant $\alpha$, we show how to answer queries to the color of any given node in sub-linear time.

\end{abstract}

\newpage
