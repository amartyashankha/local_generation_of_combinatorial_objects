\begin{abstract}

Consider a computation  on  a massive random graph: 
Does one need to generate the whole random graph up front,
prior to performing the computation?
Or, is it possible to provide an oracle to answer queries to the random graph
''on-the-fly'' in a much more efficient manner overall?
That is, to provide a \emph{local access generator}
which incrementally constructs the random graph locally, at the queried portions,
in a manner consistent with the random graph model and all previous choices.
Local access generators can be useful when studying
the local behavior of specific random graph models.
Our goal is to design local access generators whose
required resource overhead for answering each query
is significantly more efficient than generating the whole random graph.

Our results focus on undirected graphs with independent edge probabilities, that is,
each edge is chosen as an independent Bernoulli random variable.
We provide a general implementation for generators in this model.
Then, we use this construction to obtain the first efficient local implementations
for the Erd\"{o}s-R\'{e}nyi $G(n,p)$ model, and the Stochastic Block model.

As in previous local-access implementations for random graphs, we support
\func{Vertex-Pair}, \func{Next-Neighbor} queries, and \func{All-Neighbors} queries.
In addition, we introduce a new \func{Random-Neighbor} query.
We also give the first local-access generation procedure for \func{All-Neighbors}
queries in the (sparse and directed) Kleinberg's Small-World model.
Note that, in the sparse case, an \func{All-Neighbors} query
can be used to simulate the other types of queries efficiently.
All of our generators require no pre-processing time, and answer each query
using $ \mathcal{O}(\poly(\log n)) $ time, random bits, and additional space.

%rr Such an oracle for local access to the random
%graph can be useful when one only needs to query a small portion
%Local access is extensively used in sub-linear time algorithms and local algorithms,
%where the algorithms inspect the input graph by interacting with the oracle:
%optimizing the time per query (without any preprocessing overhead)
%would lead to more efficient implementations over random graphs.

%rr We give the first local-access procedures that 
%Our construction provides the first local-access generators for the
%Erd\"{o}s-R\'{e}nyi model ($G(n,p)$) and the Stochastic Block model,
%and also the first generators for \emph{dense graphs}.
%We support all three query types.
%\textcolor{red}{
%Note that in sparse graphs, an \func{All-Neighbors} query provides all three.
%}

\end{abstract}

\newpage
