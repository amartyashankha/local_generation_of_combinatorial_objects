\subsection{Random Catalan Objects}
\label{sec:overview_catalan_objects}
Many important combinatorial objects can be interpreted as Catalan objects.
One such interpretation is a Dyck path; a one dimensional random walk on the line with $n$ up and $n$ down steps, starting from the origin,
with the constraint that the height is always non-negative.
We implement $\func{Height}(t)$, which returns the position of the walk at time $t$,
and \func{First-Return}$(t)$, which returns the first time when the random walk returns to the same position as it was at time $t$.
These queries are natural for several types of Catalan objects.
As noted previously, we can use standard bijections to translate the Dyck path query implementations into
natural queries for \emph{bracketed expressions} and \emph{ordered rooted trees}.
Specifically, \func{Height} values in Dyck paths are equivalent to \emph{depth} in bracket expressions and trees.
The \func{First-Return} queries are more involved, and are equivalent to finding the \emph{matching bracket} in bracket expressions,
and alternately to finding the \emph{next child} of a node in an ordered rooted tree (see Section~\ref{sec:bijections_to_other_catalan_objects}).

Over the course of the execution, our algorithm will determine the height of a random Dyck path at many different positions $\{ x_1, x_2,\cdots, x_m\}$
(with $x_i<x_{i+1}$), both directly as a result of user given $\func{Height}$ queries, and indirectly through recursive calls to $\func{Height}$.
These positions divide the sequence into contiguous \emph{intervals} $[x_i,x_{i+1}]$,
where the height of the endpoints $y_i, y_{i+1}$ have been determined, but none of the intermediate heights are known.
The important observation is that the unknown section of the path within an \emph{interval}
is entirely determined by the positions and heights of the endpoints, and in particular is completely independent of all other \emph{intervals}.
So, each interval $[x_i,x_{i+1}]$ along with corresponding heights $\{y_i,y_{i+1}\}$,
represents a generalized Dyck problem with $U$ up steps, $D$ down steps,
and a constraint that the path never dips more than $y_i$ units below the starting height.

\paragraph*{$\func{Height}$ Queries}
\label{par:height_queries}
General $\func{Height}(x)$ queries can then be answered by recursively halving the \emph{interval} containing $x$,
by repeatedly sampling the height at the midpoint, until the height of position $x$ is sampled.
We start by implementing a subroutine that given an \emph{interval} $[x_i,x_{i+1}]$ of length $2B$, containing $2U$ up and $2D$ down steps,
determines the number of up steps $U'=U+d$ assigned to the first half of the \emph{interval}
(we parameterize $U'$ with $d$ in order to make the analysis cleaner).
Note that this is equivalent to answering the query $\func{Height}(x_i+B)$.
This is done by sampling the parameter $d$ from a distribution $\{ p_d\}$ where $p_d \equiv S_{left}(d)\cdot S_{right}(d)/S_{total}$.
Here, $S_{left}(d)$ (respectively $S_{right}(d)$) is the number of possible paths in the left (resp. right) half of the \emph{interval} when
$U+d$ up steps and $D-d$ down steps are assigned to the first half, and $S_{total}$ is the number of possible paths in the original $2B$-interval.

The problem of determining the number of up steps in the first half of the \emph{interval} was solved for the unconstrained case
(where the sequence is just a random permutation of up and down steps) in \cite{huge}.
Adding the non-negativity constraint introduces further difficulties as the distribution over $d$ has a CDF that is difficult to compute.
We construct a different distribution $\{q_d\}$ that approximates $\{p_d\}$ pointwise to a factor of $\log n$ and has an efficiently computable CDF.
This allows us to sample from $\{q_d\}$ and leverage rejection sampling techniques
(see Lemma~\ref{lem:rejection_sampling} in Section~\ref{sec:basic_tools_for_efficient_sampling}) to obtain samples from $\{p_d\}$.

\paragraph*{$\func{First-Return}$ Queries}
\label{par:_first-return_queries}
\todo[inline,color=red!80!green!25]{Why is first ret defined this way}
Recall that \func{First-Return}$(x)$ is only defined if the first step after position $x$ is upwards\footnote{
%The up and down steps in a Dyck path correspond directly to opening and closing braces in a well bracketed expression.
%The \func{First-Return} query returns the matching closing bracket for oprning bracket at position $x$ bracket.
%If the first step is down however, the symbol at position $x$ is actually a closing bracket itself
} (i.e. $\func{Height}(x+1) > \func{Height}(x)$).
\func{First-Return} queries are challenging because we first need to find the \emph{interval} containing the first return to $\func{Height}(x)$.
Since there could be up to $\Theta(n)$ intervals, it is inefficient to iterate through all of them.
To circumvent this problem, we allow each interval $[x_i,x_{i+1}]$ to sample and maintain its own boundary constraint $k_i$
instead of using the global non-negativity constraint.
A boundary constraint of $k_i$ implies that the path within the interval $[x_i,x_{i+1}]$ never reaches the height $y_i-k_i$ or lower.
Additionally, we maintain a crucial invariant that states that this boundary is achieved by the endpoint of lower height $\min(y_i,y_{i+1})$.
If this invariant holds, we can find the interval containing $\func{First-Return}(x)$ by finding the smallest known position $x_i>x$
whose sampled height $y_i \le \func{Height}(x)$, and considering the interval $[x_{i-1},x_i]$ preceding $x_i$.
We use an interval tree to update and query for the known heights.

Unfortunately, every $\func{Height}$ query creates new intervals by sub-dividing existing ones, potentially breaking the invariant.
We re-establish the invariant for $[x_i,x_{i+1}]$ by generating a ``mandatory boundary'' $h$
(a boundary constraint with the added restriction that some position within the interval \emph{must} touch the boundary),
%(a $y$-coordinate that must be achieved within the interval $[x_i,x_{i+1}]$ but not exceeded),
and then sampling a position $x_{mid}\in [x_i,x_{i+1}]$ such that $\func{Height}(x_{mid}) = h$ (Figure~\ref{fig:dyck_invariant_preserve}).
This creates new intervals $[x_i,x_{mid}]$ and $[x_{mid},x_{i+1}]$, both of which have a boundary constraint at $h$.

The first step of sampling the \emph{mandatory boundary} is performed
by binary searching on the possible boundary locations using an appropriate CDF (Section~\ref{sec:sampling_the_lowest_achievable_height}).
To find an intermediate position touching this boundary, we parameterize the position with $d$,
and find the distribution $\{p_d\}$ associated with the various possibilities.
Since we cannot directly sample from this complicated distribution, we define a \emph{piecewise continuous} probability distribution
$\hat q(\delta)$ such that $\hat q(\delta)$ approximates $p_{\floor\delta}$ (Section~\ref{sec:sampling_first_position_touching_mandatory_boundary}).
We then use this to define a discrete distribution $\{q_d\}$ where $q_d = \int_d^{d+1}\hat q(\delta)$,
where we can efficiently compute the CDF of $\{q_d\}$ by integrating the piecewise continuous $\hat q(\delta)$.
The challenge here is to construct an appropriate $\hat q(\delta)$ that only has $\mathcal O(\log^{\mathcal O(1)} n)$ continuous pieces.
This allows us to again use the rejection sampling technique (Lemma~\ref{lem:rejection_sampling}) to indirectly obtain a sample from $\{p_d\}$.
