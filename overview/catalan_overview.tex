\subsection{Random Catalan Objects}
\label{sec:overview_catalan_objects}
Consider a one dimensional random walk on the line with $n$ up and $n$ down steps, starting from the origin,
with a constraint that the height is always non-negative.
We implement two queries, $\func{Height}$ and \func{First-Return}, where $\func{Height}(t)$ which returns the position of the walk at time $t$,
and \func{First-Return}$(t)$ returns the first time when the random walk returns to the same position as it was at time $t$.

Over the course of the execution, our algorithm will sample the height of the walk at many different positions $\{ x_1, x_2,\cdots, x_m\}$
(with $x_i<x_{i+1}$), both directly as a result of user given $\func{Height}$ queries, and indirectly through recursive calls to $\func{Height}$.
These sampled positions divide the sequence into contiguous \emph{intervals} $[x_i,x_{i+1}]$,
where the height of the endpoints $y_i, y_{i+1}$ have been sampled, but none of the intermediate heights are known.
The important observation is that, since the beginning and ending heights are known,
the section of the path within an \emph{interval} is completely independent of all other \emph{intervals}.
So, each interval $[x_i,x_{i+1}]$ along with the corresponding heights $y_i,y_{i+1}$,
represents a generalized Dyck problem with $U$ up steps, $D$ down steps, and the constraint that the path never goes below $0$.

\paragraph*{$\func{Height}$ Queries}
\label{par:height_queries}
We start by implementing a subroutine that given an \emph{interval} $[x_i,x_{i+1}]$ of length of length $2B$ with $2U$ up and $2D$ down steps,
samples the number of up steps $U'=U+d$ to the first half of the \emph{interval}
(we parameterize $U'$ with $d$ in order to make the analysis cleaner).
Note that this is equivalent to answering the query $\func{Height}(x_i+B)$.
This is done by sampling the parameter $d$ from a distribution $\{ p_d\}$ with $p_d = S_{left}(d)\cdot S_{right}(d)/S_{total}$,
where $S_{left}(d)$ (respectively $S_{right}(d)$) is the number of possible paths in the left (resp. right) half of the \emph{interval} when
$U+d$ up steps and $D-d$ down steps are assigned to the first half, and $S_{total}$ is the number of possible paths in the original $2B$-interval.
General $\func{Height}(x)$ queries can then be answered by recursively halving the \emph{interval},
and sampling the height of the midpoint, until the height of $x$ is sampled.

The problem of sampling the number of up steps in the first half of the \emph{interval} was solved for the case where the sequence is fully random
in \cite{huge}.
Adding the non-negativity constraint introduces further difficulties as the distribution over $d$ has a CDF that is difficult to compute.
We construct a different distribution $\{q_d\}$ that approximates $\{p_d\}$ pointwise to a factor of $\log n$ and has an efficiently computable CDF.
This allows us to sample from $\{q_d\}$ and leverage the rejection sampling lemma (Lemma~\ref{lem:rejection_sampling}) to obtain samples from $\{p_d\}$.

\paragraph*{$\func{First-Return}$ Queries}
\label{par:_first-return_queries}
$\func{First-Return}$ queries present an additional challenge because we don't know which \emph{interval} contains the first return.
Since there could be up to $\Theta(n)$ intervals, is it inefficient to iterate through all of them.
To circumvent this problem, we allow each interval to sample its own boundary constraint $k>0$ instead of using the global non-negativity constraint.
A boundary constraint of $k$ implies that the path within the interval $[x_i,x_{i+1}]$ never reaches the height $y_i-k$ or lower.
Additionally, we maintain an invariant that states that this boundary $x_i-k$ coincides with $\min(x_i,x_{i+1})$.
If this constraint is satisfied, we can find the interval containing $\func{First-Return}(x)$ by finding the smallest sampled position $x_i>x$
whose sampled height $y_i \le \func{Height}(x)$, and considering the interval $[x_{i-1},x_i]$ preceding $x_i$.

Every time the $\func{Height}$ algorithm creates new intervals by sub-dividing an existing one, this invariant is potentially broken.
We re-establish it by sampling a ``mandatory boundary''
(a boundary constraint with the additional restriction that some position within the interval $[x_i,x_{i+1}]$ \emph{must} touch the boundary),
%(a $y$-coordinate that must be achieved within the interval $[x_i,x_{i+1}]$ but not exceeded),
and then sampling a position $x$ such that $x_i < x < x_{i+1}$ and $\func{Height}(x) = y$.
The first step of sampling the mandatory boundary is performed by binary searching on the possible boundary locations.
To find a position that touches this boundary, we parameterize the position with $d$ and find the distribution $\{p_d\}$ associated with these events.
We then define a piecewise continuous PDF $\hat q(\delta)$ such that $\hat q(\delta)$ approximates $p_{\floor\delta}$.
We then use this to construct $q_d = \int_d^{d+1}\hat q(\delta)$, where the CDF of $q_d$ is efficently computable using integration,
and use rejection sampling (Lemma~\ref{lem:rejection_sampling}) again to sample indirectly from $\{p_d\}$.
