\section{Our Contributions and Techniques}
\label{sec:our_contributions_and_techniques}

The problem of computing local information of huge random objects was pioneered in \cite{huge_old,huge}.
Further work of \cite{sparse} considers the generation of sparse random $G(n,p)$ graphs from the Erd\"{o}s-R\'{e}nyi model \cite{er},
with $p = O(\poly(\log n)/n)$, which answers $\poly(\log n)$ \func{All-Neighbors} queries, listing the neighbors of queried vertices.
While these implementations use polylogarithmic resources over their entire execution,
they generate graphs that are  only guaranteed to {\em appear random} to algorithms that inspect a {\em limited portion} of the generated graph.
For example, the greedy routing algorithm on Kleinberg's small world networks \cite{kleinberg} only uses $\mathcal O(\log^2 n)$ probes.
Using our implementation, one can execute this algorithm on a random small world instance
in $\mathcal O(\poly(\log n))$ time without incurring the $\mathcal O(n)$ prior-sampling overhead.

In \cite{reut}, the authors construct an oracle for the generation of recursive trees, and BA preferential attachment graphs.
Unlike \cite{sparse}, their implementation allows for an arbitrary number of queries.
This result is particularly interesting --  although the graphs in this model are generated via a sequential process,
the oracle is able to locally generate arbitrary portions of it and answer queries in polylogarithmic time.
Though preferential attachment graphs are sparse, they contain vertices of high degree,
thus \cite{reut} provides access to the adjacency list through \func{Next-Neighbor} queries.




\section{Preliminaries}
\label{sec:model}
\subsection{Local-Access Generators}
\label{sec:oracle_model}

We begin by formalizing a model of {\em local-access generators} (Section~\ref{sec:oracle_model}), implicitly used in \cite{reut}.
\todo{Not good}
Our work provides local-access generators for various basic classes of graphs described in the following, with 
\func{Vertex-Pair}, \func{Next-Neighbor}, and \func{Random-Neighbor} queries.
In all of our results, each query is processed using $\poly(\log n)$ time, random bits, and additional space, with \emph{no initialization overhead}.
These guarantees hold even in the case of adversarial queries.
Our bounds assume constant computation time for each arithmetic operation with $O(\log n)$-bit precision.
Each of our generators constructs a random graph drawn from a distribution that is $1/\poly(n)$-close to the desired distribution in the $L_1$-distance.
\footnote{The \emph{$L_1$-distance} between two probability distributions $\distr{p}$ and $\distr{q}$ over domain $D$
is defined as $\|\distr{p-q}\|_1 = \sum_{x \in D } |p(x)-q(x)|$.
We say that $\distr{p}$ and $\distr{q}$ are $\epsilon$-close if $\|\distr{p-q}\|_1 \leq \epsilon$.}

We consider the problem of locally generating random graphs $G = (V,E)$ drawn from the desired families of simple unweighted graphs, undirected or directed. We denote the number of vertices $n = |V|$, and refer to each vertex simply via its unique ID from $[n]$. For undirected $G$, the set of neighbors of $v \in V$ is defined as $\Gamma(v) = \{u \in V: \{v,u\} \in E\}$; denote its degree by $\deg(v) = |\Gamma(v)|$.
Inspired by the goals and results of \cite{reut}, we define a model of local-access generators as follows.
\begin{definition}
A \emph{local-access generator} of a random graph $G$ sampled from a distribution $\distr{D}$,
is a data structure that provides access to $G$ by answering various types of
\emph{supported queries}, while satisfying the following:
%For clarity, we assume that the generator is invoked until its entire graph $G$ is exposed.
%The local-access generator for a probability distribution $\distr{D}$ of the desired random graph model must satisfy the following properties:
\begin{itemize}
\item \textbf{Consistency.} The responses of the local-access generator to all probes throughout the entire execution must be consistent with a single graph $G$.
\item \textbf{Distribution equivalence.} 
The random graph $G$ provided by the generator must be sampled from some distribution $\distr{D}'$
that is $\epsilon$-close to the desired distribution $\distr{D}$ in the $L_1$-distance.
In this work we focus on supporting $\epsilon = n^{-c}$ for any desired constant $c>0$.
As for \func{Random-Neighbor}$(v)$, the distribution from which a neighbor is returned
must be $\epsilon$-close to the uniform distribution over neighbors of $v$
with respect to the sampled random graph $G$ (w.h.p $1-n^{-c}$ for each query).
\item \textbf{Performance.} The resources, consisting of (1) computation time, (2) additional random bits required, and (3) additional space required, in order to compute an answer to a single query and update the data structure, must be sub-linear, preferably $\poly(\log n)$.
\end{itemize}
\end{definition}
In particular, we allow queries to be made adversarially and non-deterministically. The adversary has full knowledge of the generator's behavior and its past random bits.

For ease of presentation, we allow generators to create graphs with self-loops.
When self-loops are not desired, it is sufficient to add a wrapper function that simply re-invokes \func{Next-Neighbor}$(v)$ or \func{Random-Neighbor}$(v)$ when the generator returns $v$.


\paragraph*{Supported Queries in our Model} 
For undirected graphs, we consider queries of the following forms.
now we might want to do \func{Next-Neighbor} first for consistency.
\begin{itemize}
\item \func{Next-Neighbor}$(v)$: The generator returns the neighbor of $v$ with the lowest ID that has not been returned during the execution of the generator so far. If all neighbors of $u$ have already been returned, the generator returns $n+1$.
\item \func{Random-Neighbor}$(v)$: The generator returns a neighbor of $v$ uniformly at random (with probability $1/\deg(v)$ each). If $v$ is isolated, $\bot$ is returned.
\item \func{Vertex-Pair}$(u,v)$: The generator returns either $\ONE$ or $\ZERO$, indicating whether $\{u,v\}\in E$ or not.
\item \func{All-Neighbors}$(v)$: The generator returns the entire list of out-neighbors of $v$. We may use this query for relatively sparse graphs, specifically in the Small-World model.
\end{itemize}

\subsection{Random Graph Models}
\label{sec:graph_model}

\paragraph*{Erd\"{o}s-R\'{e}nyi Model}
We consider the $G(n, p)$ model: each edge $\{u,v\}$ exists independently with probability $p \in [0, 1]$.
Note that $p$ is not assumed to be constant, but may be a function of $n$.

\paragraph*{Stochastic Block Model}
This model is a generalization of the Erd\"{o}s-R\'{e}nyi Model. The vertex set $V$ is partitioned into $r$ communities $C_1, \ldots, C_r$. The probability that the edge $\{u,v\}$ exists is $p_{i,j}$ when $u\in C_i$ and $v\in C_j$, where the probabilities are given as an $r\times r$ symmetric matrix $\matr{P} = [p_{i,j}]_{i,j \in [r]}$.
We assume that we are given explicitly the distribution $\distr{R}$ over the communities, and
each vertex is assigned its community according to $\distr{R}$ independently at random.\footnote{Our algorithm also supports the alternative specification where the community sizes $\langle |C_1|, \ldots, |C_r|\rangle$ are given instead, where the assignment of vertices $V$ into these communities is chosen uniformly at random.}
%The main difficulty here is being able to compute the uniformly sampled assignment of vertices to communities on-the-fly.

\paragraph*{Small-World Model}
In this model, each vertex is identified via its 2D coordinate $v = (v_x, v_y) \in [\sqrt{n}]^2$. Define the Manhattan distance as $\func{dist}(u,v)=|u_x-v_x|+|u_y-v_y|$, and the probability that each directed edge $(u,v)$ exists is $c/(\func{dist}(u,v))^{2}$. Here, $c$ is an indicator of the number of long range directed edges present at each vertex. A common choice for $c$ is given by normalizing the distribution so that there is exactly one directed edge emerging from each vertex ($c = \Theta(1/\log n)$).
We will however support a range of values of $c=\log^{\pm\Theta(1)}n$.
While not explicitly specified in the original model description of \cite{kleinberg}, we assume that the probability is rounded down to $1$ if $c/(\func{dist}(u,v))^{2} > 1$.


\subsection{Miscellaneous}

\paragraph*{Arithmetic operations} Let $N$ be a sufficiently large number of bits required to maintain a multiplicative error of at most a $\frac{1}{\poly(n)}$ factor over $\poly(n)$ elementary computations ($+, -, \cdot, /, \exp$).\footnote{In our application of $\exp$, we only compute $a^b$ for $b \in \mathbb{Z}^+$ and $0 < a \leq 1+\Theta(\frac{1}{b})$, where $a^b = \Bo(1)$. For this, $N=\Bo(\log n)$ bits are sufficient to achieve the desired accuracy, namely an additive error of $n^{-c}$.} We assume that each elementary operation on words of size $N$ bits can be performed in constant time. Likewise, a random $N$-bit integer can be acquired in constant time. We assume that the input is also given with $N$-bit precision. %Any other non-trivial operations will be justified later.

\paragraph*{Sampling via a CDF}\label{para:CDF}
Consider a probability distribution $\distr{X}$ over $O(n)$ consecutive integers, whose cumulative distribution function (CDF) for can be computed with at most $n^{-c}$ additive error for constant $c$.
Using $\Bo(\log n)$ CDF evaluations, one can sample from a distribution that is
$\frac{1}{\poly(n)}$-close to $\distr{X}$ in $L_1$-distance.\footnote{Generate a random $N$-bit number $r$, and binary-search for the smallest domain element $x$ where $\mathbb P[X\leq x] \geq r$.}

\subsection{Undirected Graphs}
\label{sec:undirected_graphs}
In Section~\ref{sec:undirected}, we implement queries to both the adjacency matrix and adjacency list representation
for the generic class of \emph{undirected graphs} with {\em independent edge probabilities} $\left\{ p_{uv} \right\}_{u,v\in V}$,
where $p_{uv}$ denotes the probability that there is an edge between $u$ and $v$.
Throughout, we identify our vertices via their unique IDs from $1$ to $n$, namely $V = [n]$.
We implement \func{Vertex-Pair}, \func{Next-Neighbor}, and \func{Random-Neighbor}
\footnote{\func{Vertex-Pair}$(u,v)$ returns whether $u$ and $v$ are adjacent, \func{Next-Neighbor}$(v)$ returns a new neighbor of $v$ each time
it is invoked (until none is left), and \func{Random-Neighbor}$(v)$ returns a uniform random neighbor of $v$ (if $v$ is not isolated).} queries.
Under reasonable assumptions on the ability to compute certain values pertaining to consecutive edge probabilities,
our implementations support all three types of queries using $\mathcal{O}(\poly(\log n))$ time, space, and random bits.
In particular, our construction yields local-access generators for the Erd\"{o}s-R\'{e}nyi $G(n,p)$ model (for \emph{all} values of $p$),
and the Stochastic Block model with random community assignment.
As in \cite{reut} (and unlike the generators in \cite{huge_old,huge,sparse}), our techniques allow unlimited queries.

While \func{Vertex-Pair} and \func{Next-Neighbor} queries, as well as \func{All-Neighbors} queries for sparse graphs,
have been considered in the prior works of \cite{reut, huge_old, huge, sparse}, we provide the first implementation (to the best of our knowledge)
of \func{Random-Neighbor} queries, which do not follow trivially from the \func{All-Neighbor} queries in \emph{non-sparse graphs}.
Such queries are useful, for instance, for sub-linear algorithms that employ random walk processes.
\func{Random-Neighbor} queries present particularly interesting challenges that are outlined below.

\paragraph*{\func{Next-Neighbor} Queries}
\label{par:next_neighbor_queries}
We note that the next neighbor of a vertex can be found trivially by generating consecutive entries of the adjacency matrix,
but for small edge probabilities $p_{uv} = o(1)$ this implementation can be too slow.
In our algorithms, we achieve speed-up by sampling multiple neighbor values at once for a given vertex $u$; more specifically,
we sample for the number of ``non-neighbors'' preceding the next neighbor.
To do this, we assume that we have access to an oracle which can estimate the ``skip'' probabilities $F(v,a,b)=\prod^{b}_{u=a} (1-p_{v,u})$,
where $F(v,a,b)$ is the probability that $v$ has no neighbors in the range $[a,b]$.
We later show that it is possible to compute this quantity efficiently for the $G(n,p)$ and Stochastic block models.

A main difficulty in our setup, as compared to \cite{reut}, arises from the fact that our graph is undirected, and thus
we must design a data structure that ``informs'' all (potentially $\Theta(n)$) non-neighbors once we decide on the query vertex's next neighbor.
%Unlike in the case of directed graphs, each $\func{Next-Neighbor}$ query
%can affect the probabilities of a large number of other vertices.
More concretely, if $u'$ is sampled as the next neighbor of $v$ after its previous neighbor $u$,
we must maintain consistency in subsequent steps by ensuring that none of the vertices in the range $(u,u')$ return $v$ as a neighbor.
This update will become even more complicated as we later handle \func{Random-Neighbor} queries, where we may generate non-neighbors at random locations.

In Section~\ref{sec:ER-rand}, we present a very simple randomized generator (Algorithm~\ref{alg:oblivious-coin-toss})
that supports \func{Next-Neighbor} queries efficiently, albeit the analysis of its performance is rather complicated.
We remark that this approach may be extended to support \func{Vertex-Pair} queries with superior performance
(given that we do not to support \func{Random-Neighbor} queries) and to provide deterministic resource usage guarantee
\todo{What is deterministic here?}
-- the full analysis can be found in Section~\ref{sec:reroll-cont} and \ref{sec:ER-det}, respectively.

\paragraph*{\func{Random-Neighbor} Queries}
\label{par:random_neighbor_queries}

We provide efficient \func{Random-Neighbor} queries (Section~\ref{sec:buckets}).
The ability to do so is surprising since:
(1) \func{Random-Neighbor} queries affect the conditional probabilities of the remaining neighbors in a non-trivial manner
\footnote{\label{conditional}Consider a $G(n,p)$ graph with small $p$, say $p = 1/\sqrt n$,
such that vertices will have $\tilde{\mathcal{O}}(\sqrt n)$ neighbors with high probability.
After $\tilde{\mathcal{O}}(\sqrt n)$ \func{Random-Neighbor} queries, we will have uncovered all the neighbors (w.h.p.),
so that the conditional probability of the remaining $\Theta(n)$ edges should now be close to zero.}, and
(2) our implementation does not resort to explicitly sampling the degree of any vertex $v$
in order to generate a random neighbor with the correct probability $\frac{1}{d_v}$.
First, even without committing to the degrees, answers to \func{Random-Neighbor} queries
affect the conditional probabilities of the remaining adjacencies in a global and non-trivial manner \footref{conditional}
-- that is, from the point of view of the \emph{agent} interacting with the generator.
Second, sampling the degree of the query vertex, we suspect, is not viable for \emph{sub-linear} generators,
because this quantity alone imposes dependence on the existence of \emph{all} of its potential incident edges.
Therefore, our generator needs to return a random neighbor, with probability reciprocal to the query vertex's degree,
without resorting to ``knowing'' its degree.
The generator, however, must somehow maintain and leverage its additional \emph{internal knowledge}
of the partially-generated graph, to keep its computation tractable throughout the entire graph generation process.
This requires a way of implicitly keeping track of all the resulting changes.

We formulate a {\em bucketing approach} (Section~\ref{sec:buckets})
which samples multiple consecutive edges at once, in such a way
that the conditional probabilities of the unsampled edges
remain independent and ``well-behaved'' during subsequent queries.
For each vertex $v$, we divide the vertex set (potential neighbors) or $v$ into consecutive ranges (buckets),
so that each bucket contains, in expectation, roughly the same number of neighbors
$\sum^{b}_{u=a} p_{v,u}$ (which we must be able to compute efficiently).
The subroutine of \func{Next-Neighbor} may be applied to sample the neighbors within a bucket in expected constant time.
Then, one may obtain a random neighbor of $v$ by picking a random neighbor from a random bucket;
probabilities of picking any neighbors may be normalized to the uniform distribution via rejection sampling,
while stilling yielding $\poly(\log n)$ complexities overall.
This bucketing approach also naturally leads to our data structure that requires
constant space for each bucket and for each edge, using $\Theta(n+m)$ overall memory requirement.
The \func{Vertex-Pair} queries are implemented by sampling the relevant bucket.

We now consider the application of our construction above to actual random graph models,
where we must realize the assumption that $\prod^{b}_{u=a} (1-p_{v,u})$
and $\sum^{b}_{u=a} p_{v,u}$ can be computed efficiently.
This holds trivially for the $G(n,p)$ model via closed-form formulas,
but requires an additional back-end data structure for the Stochastic Block models.

\paragraph*{Erd\"{o}s-R\'{e}nyi}
\label{par:erdos_renyi}
In Section~\ref{sec:app_er}, we apply our construction to random $G(n,p)$ graphs for
arbitrary $p$, and obtain
$\func{Vertex-Pair}$, $\func{Next-Neighbor}$, and $\func{Random-Neighbor}$ queries,
using polylogarithmic resources (time, space and random bits) per query.
We remark that, while $\Omega(n+m) = \Omega(p n^2)$ time and space
is clearly necessary to generate and represent a full random graph,
our implementation supports local-access via all three types of queries, 
and yet can generate a full graph in $\widetilde{O}(n+m)$ time and space (Corollary~\ref{thm:er-optimal}),
which is tight up to polylogarithmic factors.

%{\color{red}
\paragraph*{Stochastic Block Model}
\label{par:stochastic_block_model}
We generalize our construction to the Stochastic Block Model.
In this model, the vertex set is partitioned into $r$ \emph{communities}
$\left\{ C_1, \ldots, C_r \right\}$.
The probability that an edge exists depends on the communities of its endpoints:
if $u\in C_i$ and $v \in C_j$, then $\{u,v\}$ exists with probability $p_{i,j}$,
given in an $r\times r$ matrix $\matr{P}$.
As communities in the observed data are generally unknown a priori,
and significant research has been devoted to designing efficient algorithm
for community detection and recovery,
these studies generally consider the \emph{random community assignment} condition for the purpose of designing and analyzing algorithms (see e.g., \cite{mossel2015reconstruction}).
Thus, in this work, we aim to construct generators for this important case, where the community assignment of vertices are independently sampled from some given distribution $\distr{R}$.
%}

Our approach is, as before, to sample for the next neighbor or a random neighbor directly,
although our result does not simply follow closed-form formulas,
as the probabilities for the potential edges now depend
on the communities of endpoints.
To handle this issue, we observe that it is sufficient to efficiently count
the number of vertices of each community in any
range of contiguous vertex indices.
We then design a data structure extending a construction of \cite{huge},
which maintain these counts for ranges of vertices,
and ``sample'' the partition of their counts only on an as-needed basis.
This extension results in an efficient technique to sample counts
from the \emph{multivariate hypergeometric distribution} (Section~\ref{sec:multivariate_hypergeometric_sampling}).
This sampling procedure may be of independent interest.
For $r$ communities, this yields an implementation with
$ \mathcal{O}(r\cdot \poly(\log n))$ overhead in required resources for each operation.
This upholds all previous polylogarithmic guarantees when $r = \poly(\log n)$.

\paragraph*{CDF Based Sampling}
It is worth noting that our techniques for implementing local-access for the ER and SBM graphs
can easily be extended to other similar models of random graphs.
The only requirement is that the CDF of the probability sequences can be efficiently computed as in Section~\ref{para:CDF}.





\subsection{Random Catalan Objects}
\label{sec:overview_catalan_objects}
We consider a one dimensional random walk with $n$ up and $n$ down steps starting from the origin,
with a boundary constraint that the height after $t$ steps is always non-negative.
Alternately, we can view this as a random sequence (permutation) of $n$ black (corresponding to $+1$) and $n$ white balls (corresponding to $-1$),
with the restriction that the number of black balls is always at least the number of white balls in any prefix of the sequence.

Over the course of the excecution, our algorithm will sample the height of the walk at many different positions $\{ x_1, x_2,\cdots, x_m\}$
(with $x_i<x_{i+1})$) both directly as a result of user given $\func{Height}$ queries, and indirectly through recursive calls to $\func{Height}$.
These sampled positions divide the sequence into contiguous \emph{intervals} $[x_i,x_{i+1}]$,
where the height of the endpoints $y_i, y_{i+1}$ have been sampled, but non of the intermediate heights are known.
The important observation is that the section of the path within an \emph{interval} is completely independent of all other \emph{intervals}.
So, each interval $[x_i,x_{i+1}]$ represents a generalized Dyck problem with $U$ up steps, $D$ down steps and the boundary constraint that
for any prefix of the interval, the number of up steps can be at most $x_i$ smaller than the number of down steps.

Another useful fact is that the \emph{imbalance} (difference in number of up and down steps) in any contiguous sub-interval of size $B$
is bounded by $\mathcal O(\sqrt{B\log n})$ with high probability.

\paragraph*{$\func{Height}$ Queries}
\label{par:height_queries}
We start by implementing a subroutine that given an \emph{interval} $[x_i,x_{i+1}]$ of length of length $2B$ with $2U$ up and $2D$ down steps,
samples the number of up steps $U+d$ to the first half of the \emph{interval}, which effectively answers the query $\func{Height}(x_i+B)$.
This is done by sampling the parameter $d$ from a distribution $\{ p_d\}$ with $p_d = S_{left}(d)\cdot S_{right}(d)/S_{total}$,
where $S_{left}(d)$ (respectively $S_{right}(d)$) is the number of possible paths in the left (resp. right) half of the \emph{interval} when
$U+d$ up steps and $D-d$ down steps are assigned to the first half, and $S_{total}$ is the number of possible paths in the original $2B$-interval.
General $\func{Height}(x)$ queries can then be answered by recursively halving the \emph{interval} containing $x$ and performing binary search.

The problem of sampling the number of up steps in the first half of the \emph{interval} was solved for the case where the sequence is fully random
in \cite{huge}.
Adding the non-negativity constraint introduces further difficulties as the distribution over $d$ has a CDF that is difficult to compute.
We construct a different distribution $\{q_d\}$ that approximates $\{p_d\}$ pointwise to a factor of $\log n$ and has an efficiently computable CDF.
This allows us to sample from $\{q_d\}$ and leverage the rejection sampling lemma (Lemma~\ref{lem:rejection_sampling}) to obtain samples from $\{p_d\}$.

\paragraph*{$\func{First-Return}$ Queries}
\label{par:_first-return_queries}
$\func{First-Return}$ queries present an additional challenge because we don't know which \emph{interval} contains the first return.
Since there could be up to $\Theta(n)$ intervals, is it inefficient to iterate through all of them.
To circumvent this problem, we allow each interval to sample it's own boundary constraint $k>0$ instead of using the global non-negativity constraint.
A boundary constraint of $k$ implies that the path within the interval $[x_i,x_{i+1}]$ never reaches the height $y_i-k$ or lower.
Additionally, we maintain an invariant that states that this boundary $x_i-k$ coincides with $\min(x_i,x_{i+1})$.
If this constraint is satisfied, we can find the interval containing $\func{First-Return}(x)$ by finding the smallest sampled position $x_i>x$
whose sampled height $y_i \le \func{Height}(x)$, and considering the interval $[x_{i-1},x_i]$ preceding $x_i$.

Every time the $\func{Height}$ algorithm creates new intervals by sub-dividing an existing one, this invariant is potentially broken.
We re-establish it by sampling a ``mandatory boundary'' (a $y$-coordinate that must be achieved within the interval $[x_i,x_{i+1}]$ but not exceeded),
and then sampling a position $x$ where $x_i < x < x_{i+1}$ and $\func{Height}(x) = y$.
The first step of sampling the mandatory boundary is performed by binary searching on the possible boundary locations.
To find a position that touches this boundary, we parameterize the position with $d$ and find the distribution $\{p_d\}$ associated with these events.
We then define a piecewise continuous PDF $\hat q(\delta)$ such that $\hat q(\delta)$ approximates $p_{\floor\delta}$.
We then use this to construct $q_d = \int_d^{d+1}\hat q(\delta)$,
and use rejection sampling (Lemma~\ref{lem:rejection_sampling}) again to sample indirectly from $\{p_d\}$.




\subsection{Random Coloring of a Graph}
\label{sec:overview_random_coloring_of_a_graph}
Finally, we introduce a new model for implementating huge random objects
where the distribution is specified as a uniformly random solution to a huge combinatorial problem.
In all the problems we have considered so far as well as the ones studied in prior work \cite{huge,sparse,reut}, the description size
of the random object is small (typically $\mathcal O(\log n)$ to represent the size of the instance and a constant number of parameters).
In this new setting, we will implement local query access to random $q$-colorings of a given huge graph $G$ of size $n$ with maximum degree $\Delta$.
The distribution in this case is defined by the graph structure which has size $\mathcal O(n\Delta)$.
We present the following definition for local access implementations in this setting.

\begin{definition}
\label{def:local_access_LCA}
Given a combinatorial problem on graphs,
a \emph{local access implementation} of a family of query functions $\langle F_1, F_2,\cdots \rangle$,
provides an oracle $\mathcal A$ with the following properties.
$\mathcal A$ has query access to a graph $G$, and a tape of random bits $\vec R$.
Assuming that the solution set of the combinatorial problem on $G$ is $\mathbb X$,
$\mathcal A$ upon being queried with $F_i$, returns the value $F_i(X)$ for a specific solution $X\in\mathbb X$ where the choice of $X$
depends only on $\vec R$ and the distribution of $X$ (over $\vec R$) is $\epsilon$-close to the uniform distribution over $\mathbb X$.
Two different instances of $\mathcal A$ with the same graph oracle and the same random bits,
must agree on the choice of $X$ that is consistent with all answered queries regardless of what queries were actually asked.
\end{definition}

We can contrast this definition with the one for \emph{Local Computation Algorithms} \cite{LCA}
which also allow query access to \emph{some} valid solution and can read the input through local probes.
An additional difficulty in our setting is that we also have to make sure that we return a solution from the correct distribution.
Similarly to LCAs, we can have multiple independent instances of our algorithm answering different queries, but remaining consistent with one another.


\paragraph*{$\func{Color}$ queries}
\label{par:color_queries}
We are able to construct an efficient implementation for $\func{Color}(v)$ that returns the final color of $v$
in a uniformly random $q$ coloring of $G$ using only a sub-linear number of probes when $q\ge 9\Delta$.
Usually, a random coloring of a graph is sampled by proposing a random color update for a random vertex,
and accepting the update if it does not create a conflict and repeating this process $\mathcal O(n\log n)$ times.
This is an inherently sequential process with the acceptance of a particular proposal depending on all preceding neighboring proposals.

To make the runtime analysis simpler, we define a modified version of Glauber Dynamics that proceeds in $\mathcal O(\log n)$ epochs.
In each epoch, all the vertices propose a random color and update themselves if their proposals do not conflict with any of their neighbors.
This Markov Chain is a special case of the one presented in \cite{mohsen} for distributed graph coloring
and mixes in $\mathcal O(\log n)$ epochs when $q\ge 9\Delta$.
While we do not have the same restrictions as the distributed computation setting,
we choose to use this chain so as to avoid long analysis of mixing times.
In order to implement $\func{Color}(v)$, it suffices to implement a query $\func{Accepted}(v,t)$
that indicates whether the proposal for $v$ was accepted in the $t^{th}$ epoch.
The answer to this question depends on the prior colors of the potentially $\Delta$ neighbors of $v$.
Naively sampling the colors of all these neighbors would result in $\Delta$ recursive invocations on the previous epoch ($t-1$),
and stepping \emph{backwards} through the epochs to find the last accepted proposal.
This leads to a bound of $\Delta^t$ on the number of recursive invocations.

We can improve this somewhat by only considering neighbors $w$ of $v$ who had any proposal for the same color $c$.
In this case the expected number of recursive calls is bounded by $t\Delta/q$ ($t\Delta$ proposals to consider and each one is $c$ w.p. $1/q$).
So, if $q > t\Delta = \mathcal O(\Delta\log n)$, this allows us to bound the total number of resulting invocations.
The improvement to $q\ge 9\Delta$ comes from the observation that for $w\in\Gamma(v)$ such that $w$ proposed color $c$ at epoch $t'$,
the recursive call for $w$ can jump to epoch $t'$ and then step \emph{forwards} through the epochs to find the first accepted proposal.
Since the expected value of $t'$ is close to $t/2$, we dramatically reduce the sub-problem size in each recursion,
and this allows us to bound the runtime by $\mathcal O\left(t\Delta n^{6.12\Delta/q}\right)$ which is sub-linear for $q \ge 9\Delta$.
