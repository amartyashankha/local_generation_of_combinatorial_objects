\section{Model and Overview of our Techniques}
\label{sec:overview_of_our_techniques}
We begin by formalizing our model of \emph{local-access implementations}, inspired by \cite{reut},
but we add some aspects that were not addressed in earlier models.
First, we define families of random objects.

\begin{definition}
\label{def:parametrized_random_object}
A \textbf{random object family} maps a description $\Pi$ to a distribution $\mathsf X^{\Pi}$ over the set $\mathbb X^{\Pi}$.
\end{definition}

For example, the \emph{family} of Erd\"{o}s-R\'{e}nyi graphs maps $\Pi = (n, p)$ to a distribution over $\mathbb X^{\Pi}$,
which is the set of all possible $n$-vertex graphs,
where the probability assigned to any graph containing $m$ edges in the distribution $\mathsf X^{\Pi}$ is exactly $p^m\cdot (1-p)^{\binom{n}{2}-m}$.

\begin{definition}
\label{def:local_access}
Given a \textbf{random object family} $\{(\mathsf X^{\Pi}, \mathbb X^{\Pi})\}$ parameterized by $\Pi$, a \emph{local access implementation}
%Given a distribution $\mathsf X^{\Pi}$ over a set of huge random objects $\mathbb X$, a \emph{local access implementation}
of a family of query functions $\langle F_1, F_2,\cdots \rangle$ where $F_i: \mathbb X^{\Pi}\rightarrow \{0,1\}$,
provides an oracle $\mathcal M$ with an internal state for storing a partially generated random object.
Given a description $\Pi$ and a query $F_i$, the oracle returns the value $\mathcal M(\Pi, F_i)$,
and updates its internal state, while satisfying the following:
\begin{itemize}
    \item \textbf{Consistency:}
    There must be a single $X\in \mathbb X^{\Pi}$, such that for all queries $F_i$ presented to the oracle,
    the returned value $\mathcal M(\Pi,F_i)$ equals the true value $F_i(X)$.
    %must be consistent with a single $X\in \mathbb X^{\Pi}$.
    \item \textbf{Distribution equivalence:}
    The random object $X\in \mathbb X^{\Pi}$ consistent with the responses $\{ F_i(X)\}$ must be sampled from some distribution $\hat{\mathsf{X}}^{\Pi}$
    that is $\epsilon$-close to the desired distribution $\mathsf{X}^{\Pi}$ in $L_1$-distance.
    In this work, we focus on supporting $\epsilon = n^{-c}$ for any desired constant $c>0$.
    \item \textbf{Performance:}
    The computation time, and random bits required to answer a single query must be sub-linear in $|X|$ with high probability,
    without any initialization overhead.
\end{itemize}
In particular, we allow queries to be made adversarially and non-deterministically.
The adversary has full knowledge of the algorithm's behavior and its past random bits.
\end{definition}

For example, in the $G(n,p)$ family with description $\Pi = (n, p)$,
we can define \func{Vertex-Pair} query functions $\{ F_{(u,v)}\}_{u,v\in [n]}$.
So, given a graph $G\in \mathbb X^{\Pi}$, the query $F_{(u,v)}(G) = 1$ if and only if $(u,v)\in G$.

In prior work \cite{reut, huge, sparse} as well as some of our results, the input description $\Pi$ is of small size (typically $\mathcal O(\log n)$),
and the oracle $\mathcal M$ can read all of $\Pi$ (for example, $\Pi = (n, p)$ in $G(n,p)$).

\subparagraph*{{\Large Distributions with Huge Description Size:}}
\label{par:distributions_with_huge_description_size}
We initiate the study of \textbf{random object families} where the description $\Pi$ is too large to be read by a sub-linear time algorithm.
%In all the problems considered in prior work \cite{huge,sparse,reut}, the description size
%of the random object is small, typically $\mathcal O(\log n)$ to represent the size of the instance and a constant number of parameters.
%For instance, the $G(n, p)$ model is described using two parameters $n$ and $p$.
%However, if one wishes to implement local access to the uniform distribution over all valid colorings of a given input graph $G$,
In this setting, the oracle from Definition~\ref{def:local_access} cannot read the entire input $\Pi$, and instead accesses it through local probes.
For instance, consider the \textbf{random object family} that maps a graph $G$ to the uniform distribution over valid colorings of $G$.
Here, the description $\Pi$ includes the entire graph $G$, which is too large to be read by a sub-linear time algorithm.
In this case, the oracle can query the underlying graph using neighborhood probes.
The number of such probes used to answer a single query must be sub-linear in the input size.


\paragraph*{Supporting Independent Query Oracles: Memory-less Implementations}
\label{par:supporting_independent_query_oracles_memory_less_implementations}
The model in Definition~\ref{def:local_access} only supports sequential queries,
since the response to a future query may depend on the changes in internal state caused by past queries.
In some applications, we may want to have multiple independent query oracles whose responses are all consistent with each other.
One way to achieve this is to restrict our attention to \emph{memory-less} implementations; ones without any internal state.
An important implication of being memory-less is that the responses to each query is oblivious to the order of queries being asked.
In fact, the lack of internal state implies that independent implementations that use the same random bits and the same input description
must respond to queries in the same way.
Thus, instead of using the internal state to maintain consistency, memory-less implementations are given access to the same public random oracle.

%In addition to the above generalization of Definition~\ref{def:local_access}, we also consider ``memory-less'' oracle implementations,
%where the oracle does not have any internal state, and instead has access to a source of common randomness.
%An important implication of being memory-less is that the responses to each query is oblivious of the order of queries being asked.
%In fact, the lack of internal state implies that independent implementations that use the same common randomness and the same input description
%must respond to queries in the same way.
%So, we can have multiple (and even simultaneous) copies of the implementation,
%that are all consistent with a single random object drawn from the distribution,
%without having to write to shared memory or communicate in any other way.

For the problem of sampling a random graph coloring,
we present an implementation that is memory-less and also accesses the input description through local probes,
as elaborated in the following model:

\begin{definition}
\label{def:local_access_LCA}
Given a \textbf{random object family} $\{(\mathsf X^{\Pi}, \mathbb X^{\Pi})\}$ parameterized by input $\Pi$,
a \emph{local access implementation} of a family of query functions $\langle F_1, F_2,\cdots \rangle$,
provides an oracle $\mathcal M$ with the following properties.
$\mathcal M$ has query access to the input description $\Pi$, and a tape of public random bits $\vec R$.
Upon being queried with $\Pi$ and $F_i$, the oracle uses sub-linear resources to return the value $\mathcal M(\Pi,\vec R,F_i)$,
which must equal $F_i(X)$ for a specific $X\in\mathbb X^{\Pi}$, where the choice of $X$ depends only on $\vec R$,
and the distribution of $X$ (over $\vec R$) is $\frac1{n^c}$-close to the distribution over $\mathbb X^{\Pi}$, for any given constant $c$.
Thus, different instances of $\mathcal M$ with the same description $\Pi$ and the same random bits $\vec R$,
must agree on the choice of $X$ that is consistent with all answered queries regardless of the order and content of queries that were actually asked.
\end{definition}

We can contrast Definition~\ref{def:local_access_LCA} with the one for \emph{Local Computation Algorithms} \cite{LCA, LCA_space_efficient}
which also allow query access to \emph{some} valid solution by reading the input through local probes.
The additional challenge in our setting is that we also have to make sure that we return a uniformly random solution, rather than an arbitrary one.
%Similarly to LCAs, we can have multiple independent instances of our algorithm answering different queries, but remaining consistent with one another.
We also note that the memory-less property may be achieved for small description size \textbf{random object families}.
For instance, our implementation for the directed small world model admits such a memory-less implementation using public random bits.



\subsection{Undirected Graphs}
\label{sec:undirected_graphs}
We consider the generic class of \emph{undirected graphs} with {\em independent edge probabilities} $\left\{ p_{uv} \right\}_{u,v\in V}$,
(where $p_{uv}$ denotes the probability that there is an edge between $u$ and $v$),
from which the results can be applied to Erd\"os-R\'enyi random graphs and the Stochastic Block Model.
Throughout, we identify our vertices via their unique IDs from $1$ to $n$, namely $V = [n]$.
In this model, \func{Vertex-Pair} queries by themselves can be implemented trivially,
since the existence of any edge $(u,v)$ is an independent Bernoulli random variable,
but it becomes harder to maintain consistency when implementing them in conjunction with the other queries.
Inspired by \cite{reut}, we provide an implementation of $\func{next-neighbor}$ queries,
which return the neighbors of any given vertex one by one in lexicographic order.
Finally, we introduce a new query: \func{Random-Neighbor} that returns a uniformly random neighbor of any given vertex.
This would be useful for any algorithm that performs random walks.
$\func{Random-Neighbor}$ queries in non-sparse graphs present particularly interesting challenges that are outlined below.

\paragraph*{\func{Next-Neighbor} Queries}
\label{par:next_neighbor_queries}
In Erd\"os-R\'enyi graphs, the (lexicographically) next neighbor of a vertex can be recovered by generating consecutive entries
of the adjacency matrix until a neighbor is found, which takes roughly $\Omega(1/p_{uv})$ time.
For small edge probabilities $p_{uv} = o(1)$, this implementation is inefficient, and we show how to improve the runtime to $\mathcal O(\poly(\log n))$.
Our main technique is to sample the number of ``non-neighbors'' preceding the next neighbor.
To do this, we assume that we can estimate the ``skip'' probabilities $F(v,a,b)=\prod^{b}_{u=a} (1-p_{vu})$,
where $F(v,a,b)$ is the probability that $v$ has no neighbors in the range $[a,b]$.
We later show how to compute this quantity efficiently for $G(n,p)$ and the Stochastic Block Model.

This strategy of \emph{skip-sampling} is also used in \cite{reut}.
However, in our work, the main difficulty arises from the fact that our graph is undirected,
and thus we must ``inform'' all (potentially $\Theta(n)$) non-neighbors once we decide on the query vertex's next neighbor.
Concretely, if $u'$ is sampled as the next neighbor of $v$ after its previous neighbor $u$,
we must maintain consistency in subsequent steps by ensuring that none of the vertices in the range $(u,u')$ return $v$ as a neighbor.
This update will become even more complicated as we handle \func{Random-Neighbor} queries, where we may generate non-neighbors at random locations.

In Section~\ref{sec:ER-rand}, we present a randomized implementation (Algorithm~\ref{alg:oblivious-coin-toss})
that supports \func{Next-Neighbor} queries efficiently, but has a complicated performance analysis.
We remark that this approach may be extended to support \func{Vertex-Pair} queries (Section~\ref{sec:reroll-cont}) with superior performance
(if we do not support \func{Random-Neighbor} queries),
and also to provide deterministic resource usage guarantee (Section~\ref{sec:ER-det}).


\paragraph*{\func{Random-Neighbor} Queries}
\label{par:random_neighbor_queries}
We implement \func{Random-Neighbor} queries (Section~\ref{sec:blocks}) using $\poly(\log n)$ resources.
The ability to do so is surprising since: \textcolor{Maroon}{
(1) Sampling the degree of vertex, may not be viable for \emph{sub-linear} implementations, because this quantity alone
imposes dependence on the existence of \emph{all} of its potential incident edges and consequently on the rest of the graph (since it is undirected).}
Thus, our implementation needs to return a random neighbor, with probability reciprocal to the query vertex's degree,
without resorting to \emph{determining} its degree.
(2) Even without committing to the degrees, answers to \func{Random-Neighbor} queries
affect the conditional probabilities of the remaining adjacencies in a global and non-trivial manner.\footnote{
\label{conditional} Consider a $G(n,p)$ graph with small $p$, say $p = 1/\sqrt n$,
such that vertices will have $\widetilde{\mathcal{O}}(\sqrt n)$ neighbors with high probability.
After $\widetilde{\mathcal{O}}(\sqrt n)$ \func{Random-Neighbor} queries, we will have uncovered all the neighbors (w.h.p.),
so that the conditional probability of the remaining $\Theta(n)$ edges should now be close to zero.}

We formulate an approach which samples many consecutive edges simultaneously,
in such a way that the conditional probabilities of the unsampled edges remain independent and ``well-behaved'' during subsequent queries.
For each vertex $v$, we divide the potential neighbors of $v$ into consecutive ranges $\{B^{(i)}_v\}$ called blocks,
so that each $B^{(i)}_v$ contains $\Theta(1)$ neighbors in expectation (i.e. $\sum_{u\in B_i} p_{vu} = \Theta(1)$).
The subroutine of \func{Next-Neighbor} is applied to sample the neighbors within a block in expected $\mathcal{\widetilde O}(1)$ time.
We can now obtain a neighbor of $v$ by picking a random neighbor from a random block,
but this introduces a bias because all blocks may not have the same number of neighbors.
We remove this bias by rejecting samples from block $B^{(i)}_v$ with probability proportional to the number of neighbors in $B^{(i)}_v$.




\subsubsection{Applications to Random Graph Models}
\label{sec:applications_to_random_graph_models}
We now consider the application of our construction above to actual random graph models,
where we must realize the assumption that $\prod^{b}_{u=a} (1-p_{vu})$ and $\sum^{b}_{u=a} p_{vu}$ can be computed efficiently.
For the Erd\"{o}s-R\'{e}nyi $G(n,p)$ model, these quantities have simple closed-form expressions.
Thus, we obtain implementations of $\func{Vertex-Pair}$, $\func{Next-Neighbor}$, and $\func{Random-Neighbor}$ queries,
using polylogarithmic resources (time, space and random bits) per query, for \emph{arbitrary} values of $p$,
We remark that, while $\Omega(n+m) = \Omega(p n^2)$ time and space is clearly necessary to generate and represent a full random graph,
our implementation supports local-access via all three types of queries, and yet can generate a full graph in $\widetilde{O}(n+m)$ time and space
(Corollary~\ref{thm:er-optimal}), which is tight up to polylogarithmic factors.

We also generalize our construction to implement the Stochastic Block Model.
In this model, the vertex set is partitioned into $r$ \emph{communities} $\left\{ C_1, \ldots, C_r \right\}$.
The probability that an edge exists between $u\in C_i$ and $v \in C_j$ is $p_{ij}$.
As communities in the observed data are generally unknown a priori,
and significant research has been devoted to designing efficient algorithms for community detection and recovery, these studies generally consider
the \emph{random community assignment} condition for the purpose of designing and analyzing algorithms \cite{mossel2015reconstruction}.
Thus, we construct implementations where the community assignments are sampled from some given distribution,
or from a collection of specified community sizes.
%\footnote{
%Our algorithm also supports the alternative specification where the community sizes $\langle |C_1|, \ldots, |C_r|\rangle$
%are given instead, where the assignment of vertices $V$ into these communities is chosen uniformly at random.}.
The main difficulty here is to obtain a uniformly sampled assignment of vertices to communities on-the-fly.

Our approach is, as before, to sample for the next neighbor or a random neighbor directly,
although our result does not simply follow closed-form formulas,
as the probabilities for the potential edges now depend on the communities of endpoints.
To handle this issue, we observe that it is sufficient to efficiently count
the number of vertices of each community in any range of contiguous vertex indices.
We then design a data structure extending a construction of \cite{huge}, which maintains these counts for ranges of vertices,
and determines the partition of their counts only on an as-needed basis.
This extension results in an efficient technique to sample counts from the \emph{multivariate hypergeometric distribution}
(Section~\ref{sec:multivariate_hypergeometric_sampling}) which may be of independent interest.
For $r$ communities, this yields an implementation with $ \mathcal{O}(r\cdot \poly(\log n))$ overhead in required resources for each operation.




\subsection{Directed Graphs}
\label{sec:directed_graphs}
Lastly, we consider Kleinberg's Small World model (\cite{kleinberg, klein}) in Section~\ref{sec:small_world}.
While small world models are proposed to capture properties of observed data such as small shortest-path
distances and large clustering coefficients \cite{watts1998collective},
this important special case of Kleinberg's model, defined on two-dimensional grids, demonstrates the underlying geographical structures of networks.

In this model, each vertex is identified via its 2D coordinate $v = (v_x, v_y) \in [\sqrt{n}]^2$.
Defining the Manhattan distance as $\func{dist}(u,v)=|u_x-v_x|+|u_y-v_y|$,
the probability that each directed edge $(u,v)$ exists is $c/(\func{dist}(u,v))^{2}$.
A common choice for $c$ is given by normalizing the distribution such that the expected out-degree of each vertex is 1 ($c = \Theta(1/\log n)$).
We can also support a range of values of $c=\log^{\pm\Theta(1)}n$.
Since the degree of each vertex in this model is $\mathcal O(\log n)$ with high probability, we implement \func{All-Neighbor} queries,
which in turn can emulate \func{Vertex-Pair}, \func{Next-Neighbor} and \func{Random-Neighbor} queries.
In contrast to our previous cases, this model imposes an underlying two-dimensional structure of the vertex set,
which governs the distance function as well as complicates the individual edge probabilities.
%Providing local access for directed graphs is simpler because the out-neighbors of vertices may be chosen independently at each vertex.

We implement \func{All-Neighbors} queries in the small world model by listing all neighbors from closest to furthest away from the queried vertex,
using $\poly(\log n)$ resources per query.
The main challenge is to sample for the next closest neighbor, when the probabilities are a function of the Manhattan distance on the lattice.
Rather than sampling for a neighbor directly, we partition the nodes based on their distance from $v$ (there are $\Theta(d)$ vertices at distance $d$).
We first choose the next smallest distance partition with a neighbor using rejection sampling techniques
(Lemma~\ref{lem:rejection_sampling}) on the appropriate distribution.
In the second step, we generate all the neighbors within that partition using \emph{skip-sampling}.

\subsection{Random Catalan Objects}
\label{sec:overview_catalan_objects}
Consider a one dimensional random walk on the line with $n$ up and $n$ down steps, starting from the origin,
with a constraint that the height is always non-negative.
We implement two queries, $\func{Height}$ and \func{First-Return}, where $\func{Height}(t)$ which returns the position of the walk at time $t$,
and \func{First-Return}$(t)$ returns the first time when the random walk returns to the same position as it was at time $t$.

Over the course of the execution, our algorithm will sample the height of the walk at many different positions $\{ x_1, x_2,\cdots, x_m\}$
(with $x_i<x_{i+1}$), both directly as a result of user given $\func{Height}$ queries, and indirectly through recursive calls to $\func{Height}$.
These sampled positions divide the sequence into contiguous \emph{intervals} $[x_i,x_{i+1}]$,
where the height of the endpoints $y_i, y_{i+1}$ have been sampled, but none of the intermediate heights are known.
The important observation is that, since the beginning and ending heights are known,
the section of the path within an \emph{interval} is completely independent of all other \emph{intervals}.
So, each interval $[x_i,x_{i+1}]$ along with the corresponding heights $y_i,y_{i+1}$,
represents a generalized Dyck problem with $U$ up steps, $D$ down steps, and the constraint that the path never goes below $0$.

\paragraph*{$\func{Height}$ Queries}
\label{par:height_queries}
We start by implementing a subroutine that given an \emph{interval} $[x_i,x_{i+1}]$ of length of length $2B$ with $2U$ up and $2D$ down steps,
samples the number of up steps $U'=U+d$ to the first half of the \emph{interval}
(we parameterize $U'$ with $d$ in order to make the analysis cleaner).
Note that this is equivalent to answering the query $\func{Height}(x_i+B)$.
This is done by sampling the parameter $d$ from a distribution $\{ p_d\}$ with $p_d = S_{left}(d)\cdot S_{right}(d)/S_{total}$,
where $S_{left}(d)$ (respectively $S_{right}(d)$) is the number of possible paths in the left (resp. right) half of the \emph{interval} when
$U+d$ up steps and $D-d$ down steps are assigned to the first half, and $S_{total}$ is the number of possible paths in the original $2B$-interval.
General $\func{Height}(x)$ queries can then be answered by recursively halving the \emph{interval},
and sampling the height of the midpoint, until the height of $x$ is sampled.

The problem of sampling the number of up steps in the first half of the \emph{interval} was solved for the case where the sequence is fully random
in \cite{huge}.
Adding the non-negativity constraint introduces further difficulties as the distribution over $d$ has a CDF that is difficult to compute.
We construct a different distribution $\{q_d\}$ that approximates $\{p_d\}$ pointwise to a factor of $\log n$ and has an efficiently computable CDF.
This allows us to sample from $\{q_d\}$ and leverage the rejection sampling lemma (Lemma~\ref{lem:rejection_sampling}) to obtain samples from $\{p_d\}$.

\paragraph*{$\func{First-Return}$ Queries}
\label{par:_first-return_queries}
$\func{First-Return}$ queries present an additional challenge because we don't know which \emph{interval} contains the first return.
Since there could be up to $\Theta(n)$ intervals, is it inefficient to iterate through all of them.
To circumvent this problem, we allow each interval to sample its own boundary constraint $k>0$ instead of using the global non-negativity constraint.
A boundary constraint of $k$ implies that the path within the interval $[x_i,x_{i+1}]$ never reaches the height $y_i-k$ or lower.
Additionally, we maintain an invariant that states that this boundary $x_i-k$ coincides with $\min(x_i,x_{i+1})$.
If this constraint is satisfied, we can find the interval containing $\func{First-Return}(x)$ by finding the smallest sampled position $x_i>x$
whose sampled height $y_i \le \func{Height}(x)$, and considering the interval $[x_{i-1},x_i]$ preceding $x_i$.

Every time the $\func{Height}$ algorithm creates new intervals by sub-dividing an existing one, this invariant is potentially broken.
We re-establish it by sampling a ``mandatory boundary''
(a boundary constraint with the additional restriction that some position within the interval $[x_i,x_{i+1}]$ \emph{must} touch the boundary),
%(a $y$-coordinate that must be achieved within the interval $[x_i,x_{i+1}]$ but not exceeded),
and then sampling a position $x$ such that $x_i < x < x_{i+1}$ and $\func{Height}(x) = y$.
The first step of sampling the mandatory boundary is performed by binary searching on the possible boundary locations.
To find a position that touches this boundary, we parameterize the position with $d$ and find the distribution $\{p_d\}$ associated with these events.
We then define a piecewise continuous PDF $\hat q(\delta)$ such that $\hat q(\delta)$ approximates $p_{\floor\delta}$.
We then use this to construct $q_d = \int_d^{d+1}\hat q(\delta)$, where the CDF of $q_d$ is efficently computable using integration,
and use rejection sampling (Lemma~\ref{lem:rejection_sampling}) again to sample indirectly from $\{p_d\}$.

\subsection{Random Coloring of a Graph}
\label{sec:overview_random_coloring_of_a_graph}
Finally, we introduce a new model (Definition~\ref{def:local_access_LCA}) for implementing huge random objects,
where the distribution is specified as a uniformly random solution to a huge combinatorial problem.
In this new setting, we will implement local query access to random $q$-colorings of a given huge graph $G$ of size $n$ with maximum degree $\Delta$.
%by implementing a query $\func{Color}(v)$, which returns the color of a \emph{single vertex} $v$.
Since the implementation has to run in sub-linear time, it is not possible to read the entire input $G$ during a single query execution.


\paragraph*{$\func{Color}$ Queries}
\label{par:color_queries}
Given a graph $G$ with maximum degree $\Delta$, and the number of colors $q\ge 9\Delta$,
we are able to construct an efficient implementation for $\func{Color}(v)$ that returns the final color of $v$
in a uniformly random $q$-coloring of $G$ using only a sub-linear number of probes.
Random colorings of a graph are sampled using $\mathcal O(n\log n)$ iterations of a Markov chain \cite{glauber_survey}.
Each step of the chain proposes a random color update for a random vertex, and accepts the update if it does not create a conflict.
This is an inherently sequential process, with the acceptance of a particular proposal depending on all preceding neighboring proposals.

To make the runtime analysis simpler, we define a modified version of Glauber Dynamics that proceeds in $\mathcal O(\log n)$ epochs.
In each epoch, all of the $n$ vertices propose a random color and update themselves if their proposals do not conflict with any of their neighbors.
This Markov chain is a special case of the one presented in \cite{ghaffari_fischer} for distributed graph coloring,
and mixes in $\mathcal O(\log n)$ epochs when $q\ge 9\Delta$.
%While we do not have the same restrictions as the distributed computation setting,
%we choose to use this chain so as to avoid long analysis of mixing times.
In order to implement the query $\func{Color}(v)$, it suffices to implement $\func{Accepted}(v,t)$
that indicates whether the proposal for $v$ was accepted in the $t^{th}$ epoch.
This depends on the prior colors of the potentially $\Delta$ neighbors of $v$.
Determining the prior colors of all the neighbors $w$ using recursive calls would result in
$\Delta$ invocations of $\func{Accepted}(w, t-1)$ (at the preceding epoch $t-1$).
Naively, this gives a bound of $\Delta^t$ on the number of invocations.
We can prune the recursions by only considering neighbors who proposed the color $c$ during \emph{some} past epoch.
This reduces the expected number of recursive calls to $t\Delta/q$,
since there are $t\Delta$ potential proposals and each one is $c$ with probability $1/q$.
If $q$ is larger than $t\Delta$, the number of recursive calls is less than $1$,
which gives a sub-linear bound on the total number of resulting invocations.
Since the number of epochs $t$ can be as large as $\Theta(\log n)$, this strategy will only work when $q = \Omega(\Delta\log n)$.

The improvement to $q = \Omega(\Delta)$ follows from the observation that for a neighbor $w$ that proposed color $c$ at epoch $t'$,
the recursive call corresponding to $w$ can directly jump to epoch $t'$.
If the conflicting color $c$ was indeed accepted at that epoch,
we then step \emph{forwards} through epochs $t'+1, t'+2,\cdots$, to check whether $c$ was overwritten by some future accepted proposal.
This strategy dramatically reduces the recursive sub-problem size (given by the epoch number $t'$),
and furthermore we show that we do not have to step through too many future epochs in order to check whether $c$ was overwritten.
This allows us to bound the runtime by $\widetilde{\mathcal O}\left(t\Delta (n/\epsilon)^{6.12\Delta/q}\right)$,
where the overall coloring is sampled from a distribution that is $\epsilon$-close to uniform (see Definition~\ref{def:local_access_LCA}).

One requirement for our strategy is the ability to access a \emph{valid initial coloring} (the initial state of the Markov Chain)
through local probes, in addition to local probes to the underlying graph structure.
This assumption can be removed by using a result of \cite{coloring_initialize},
that presents an LCA for $\Delta+1$ graph coloring using $\Delta^{\mathcal O(1)}\log n$ graph probes.
Alternately, we can assign random initial colors to the vertices, which may result in an \emph{invalid} final coloring.
However, the Markov Chain will transform the initial invalid coloring into a valid one with high probability.





\subsection{Basic Tools for Efficient Sampling}
\label{sec:basic_tools_for_efficient_sampling}
In this section, we describe the main techniques used to sample from a distribution $\{ p_i\}_{i\in [n]}$,
which differs based on the type of access to $\{p_i\}$ provided to the algorithm.
If the algorithm is given cumulative distribution function (CDF) access to $\{p_i\}$,
then it is well known that via $\mathcal O(\log n)$ CDF evaluations, one can sample according
to a distribution that is at most $n^{-c}$ far from $\{p_i\}$ in $L_1$ distance.

Sampling can be more challenging when when we can only access the probability distribution function (PDF) of $\{p_i\}$,
The approach that we use in this work is to construct an auxiliary distribution $\{q_i\}$ such that:
(1) $\{ q_i\}$ has an efficiently computable CDF, and
(2) $q_i$ approximates $p_i$ pointwise to within a polylogarithmic multiplicative factor for ``most'' of the support of $\{ p_i\}$.
The following Lemma inspired by \cite{huge} formalizes this concept.
\begin{lemma}
\label{lem:rejection_sampling}
Let $\{p_i\}_{i\in [n]}$ and $\{q_i\}_{i\in [n]}$ be distributions on $[n]$ satisfying the following conditions:
\begin{enumerate}
    \item There is a $\log^{\mathcal O(1)}n$ time algorithm to approximate $p_i$ and $q_i$
    up to a multiplicative $\left(1\pm \frac{1}{n^c}\right)$ factor.
    \item We can generate an index $i$ according to a distribution $\{\hat q_i\}$,
    where $\hat q_i$ is a $\left(1\pm \frac{1}{n^c}\right)$ multiplicative approximation to $q_i$.
    \item There exists a $poly(\log n)$-time recognizable set $S$ such that
    \begin{itemize}
        \item $1-\sum\limits_{i\in S} p_i < \frac 1{n^c}$
        \item For every $i\in S$, it holds that $p_i\le \log^{\mathcal{O}(1)} n\cdot q_i$
    \end{itemize}
\end{enumerate}
Then, with high probability we can use only $\log^{\mathcal O(1)}n$ samples from $\{\hat q_i\}$ to generate an index $i$
according to a distribution that is $ \mathcal O\left(\frac{1}{n^c}\right)$-close to $\{p_i\}$ in $L_1$ distance.
\end{lemma}
\begin{proof}
We begin by setting an upper bound $U = \log^{\mathcal O(1)}n$ on $p_i/q_i$ for all $i\in S$.
The sampling proceeds in iterations, such that in each iteration we obtain an index $i$ according to the distribution $\{ \hat q_i\}$.
If $i\not\in S$, this index is returned with probability $\tilde p_i/U\tilde q_i$,
where $\tilde p_i$ and $\tilde q_i$ are the $\left( 1\pm \frac{1}{n^c} \right)$ multiplicative approximations to $p_i$ and $q_i$,
Otherwise, we repeat this process until some output is returned.

The probability of returning index $i\in S$ in a particular iteration is $\hat q_i\cdot (\tilde p_i/U \tilde q_i)$,
which is in turn a $\left( 1\pm \frac{1}{n^c} \right)$ multiplicative approximation to $p_i/U$.
Hence, the probability of success in a single iteration is roughly $1/U$,
and therefore we only need $\mathcal O(U\log n) = \log^{\mathcal O(1)}n$ iterations (and the same number of samples from $\{ \hat q_i\}$)
in order to succeed with high probability.
The resulting distribution of indices approximates $\{ p_i\}$ pointwise on the domain $S$, up to a factor of $\left( 1\pm \frac{1}{n^c} \right)$.
Since the remainder of the domain contains at most $ \frac{1}{n^c}$ probability mass,
the output distribution is $\mathcal O\left(\frac{1}{n^c}\right)$ close to $\{ p_i\}$ in $L_1$ distance.
\end{proof}
