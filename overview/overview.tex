\section{Model and Overview of our Techniques}
\label{sec:overview_of_our_techniques}
We begin by formalizing our model of \emph{local-access implementations}, inspired by \cite{reut},
but we add some aspects that were not addressed in earlier models.
First, we define families of random objects.

\begin{definition}
\label{def:parametrized_random_object}
A \textbf{random object family} maps a description $\Pi$ to a distribution $\mathsf X^{\Pi}$ over the set $\mathbb X^{\Pi}$.
\end{definition}

For example, the \emph{family} of Erd\"{o}s-R\'{e}nyi graphs maps $\Pi = (n, p)$ to a distribution over $\mathbb X^{\Pi}$,
which is the set of all possible $n$-vertex graphs,
where the probability assigned to any graph containing $m$ edges in the distribution $\mathsf X^{\Pi}$ is exactly $p^m\cdot (1-p)^{\binom{n}{2}-m}$.

\begin{definition}
\label{def:local_access}
Given a \textbf{random object family} $\{(\mathsf X^{\Pi}, \mathbb X^{\Pi})\}$ parameterized by $\Pi$, a \emph{local access implementation}
%Given a distribution $\mathsf X^{\Pi}$ over a set of huge random objects $\mathbb X$, a \emph{local access implementation}
of a family of query functions $\langle F_1, F_2,\cdots \rangle$ where $F_i: \mathbb X^{\Pi}\rightarrow \{0,1\}$,
provides an oracle $\mathcal M$ with an internal state for storing partially generated random object.
Given a description $\Pi$ and a query $F_i$, the oracle returns the value $\mathcal M(\Pi, F_i)$,
and updates it's internal state, while satisfying the following:
\begin{itemize}
    \item \textbf{Consistency:}
    There must be a single $X\in \mathbb X^{\Pi}$, such that for all queries $F_i$ presented to the oracle,
    the returned value $\mathcal M(\Pi,F_i)$ equals the true value $F_i(X)$.
    %must be consistent with a single $X\in \mathbb X^{\Pi}$.
    \item \textbf{Distribution equivalence:}
    The random object $X\in \mathbb X^{\Pi}$ consistent with the responses $\{ F_i(X)\}$ must be sampled from some distribution $\hat{\mathsf{X}}^{\Pi}$
    that is $\epsilon$-close to the desired distribution $\mathsf{X}^{\Pi}$ in $L_1$-distance.
    In this work, we focus on supporting $\epsilon = n^{-c}$ for any desired constant $c>0$.
    \item \textbf{Performance:}
    The computation time, and random bits required to answer a single query must be sub-linear in $|X|$ with high probability,
    without any initialization overhead.
\end{itemize}
In particular, we allow queries to be made adversarially and non-deterministically.
The adversary has full knowledge of the algorithm's behavior and its past random bits.
\end{definition}

For example, in the $G(n,p)$ family with description $\Pi = (n, p)$,
we can define \func{Vertex-Pair} query functions $\{ F_{(u,v)}\}_{u,v\in [n]}$.
So, given a graph $G\in \mathbb X^{\Pi}$, the query $F_{(u,v)}(G) = 1$ if and only if $(u,v)\in G$.

In prior work \cite{reut, huge, sparse} as well as some of our results, the input description $\Pi$ is of small size (typically $\mathcal O(\log n)$),
and the oracle $\mathcal M$ can read all of $\Pi$ (for example, $\Pi = (n, p)$ in $G(n,p)$).

\subparagraph*{{\Large Distributions with Huge Description Size:}}
\label{par:distributions_with_huge_description_size}
We initiate the study of \textbf{random object families} where the description $\Pi$ is too large to be read by a sub-linear time algorithm.
%In all the problems considered in prior work \cite{huge,sparse,reut}, the description size
%of the random object is small, typically $\mathcal O(\log n)$ to represent the size of the instance and a constant number of parameters.
%For instance, the $G(n, p)$ model is described using two parameters $n$ and $p$.
%However, if one wishes to implement local access to the uniform distribution over all valid colorings of a given input graph $G$,
In this setting, the oracle from Definition~\ref{def:local_access} cannot read the entire input $\Pi$, and instead accesses it through local probes.
For instance, consider the \textbf{random object family} that maps a graph $G$ to the uniform distribution over valid colorings of $G$.
Here, the description $\Pi$ includes the entire graph $G$, which is too large to be read by a sublinear time algorithm.
In this case, the oracle can query the underlying graph using neighborhood probes.
The number of such probes used to answer a single query must be sub-linear in the input size.


\paragraph*{Supporting Independent Query Oracles: Memory-less Implementations}
\label{par:supporting_independent_query_oracles_memory_less_implementations}
The model in Definition~\ref{def:local_access} only supports sequential queries,
since the response to a future query may depend on the changes in internal state caused by past queries.
In some applications, we may want to have multiple independent query oracles whose responses are all consistent with each other.
One way to achieve this is to restrict our attention to \emph{memory-less} implementations; ones without any internal state.
An important implication of being memory-less is that the responses to each query is oblivious to the order of queries being asked.
In fact, the lack of internal state implies that independent implementations that use the same random bits and the same input description
must respond to queries in the same way.
Thus, instead of using the internal state to maintain consistency, memory-less implementations are given access to a public random oracle.

%In addition to the above generalization of Definition~\ref{def:local_access}, we also consider ``memory-less'' oracle implementations,
%where the oracle does not have any internal state, and instead has access to a source of common randomness.
%An important implication of being memory-less is that the responses to each query is oblivious of the order of queries being asked.
%In fact, the lack of internal state implies that independent implementations that use the same common randomness and the same input description
%must respond to queries in the same way.
%So, we can have multiple (and even simultaneous) copies of the implementation,
%that are all consistent with a single random object drawn from the distribution,
%without having to write to shared memory or communicate in any other way.

For the problem of sampling a random graph coloring,
we present an implementation that is memory-less and also accesses the input description through local probes,
as elaborated in the following model:

\begin{definition}
\label{def:local_access_LCA}
Given a \textbf{random object family} $\{(\mathsf X^{\Pi}, \mathbb X^{\Pi})\}$ parameterized by input $\Pi$,
a \emph{local access implementation} of a family of query functions $\langle F_1, F_2,\cdots \rangle$,
provides an oracle $\mathcal M$ with the following properties.
$\mathcal M$ has query access to the input description $\Pi$, and a tape of public random bits $\vec R$.
Upon being queried with $\Pi$ and $F_i$, the oracle returns the value $\mathcal M(\Pi,\vec R,F_i)$,
which must equal $F_i(X)$ for a specific $X\in\mathbb X^{\Pi}$, where the choice of $X$ depends only on $\vec R$,
and the distribution of $X$ (over $\vec R$) is $\frac1{n^c}$-close to the distribution over $\mathbb X^{\Pi}$, for any given constant $c$.
Thus, different instances of $\mathcal M$ with the same description $\Pi$ and the same random bits $\vec R$,
must agree on the choice of $X$ that is consistent with all answered queries regardless of the order and content of queries that were actually asked.
\end{definition}

We can contrast Definition~\ref{def:local_access_LCA} with the one for \emph{Local Computation Algorithms} \cite{LCA, LCA_space_efficient}
which also allow query access to \emph{some} valid solution by reading the input through local probes.
The additional challenge in our setting is that we also have to make sure that we return a uniformly random solution, rather than an arbitrary one.
%Similarly to LCAs, we can have multiple independent instances of our algorithm answering different queries, but remaining consistent with one another.
We also note that the memory-less property may be achieved for small description size \textbf{random object families}.
For instance, our implementation for the directed small world model admits such a memory-less implementation using public random bits.



\subsection{Undirected Graphs}
\label{sec:undirected_graphs}
In Section~\ref{sec:undirected}, we implement queries to both the adjacency matrix and adjacency list representation
for the generic class of \emph{undirected graphs} with {\em independent edge probabilities} $\left\{ p_{uv} \right\}_{u,v\in V}$,
where $p_{uv}$ denotes the probability that there is an edge between $u$ and $v$.
Throughout, we identify our vertices via their unique IDs from $1$ to $n$, namely $V = [n]$.
In this model, \func{Vertex-Pair} queries by themselves can be implemented trivially,
since the existence of any edge $(u,v)$ is an independent Bernoulli random variable,
but it becomes harder to maintain consistency when implementing them in conjunction with the other queries.
Inspired by \cite{reut}, we provide an implementation of $\func{next-neighbor}$ queries,
which return the neighbors of any given vertex one by one in lexicographic order.
Finally, we introduce a new query: \func{Random-Neighbor} that returns a uniformly random neighbor of any given vertex.
This would be useful for any algorithm that performs random walks.
$\func{Random-Neighbor}$ queries present particularly interesting challenges that are outlined below.

\paragraph*{\func{Next-Neighbor} Queries}
\label{par:next_neighbor_queries}
The next neighbor of a vertex can be found trivially by generating consecutive entries of the adjacency matrix,
but for small edge probabilities $p_{uv} = o(1)$ this implementation is inefficient, and uses $\Omega (1/p)$ time.
As in \cite{reut}, we speed this up by sampling the number of ``non-neighbors'' preceding the next neighbor.
To do this, we assume that we can estimate the ``skip'' probabilities $F(v,a,b)=\prod^{b}_{u=a} (1-p_{vu})$,
where $F(v,a,b)$ is the probability that $v$ has no neighbors in the range $[a,b]$.
We later show that it is possible to compute this quantity efficiently for the $G(n,p)$ and Stochastic block models.

A main difficulty as compared to \cite{reut}, arises from the fact that our graph is undirected,
and thus we must ``inform'' all (potentially $\Theta(n)$) non-neighbors once we decide on the query vertex's next neighbor.
More concretely, if $u'$ is sampled as the next neighbor of $v$ after its previous neighbor $u$,
we must maintain consistency in subsequent steps by ensuring that none of the vertices in the range $(u,u')$ return $v$ as a neighbor.
This update will become even more complicated as we handle \func{Random-Neighbor} queries, where we may generate non-neighbors at random locations.

In Section~\ref{sec:ER-rand}, we present a very simple randomized implementation (Algorithm~\ref{alg:oblivious-coin-toss})
that supports \func{Next-Neighbor} queries efficiently, albeit the analysis of its performance is rather complicated.
We remark that this approach may be extended to support \func{Vertex-Pair} queries with superior performance
(given that we do not to support \func{Random-Neighbor} queries) and to provide deterministic resource usage guarantee
-- the full analysis can be found in Section~\ref{sec:reroll-cont} and \ref{sec:ER-det}, respectively.


\paragraph*{\func{Random-Neighbor} Queries}
\label{par:random_neighbor_queries}

We provide \func{Random-Neighbor} queries (Section~\ref{sec:blocks}) using $\poly(\log n)$ resources.
The ability to do so is surprising since:
(1) Sampling the degree of the query vertex, we suspect, is not viable for \emph{sub-linear} implementations, because this quantity alone
imposes dependence on the existence of \emph{all} of its potential incident edges and on the rest of the graph.
\todo[inline]{Why?}
Therefore, our implementation needs to return a random neighbor, with probability reciprocal to the query vertex's degree,
without resorting to ``knowing'' its degree.
(2) Even without committing to the degrees, answers to \func{Random-Neighbor} queries
affect the conditional probabilities of the remaining adjacencies in a global and non-trivial manner.
\footnote{\label{conditional}Consider a $G(n,p)$ graph with small $p$, say $p = 1/\sqrt n$,
such that vertices will have $\widetilde{\mathcal{O}}(\sqrt n)$ neighbors with high probability.
After $\widetilde{\mathcal{O}}(\sqrt n)$ \func{Random-Neighbor} queries, we will have uncovered all the neighbors (w.h.p.),
so that the conditional probability of the remaining $\Theta(n)$ edges should now be close to zero.}

\todo[inline,color=red!80!green!25]{Blocking??}
We formulate a \emph{blocking approach} which samples multiple consecutive edges at once,
in such a way that the conditional probabilities of the unsampled edges remain independent and ``well-behaved'' during subsequent queries.
For each vertex $v$, we divide the potential neighbors of $v$ into consecutive ranges $\{B^{(i)}_v\}$ (blocks),
so that each $B^{(i)}_v$ contains, in expectation, $\Theta(1)$ neighbors (i.e. $\sum_{u\in B_i} p_{vu} = \Theta(1)$).
The subroutine of \func{Next-Neighbor} is applied to sample the neighbors within a block in expected $\mathcal{\widetilde O}(1)$ time.
We can now obtain a neighbor of $v$ by picking a random neighbor from a random block,
but this introduces a bias because all blocks may not have the same number of neighbors.
We remove this bias by rejecting samples from block $B^{(i)}_v$ with probability proportional to the number of neighbors in $B^{(i)}_v$.
\func{Vertex-Pair} queries are implemented by sampling the relevant block.




\subsubsection{Applications to Random Graph Models}
\label{sec:applications_to_random_graph_models}
We now consider the application of our construction above to actual random graph models,
where we must realize the assumption that $\prod^{b}_{u=a} (1-p_{vu})$ and $\sum^{b}_{u=a} p_{vu}$ can be computed efficiently.
For the Erd\"{o}s-R\'{e}nyi $G(n,p)$ model, these quantities have simple closed-form expressions.
Thus, we obtain implementations of $\func{Vertex-Pair}$, $\func{Next-Neighbor}$, and $\func{Random-Neighbor}$ queries,
using polylogarithmic resources (time, space and random bits) per query, for \emph{arbitrary} values of $p$,
We remark that, while $\Omega(n+m) = \Omega(p n^2)$ time and space is clearly necessary to generate and represent a full random graph,
our implementation supports local-access via all three types of queries, and yet can generate a full graph in $\widetilde{O}(n+m)$ time and space
(Corollary~\ref{thm:er-optimal}), which is tight up to polylogarithmic factors.

We also generalize our construction to implement the Stochastic Block Model.
In this model, the vertex set is partitioned into $r$ \emph{communities} $\left\{ C_1, \ldots, C_r \right\}$.
The probability that an edge exists between $u\in C_i$ and $v \in C_j$ is $p_{ij}$.
As communities in the observed data are generally unknown a priori,
and significant research has been devoted to designing efficient algorithms for community detection and recovery, these studies generally consider
the \emph{random community assignment} condition for the purpose of designing and analyzing algorithms (see e.g., \cite{mossel2015reconstruction}).
Thus, we aim to construct implementations where the community assignment of vertices are independently sampled from some given distribution $\mathsf{R}$
\footnote{Our algorithm also supports the alternative specification where the community sizes $\langle |C_1|, \ldots, |C_r|\rangle$
are given instead, where the assignment of vertices $V$ into these communities is chosen uniformly at random.}.
The difficulty here is to obtain a uniformly sampled assignment of vertices to communities on-the-fly.

Our approach is, as before, to sample for the next neighbor or a random neighbor directly,
although our result does not simply follow closed-form formulas,
as the probabilities for the potential edges now depend on the communities of endpoints.
To handle this issue, we observe that it is sufficient to efficiently count
the number of vertices of each community in any range of contiguous vertex indices.
We then design a data structure extending a construction of \cite{huge}, which maintain these counts for ranges of vertices,
and ``sample'' the partition of their counts only on an as-needed basis.
This extension results in an efficient technique to sample counts from the \emph{multivariate hypergeometric distribution}
(Section~\ref{sec:multivariate_hypergeometric_sampling}) which may be of independent interest.
For $r$ communities, this yields an implementation with $ \mathcal{O}(r\cdot \poly(\log n))$ overhead in required resources for each operation.




\subsection{Directed Graphs}
\label{sec:directed_graphs}
Lastly, we consider Kleinberg's Small World model (\cite{kleinberg, klein}) in Section~\ref{sec:small_world}.
While Small-World models are proposed to capture properties of observed data such as small shortest-path 
distances and large clustering coefficients \cite{watts1998collective}, 
this important special case of Kleinberg's model, defined on two-dimensional grids, demonstrates underlying geographical structures of networks.

In this model, each vertex is identified via its 2D coordinate $v = (v_x, v_y) \in [\sqrt{n}]^2$.
Define the Manhattan distance as $\func{dist}(u,v)=|u_x-v_x|+|u_y-v_y|$,
and the probability that each directed edge $(u,v)$ exists is $c/(\func{dist}(u,v))^{2}$.
A common choice for $c$ is given by normalizing the distribution such that the expected out-degree of each vertex is 1 ($c = \Theta(1/\log n)$).
We can also support a range of values of $c=\log^{\pm\Theta(1)}n$.
Since the degree of each vertex in this model is $\Bo(\log n)$ with high probability, we design implementations supporting \func{All-Neighbor} queries.
In contrast to our previous cases, this model imposes an underlying two-dimensional structure of the vertex set,
which governs the distance function as well as complicates the individual edge probabilities.

We design generators for the aforementioned case of the Small-World model, supporting each \func{all-neighbors} query,
listing all neighbors from closest to furthest away from the queried vertex, using $\poly(\log n)$ resources per query.
Providing local access for directed graphs is simpler because the out-neighbors of vertices may be chosen independently at each vertex.
So, the main challenge is to sample for the next (closest) neighbor, when the probabilities are a function of the Manhattan distance on the lattice.
Rather than sampling for a neighbor directly, we sample the next smallest distance with a neighbor,
employing the rejection sampling technique that allows efficient sampling through an approximate distribution that have closed-form description,
then as a second step, sample for all neighbors for each chosen distance.

\subsection{Random Catalan Objects}
\label{sec:overview_catalan_objects}
Many important combinatorial objects can be interpreted as Catalan objects.
One such interpretation is a Dyck path; a one dimensional random walk on the line with $n$ up and $n$ down steps, starting from the origin,
with the constraint that the height is always non-negative.
We implement $\func{Height}(t)$, which returns the position of the walk at time $t$,
and \func{First-Return}$(t)$, which returns the first time when the random walk returns to the same position as it was at time $t$.
These queries are natural for several types of Catalan objects.
As noted previously, we can use standard bijections to translate the Dyck path query implementations into
natural queries for \emph{bracketed expressions} and \emph{ordered rooted trees}.
Specifically, \func{Height} values in Dyck paths are equivalent to \emph{depth} in bracket expressions and trees.
The \func{First-Return} queries are more involved, and are equivalent to finding the \emph{matching bracket} in bracket expressions,
and alternately to finding the \emph{next child} of a node in an ordered rooted tree (see Section~\ref{sec:bijections_to_other_catalan_objects}).

Over the course of the execution, our algorithm will determine the height of a random Dyck path at many different positions $\{ x_1, x_2,\cdots, x_m\}$
(with $x_i<x_{i+1}$), both directly as a result of user given $\func{Height}$ queries, and indirectly through recursive calls to $\func{Height}$.
These positions divide the sequence into contiguous \emph{intervals} $[x_i,x_{i+1}]$,
where the height of the endpoints $y_i, y_{i+1}$ have been determined, but none of the intermediate heights are known.
The important observation is that the unknown section of the path within an \emph{interval}
is entirely determined by the positions and heights of the endpoints, and in particular is completely independent of all other \emph{intervals}.
So, each interval $[x_i,x_{i+1}]$ along with corresponding heights $\{y_i,y_{i+1}\}$,
represents a generalized Dyck problem with $U$ up steps, $D$ down steps,
and a constraint that the path never dips more than $y_i$ units below the starting height.

\paragraph*{$\func{Height}$ Queries}
\label{par:height_queries}
General $\func{Height}(x)$ queries can then be answered by recursively halving the \emph{interval} containing $x$,
by repeatedly sampling the height at the midpoint, until the height of position $x$ is sampled.
We start by implementing a subroutine that given an \emph{interval} $[x_i,x_{i+1}]$ of length $2B$, containing $2U$ up and $2D$ down steps,
determines the number of up steps $U'=U+d$ assigned to the first half of the \emph{interval}
(we parameterize $U'$ with $d$ in order to make the analysis cleaner).
Note that this is equivalent to answering the query $\func{Height}(x_i+B)$.
This is done by sampling the parameter $d$ from a distribution $\{ p_d\}$ where $p_d \equiv S_{left}(d)\cdot S_{right}(d)/S_{total}$.
Here, $S_{left}(d)$ (respectively $S_{right}(d)$) is the number of possible paths in the left (resp. right) half of the \emph{interval} when
$U+d$ up steps and $D-d$ down steps are assigned to the first half, and $S_{total}$ is the number of possible paths in the original $2B$-interval.

The problem of determining the number of up steps in the first half of the \emph{interval} was solved for the unconstrained case
(where the sequence is just a random permutation of up and down steps) in \cite{huge}.
Adding the non-negativity constraint introduces further difficulties as the distribution over $d$ has a CDF that is difficult to compute.
We construct a different distribution $\{q_d\}$ that approximates $\{p_d\}$ pointwise to a factor of $\log n$ and has an efficiently computable CDF.
This allows us to sample from $\{q_d\}$ and leverage rejection sampling techniques
(see Lemma~\ref{lem:rejection_sampling} in Section~\ref{sec:basic_tools_for_efficient_sampling}) to obtain samples from $\{p_d\}$.

\paragraph*{$\func{First-Return}$ Queries}
\label{par:_first-return_queries}
Note that \func{First-Return}$(x)$ is only of interest when the first step after position $x$ is upwards (i.e. $\func{Height}(x+1)>\func{Height}(x)$),
since this situation is important for complex queries to random bracketed expressions and random rooted trees.
Section~\ref{sec:bijections_to_other_catalan_objects} details the rationale for this definition based on bijections between these objects.
\func{First-Return} queries are challenging because we need to find the \emph{interval} containing the first return to $\func{Height}(x)$.
Since there could be up to $\Theta(n)$ intervals, it is inefficient to iterate through all of them.
To circumvent this problem, we allow each interval $[x_i,x_{i+1}]$ to sample and maintain its own boundary constraint $k_i$
instead of using the global non-negativity constraint.
This implies that the path within the interval $[x_i,x_{i+1}]$ never reaches the height $y_i-k_i$ or lower.
Additionally, we maintain a crucial invariant that states that this boundary is achieved by the endpoint of lower height i.e. $\min(y_i,y_{i+1})$.
If the invariant holds, we can find the interval containing $\func{First-Return}(x)$ by finding the smallest detrmined position $x_j>x$
whose sampled height $y_j \le \func{Height}(x)$, and considering the interval $[x_{j-1},x_j]$ preceding $x_j$.
We use an interval tree to update and query for the known heights.

Unfortunately, every $\func{Height}$ query creates new intervals by sub-dividing existing ones, potentially breaking the invariant.
We re-establish the invariant for $[x_i,x_{i+1}]$ by generating a ``mandatory boundary'' $h$
(a boundary constraint with the added restriction that some position within the interval \emph{must} touch the boundary),
%(a $y$-coordinate that must be achieved within the interval $[x_i,x_{i+1}]$ but not exceeded),
and then sampling a position $x_{mid}\in [x_i,x_{i+1}]$ such that $\func{Height}(x_{mid}) = h$ (Figure~\ref{fig:dyck_invariant_preserve}).
This creates new intervals $[x_i,x_{mid}]$ and $[x_{mid},x_{i+1}]$, both of which have a boundary constraint at $h$.

The first step of sampling the \emph{mandatory boundary} is performed
by binary searching on the possible boundary locations using an appropriate CDF (Section~\ref{sec:sampling_the_lowest_achievable_height}).
To find an intermediate position touching this boundary, we parameterize the position with $d$,
and find the distribution $\{p_d\}$ associated with the various possibilities.
Since we cannot directly sample from this complicated distribution, we define a \emph{piecewise continuous} probability distribution
$\hat q(\delta)$ such that $\hat q(\delta)$ approximates $p_{\floor\delta}$ (Section~\ref{sec:sampling_first_position_touching_mandatory_boundary}).
We then use this to define a discrete distribution $\{q_d\}$ where $q_d = \int_d^{d+1}\hat q(\delta)$,
where we can efficiently compute the CDF of $\{q_d\}$ by integrating the piecewise continuous $\hat q(\delta)$.
The challenge here is to construct an appropriate $\hat q(\delta)$ that only has $\mathcal O(\log^{\mathcal O(1)} n)$ continuous pieces.
This allows us to again use the rejection sampling technique (Lemma~\ref{lem:rejection_sampling}) to indirectly obtain a sample from $\{p_d\}$.

\subsection{Random Coloring of a Graph}
\label{sec:overview_random_coloring_of_a_graph}
Finally, we introduce a new model for implementating huge random objects
where the distribution is specified as a uniformly random solution to a huge combinatorial problem.
In all the problems we have considered so far as well as the ones studied in prior work \cite{huge,sparse,reut}, the description size
of the random object is small (typically $\mathcal O(\log n)$ to represent the size of the instance and a constant number of parameters).
In this new setting, we will implement local query access to random $q$-colorings of a given huge graph $G$ of size $n$ with maximum degree $\Delta$.
The distribution in this case is defined by the graph structure which has size $\mathcal O(n\Delta)$.
We present the following definition for local access implementations in this setting.

\begin{definition}
\label{def:local_access_LCA}
Given a combinatorial problem on graphs,
a \emph{local access implementation} of a family of query functions $\langle F_1, F_2,\cdots \rangle$,
provides an oracle $\mathcal A$ with the following properties.
$\mathcal A$ has query access to a graph $G$, and a tape of random bits $\vec R$.
Assuming that the solution set of the combinatorial problem on $G$ is $\mathbb X$,
$\mathcal A$ upon being queried with $F_i$, returns the value $F_i(X)$ for a specific solution $X\in\mathbb X$ where the choice of $X$
depends only on $\vec R$ and the distribution of $X$ (over $\vec R$) is $\epsilon$-close to the uniform distribution over $\mathbb X$.
Two different instances of $\mathcal A$ with the same graph oracle and the same random bits,
must agree on the choice of $X$ that is consistent with all answered queries regardless of what queries were actually asked.
\end{definition}

We can contrast this definition with the one for \emph{Local Computation Algorithms} \cite{LCA, LCA_space_efficient}
which also allow query access to \emph{some} valid solution and can read the input through local probes.
An additional difficulty in our setting is that we also have to make sure that we return a solution from the correct distribution.
Similarly to LCAs, we can have multiple independent instances of our algorithm answering different queries, but remaining consistent with one another.


\paragraph*{$\func{Color}$ queries}
\label{par:color_queries}
Given a graph $G$ with maximum degree $\Delta$, and the number of colors $q\ge 9\Delta$,
we are able to construct an efficient implementation for $\func{Color}(v)$ that returns the final color of $v$
in a uniformly random $q$-coloring of $G$ using only a sub-linear number of probes.
Random colorings of a graph are sampled using $\mathcal O(n\log n)$ iterations of a Markov chain \cite{glauber_survey}.
Each step of the chain proposes a random color update for a random vertex, and accepts the update if it does not create a conflict.
This is an inherently sequential process, with the acceptance of a particular proposal depending on all preceding neighboring proposals.

To make the runtime analysis simpler, we define a modified version of Glauber Dynamics that proceeds in $\mathcal O(\log n)$ epochs.
In each epoch, all of the $n$ vertices propose a random color and update themselves if their proposals do not conflict with any of their neighbors.
This Markov chain is a special case of the one presented in \cite{ghaffari_fischer} for distributed graph coloring,
and mixes in $\mathcal O(\log n)$ epochs when $q\ge 9\Delta$.
%While we do not have the same restrictions as the distributed computation setting,
%we choose to use this chain so as to avoid long analysis of mixing times.
In order to implement the query $\func{Color}(v)$, it suffices to implement a query $\func{Accepted}(v,t)$
that indicates whether the proposal for $v$ was accepted in the $t^{th}$ epoch.
The answer to this question depends on the prior colors of the potentially $\Delta$ neighbors of $v$.
Naively sampling the colors of all these neighbors would result in $\Delta$ recursive invocations on the previous epoch ($t-1$),
and stepping \emph{backwards} through the epochs to find the last accepted proposal.
Naively, this leads to a bound of $\Delta^t$ on the number of recursive invocations.

We can improve this somewhat by only considering neighbors $w$ of $v$ who had any proposal for the same color $c$.
In this case the expected number of recursive calls is bounded by $t\Delta/q$
($t\Delta$ proposals to consider and each one is $c$ with probability $1/q$).
So, if $q > t\Delta = \mathcal O(\Delta\log n)$, this allows us to bound the total number of resulting invocations.
The improvement to $q\ge 9\Delta$ comes from the observation that for $w\in\Gamma(v)$ such that $w$ proposed color $c$ at epoch $t'$,
the recursive call for $w$ can jump to epoch $t'$ and then step \emph{forwards} through the epochs to find the first accepted proposal.
We show that this dramatically reduces the sub-problem size (given by $t'$) in each recursion,
thus allowing us to bound the runtime by $\mathcal O\left(t\Delta n^{6.12\Delta/q}\right)$ which is sub-linear for $q \ge 9\Delta$.





\subsection{Basic Tools for Efficient Sampling}
\label{sec:basic_tools_for_efficient_sampling}
In this section, we describe the main techniques used to sample from a distribution $\{ p_i\}_{i\in [n]}$,
which differs based on the type of access to $\{p_i\}$ provided to the algorithm.
If the algorithm is given cumulative distribution function (CDF) access to $\{p_i\}$,
then it is well known that via $\mathcal O(\log n)$ CDF evaluations, one can sample according
to a distribution that is at most $n^{-c}$ far from $\{p_i\}$ in $L_1$ distance.

Sampling can be more challenging when when we can only access the probability distribution function (PDF) of $\{p_i\}$,
The approach that we use in this work is to construct an auxiliary distribution $\{q_i\}$ such that:
(1) $\{ q_i\}$ has an efficiently computable CDF, and
(2) $q_i$ approximates $p_i$ pointwise to within a polylogarithmic multiplicative factor for ``most'' of the support of $\{ p_i\}$.
The following Lemma inspired by \cite{huge} formalizes this concept.
\begin{lemma}
\label{lem:rejection_sampling}
Let $\{p_i\}_{i\in [n]}$ and $\{q_i\}_{i\in [n]}$ be distributions on $[n]$ satisfying the following conditions:
\begin{enumerate}
    \item There is a $\log^{\mathcal O(1)}n$ time algorithm to approximate $p_i$ and $q_i$
    up to a multiplicative $\left(1\pm \frac{1}{n^c}\right)$ factor.
    \item We can generate an index $i$ according to a distribution $\{\hat q_i\}$,
    where $\hat q_i$ is a $\left(1\pm \frac{1}{n^c}\right)$ multiplicative approximation to $q_i$.
    \item There exists a $poly(\log n)$-time recognizable set $S$ such that
    \begin{itemize}
        \item $1-\sum\limits_{i\in S} p_i < \frac 1{n^c}$
        \item For every $i\in S$, it holds that $p_i\le \log^{\mathcal{O}(1)} n\cdot q_i$
    \end{itemize}
\end{enumerate}
Then, with high probabiltiy we can use only $\log^{\mathcal O(1)}n$ samples from $\{\hat q_i\}$ to generate an index $i$
according to a distribution that is $ \mathcal O\left(\frac{1}{n^c}\right)$-close to $\{p_i\}$ in $L_1$ distance.
\end{lemma}
\begin{proof}
We begin by setting an upper bound $U = \log^{\mathcal O(1)}n$ on $p_i/q_i$ for all $i\in S$.
The sampling proceeds in iterations, such that in each iteration we obtain an index $i$ according to the distribution $\{ \hat q_i\}$.
If $i\not\in S$, this index is returned with probability $\tilde p_i/U\tilde q_i$,
where $\tilde p_i$ and $\tilde q_i$ are the $\left( 1\pm \frac{1}{n^c} \right)$ multiplicative approximations to $p_i$ and $q_i$,
Otherwise, we repeat this process until some output is returned.

The probability of returning index $i\in S$ in a particular iteration is $\hat q_i\cdot (\tilde p_i/U \tilde q_i)$,
which is in turn a $\left( 1\pm \frac{1}{n^c} \right)$ multiplicative approximation to $p_i/U$.
Hence, the probability of success in a single iteration is roughly $1/U$,
and therefore we only need $\mathcal O(U\log n) = \log^{\mathcal O(1)}n$ iterations (and the same number of samples from $\{ \hat q_i\}$)
in order to succeed with high probability.
The resulting distribution of indices approximates $\{ p_i\}$ pointwise on the domain $S$, up to a factor of $\left( 1\pm \frac{1}{n^c} \right)$.
Since the remainder of the domain contains at most $ \frac{1}{n^c}$ probability mass,
the output distribution is $\mathcal O\left(\frac{1}{n^c}\right)$ close to $\{ p_i\}$ in $L_1$ distance.
\end{proof}
