\subsection{Random Coloring of a Graph}
\label{sec:overview_random_coloring_of_a_graph}
Finally, we introduce a new model (Definition~\ref{def:local_access_LCA}) for implementing huge random objects,
where the distribution is specified as a random solution to a huge combinatorial problem.
In this new setting, we will implement local query access to random $q$-colorings of a given huge graph $G$ of size $n$ with maximum degree $\Delta$.
%by implementing a query $\func{Color}(v)$, which returns the color of a \emph{single vertex} $v$.
Since the implementation has to run in sub-linear time, it is not possible to read the entire input $G$ during a single query execution.


\paragraph*{$\func{Color}$ Queries}
\label{par:color_queries}
Given a graph $G$ with maximum degree $\Delta$, and the number of colors $q\ge 9\Delta$,
we are able to construct an efficient implementation for $\func{Color}(v)$ that returns the final color of $v$
in a uniformly random $q$-coloring of $G$ using only a sub-linear number of probes.
Random colorings of a graph are sampled using $\mathcal O(n\log n)$ iterations of a Markov chain \cite{glauber_survey}.
Each step of the chain proposes a random color update for a random vertex, and accepts the update if it does not create a conflict.
This is an inherently sequential process, with the acceptance of a particular proposal depending on all preceding neighboring proposals.

To make the runtime analysis simpler, we use a modified version of Glauber Dynamics that proceeds in $\mathcal O(\log n)$ epochs.
In each epoch, all of the $n$ vertices propose a random color and update themselves if their proposals do not conflict with any of their neighbors.
This Markov chain was presented in \cite{what_local} for distributed graph coloring, and mixes in $\mathcal O(\log n)$ epochs when $q\ge 9\Delta$.
%While we do not have the same restrictions as the distributed computation setting,
%we choose to use this chain so as to avoid long analysis of mixing times.
In order to implement the query $\func{Color}(v)$, it suffices to implement $\func{Accepted}(v,t)$
that indicates whether the proposal for $v$ was accepted in the $t^{th}$ epoch.
This depends on the prior colors of the potentially $\Delta$ neighbors of $v$.
Determining the prior colors of all the neighbors $w$ using recursive calls would result in
$\Delta$ invocations of $\func{Accepted}(w, t-1)$ (at the preceding epoch $t-1$).
Naively, this gives a bound of $\Delta^t$ on the number of invocations.
We can prune the recursions by only considering neighbors who proposed the color $c$ during \emph{some} past epoch.
This reduces the expected number of recursive calls to $t\Delta/q$,
since there are $t\Delta$ potential proposals and each one is $c$ with probability $1/q$.
If $q$ is larger than $t\Delta$, the number of recursive calls is less than $1$,
which gives a sub-linear bound on the total number of resulting invocations.
Since the number of epochs $t$ can be as large as $\Theta(\log n)$, this strategy will only work when $q = \Omega(\Delta\log n)$.

The improvement to $q = \Omega(\Delta)$ follows from the observation that for a neighbor $w$ that proposed color $c$ at epoch $t'$,
the recursive call corresponding to $w$ can directly jump to epoch $t'$.
If the conflicting color $c$ was indeed accepted at that epoch,
we then step \emph{forwards} through epochs $t'+1, t'+2,\cdots$, to check whether $c$ was overwritten by some future accepted proposal.
This strategy dramatically reduces the recursive sub-problem size (given by the epoch number $t'$),
and furthermore we show that we do not have to step through too many future epochs in order to check whether $c$ was overwritten.
This allows us to bound the runtime by $\widetilde{\mathcal O}\left(t\Delta (n/\epsilon)^{6.12\Delta/q}\right)$,
where the overall coloring is sampled from a distribution that is $\epsilon$-close to uniform (see Definition~\ref{def:local_access_LCA}).

One requirement for our strategy is the ability to access a \emph{valid initial coloring} (the initial state of the Markov Chain)
through local probes, in addition to local probes to the underlying graph structure.
This assumption can be removed by using a result of \cite{coloring_initialize},
that presents an LCA for $\Delta+1$ graph coloring using $\Delta^{\mathcal O(1)}\log n$ graph probes.
Alternately, we can assign random initial colors to the vertices, which may result in an \emph{invalid} final coloring.
However, the Markov Chain will transform the initial invalid coloring into a valid one with high probability.
