\subsection{Random Coloring of a Graph}
\label{sec:overview_random_coloring_of_a_graph}
Finally, we introduce a new model for implementating huge random objects
where the distribution is specified as a uniformly random solution to a huge combinatorial problem.
In all the problems we have considered so far as well as the ones studied in prior work \cite{huge,sparse,reut}, the description size
of the random object is small (typically $\mathcal O(\log n)$ to represent the size of the instance and a constant number of parameters).
In this new setting, we will implement local query access to random $q$-colorings of a given huge graph $G$ of size $n$ with maximum degree $\Delta$.
The distribution in this case is defined by the graph structure which has size $\mathcal O(n\Delta)$.
We present the following definition for local access implementations in this setting.

\begin{definition}
\label{def:local_access_LCA}
Given a combinatorial problem on graphs,
a \emph{local access implementation} of a family of query functions $\langle F_1, F_2,\cdots \rangle$,
provides an oracle $\mathcal A$ with the following properties.
$\mathcal A$ has query access to a graph $G$, and a tape of random bits $\vec R$.
Assuming that the solution set of the combinatorial problem on $G$ is $\mathbb X$,
$\mathcal A$ upon being queried with $F_i$, returns the value $F_i(X)$ for a specific solution $X\in\mathbb X$ where the choice of $X$
depends only on $\vec R$ and the distribution of $X$ (over $\vec R$) is $\epsilon$-close to the uniform distribution over $\mathbb X$.
Two different instances of $\mathcal A$ with the same graph oracle and the same random bits,
must agree on the choice of $X$ that is consistent with all answered queries regardless of what queries were actually asked.
\end{definition}

We can contrast this definition with the one for \emph{Local Computation Algorithms} \cite{LCA, LCA_space_efficient}
which also allow query access to \emph{some} valid solution and can read the input through local probes.
An additional difficulty in our setting is that we also have to make sure that we return a solution from the correct distribution.
Similarly to LCAs, we can have multiple independent instances of our algorithm answering different queries, but remaining consistent with one another.


\paragraph*{$\func{Color}$ queries}
\label{par:color_queries}
Given a graph $G$ with maximum degree $\Delta$, and the number of colors $q\ge 9\Delta$,
we are able to construct an efficient implementation for $\func{Color}(v)$ that returns the final color of $v$
in a uniformly random $q$-coloring of $G$ using only a sub-linear number of probes.
Random colorings of a graph are sampled using $\mathcal O(n\log n)$ iterations of a Markov chain \cite{glauber_survey}.
Each step of the chain proposes a random color update for a random vertex, and accepts the update if it does not create a conflict.
This is an inherently sequential process, with the acceptance of a particular proposal depending on all preceding neighboring proposals.

To make the runtime analysis simpler, we define a modified version of Glauber Dynamics that proceeds in $\mathcal O(\log n)$ epochs.
In each epoch, all of the $n$ vertices propose a random color and update themselves if their proposals do not conflict with any of their neighbors.
This Markov chain is a special case of the one presented in \cite{mohsen} for distributed graph coloring,
and mixes in $\mathcal O(\log n)$ epochs when $q\ge 9\Delta$.
%While we do not have the same restrictions as the distributed computation setting,
%we choose to use this chain so as to avoid long analysis of mixing times.
In order to implement the query $\func{Color}(v)$, it suffices to implement a query $\func{Accepted}(v,t)$
that indicates whether the proposal for $v$ was accepted in the $t^{th}$ epoch.
The answer to this question depends on the prior colors of the potentially $\Delta$ neighbors of $v$.
Naively sampling the colors of all these neighbors would result in $\Delta$ recursive invocations on the previous epoch ($t-1$),
and stepping \emph{backwards} through the epochs to find the last accepted proposal.
Naively, this leads to a bound of $\Delta^t$ on the number of recursive invocations.

We can improve this somewhat by only considering neighbors $w$ of $v$ who had any proposal for the same color $c$.
In this case the expected number of recursive calls is bounded by $t\Delta/q$
($t\Delta$ proposals to consider and each one is $c$ with probability $1/q$).
So, if $q > t\Delta = \mathcal O(\Delta\log n)$, this allows us to bound the total number of resulting invocations.
The improvement to $q\ge 9\Delta$ comes from the observation that for $w\in\Gamma(v)$ such that $w$ proposed color $c$ at epoch $t'$,
the recursive call for $w$ can jump to epoch $t'$ and then step \emph{forwards} through the epochs to find the first accepted proposal.
We show that this dramatically reduces the sub-problem size (given by $t'$) in each recursion,
thus allowing us to bound the runtime by $\mathcal O\left(t\Delta n^{6.12\Delta/q}\right)$ which is sub-linear for $q \ge 9\Delta$.
