\subsection{Permutations}

\anak{Did we try to support more information that just computing $\pi(i)$?}
[Specification]

In addition to a table (dictionary) containing all assigned values $\pi(i)$, we maintain the following binary trees,
whose nodes are generated on-the-fly in response to queries.

$T_1$: Each node of $T_1$ corresponds to a specific range of indices,
where the root represents the entire range $\{1, \ldots, N\}$,
and its two children represents each (approximately) half of the parent's range.
Each node counts the indices $i$ in the range, such that $\pi(i) = \bot$.
Initially $T_1$ only contains the root node, and the number of unassigned indices are $N$.
The children are only generated when we need to traverse down from the root; as these nodes are generated,
all of the indices in their ranges are unassigned.
Once an index becomes assigned, we simply update the information along the path in $\Bo(\log n)$ time.

$T_2$: $T_2$ is similar to $T_1$ but instead of maintaining the number of indices in the range that are still unassigned,
it maintains the number of values in the range that are still unused (have not been assigned to an index).
Similarly, we may sample an unused value or mark it as used within $\Bo(\log n)$ time.

To compute $\pi(i)$, first we check the table for $\pi(i)$ and return its value if $\pi(i)\neq\bot$.
Otherwise, sample an unused value $j$ from $T_2$ and mark that value as used.
Add $\pi(i) = j$ to the table, and mark index $i$ as used on $T_1$.

If we wish to support $\pi^{-1}(i)$, then also store the table of $\pi^{-1}(i)$.
To assign $\pi^{-1}(i)$, sample an unassigned index $j$ from $T_1$ then mark it as assigned,
add $\pi(j)=i$ and $\pi^{-1}(i)=j$ to the table, and mark the value $i$ as used on $T_2$.

\subsection{Generating Permutations with given Cyclic Structure}
We will use the technique from \cite{cyclic} to locally generate a random permutation with a given cyclic structure.
This algorithm uses two permutations $\pi$ and $\sigma$,
where $\pi$ is a uniformly random permutation and $\sigma$ is a fixed permutation with the given structure.
The resulting random permutation is formed by the composition $\pi^{-1}(\sigma(\pi(\cdot)))$.
We will generate the permutation $\pi$ as described in the previous section.;



We receive as input a list of cycle sizes \{$c_1, c_2,\cdots, c_k\}$ with the restriction that $\sum c_i = n$.
Now we need to locally generate the permutation $\sigma$ with the prescribed structure.
Define the indices $C_j = \SL{i=1}{j}c_i$ with $C_0 = 0$.
We will construct the cycle corresponding to $c_i$ as all the elements in the interval $\{c_{i-1}+1,c_{i-1}+2,\cdots,c_i\}$.

Of course we will not be computing $\sigma$ explicitly.
Instead, we will pre-compute the $C_j$ indices, and when given a query $\sigma(x)$,
we binary search amongst $C_j$ to find the cycle that $x$ belongs to.
Then we can report the value of $\sigma(x)$ accordingly.

So, we can now compose the generator oracles for $\pi$, $\sigma$, and $\pi^{-1}$ to get the full generator. 
