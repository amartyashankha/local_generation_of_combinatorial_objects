\subsection{Path Segments Close to Zero}
The problem arises when we $k <\mathcal{O}(\log n)\sqrt{S}$. In this case we need to compute the actual probability,
Using the formula from \cite{trap}, we find that.
\begin{align}
D_{left} = {{S}\choose{D-d}}-{{S}\choose{D-d-k}} &&D_{right} = {{S}\choose{U-d}}-{{S}\choose{U-d-k'}}
\end{align}
Here, $k' = k+2U-2D$, and so $k' = \Bo(\log n)\sqrt S$.\todo{prove using Lemma~\ref{lem:dyck_var0}}

Finally, we compute the total number of Dyck paths as
$$
D_{tot} = {{2S}\choose{2D}}-{{2S}\choose{2D-k}}
$$

Now, we are going to use the following Lemma from \cite{huge}.
\begin{lemma}
\label{lem:huge}
Let $\{p_i\}$ and $\{q_i\}$ be distributions satisfying the following conditions
\begin{enumerate}
    \item There is a poly-time algorithm to approximate $p_i$ and $q_i$ up to $\pm n^{-2}$
    \item Generating an index $i$ according to $q_i$ is closely implementable.
    \item There exists a $poly(log n)$-time recognizable set $S$ such that
    \begin{itemize}
        \item $1-\SL{i\in S}{} p_i$ is negligible
        \item There exists a constant $c$ such that for every $i$, it holds that $p_i\le \log^{\mathcal{O}(1)} n\cdot q_i$
    \end{itemize}
\end{enumerate}
Then, generating an index $i$ according to the distribution $\{p_i\}$ is closely-implementable.
\end{lemma}

In this process, we will first disregard all values of $d$ where $|d|>\Theta(\sqrt S)$.
The probability mass associated with these values can be shown to be negligible \todo{bound variance of path}.

Next, we will construct an appropriate $\{q_i\}$ and show that $p_d < \log^{\mathcal{O}(1)} n\cdot q_d$
for all $|d|<\Theta(\sqrt S)$ and some constant $c$.
We will use the following distribution
$$
q_d = \frac{{S\choose D-d}\cdot{S\choose D+d}}{{2S\choose 2D}} = \frac{{S\choose D-d}\cdot{S\choose U-d}}{{2S\choose 2D}}
$$
It is shown in \cite{huge} that this distribution is closely implementable.

\begin{lemma}
First we show that $D_{left} \le \frac{c_1\cdot k}{\sqrt{S}}\cdot{{S}\choose{D-d}}$ for some constant $c_1$.
\end{lemma}
\begin{proof}
This involves some simple manipulations.
\begin{align}
D_{left} &= {{S}\choose{D-d}}-{{S}\choose{D-d-k}}\\
&= {{S}\choose{D-d}}\cdot \left[1-\frac{(D-d)(D-d-1)\cdots(D-d-k+1)}{(S-D-d+k)(S-D-d+k-1)\cdots(S-D-d+1)}\right]\\
&\le {{S}\choose{D-d}}\cdot \left[1-\left(\frac{D-d-k+1}{S-D+d+k}\right)^k\right]\\
&\le {{S}\choose{D-d}}\cdot \left[1-\left(\frac{U+d+k-(U-D+d+k-1)}{U+d+k}\right)^k\right]\\
&\le {{S}\choose{D-d}}\cdot \left[1-\left(\frac{U+d+k-\Bo(\sqrt{U})}{U+d+k}\right)^k\right]\\
&\le \frac{k}{\Theta(\sqrt{S})}\cdot{{S}\choose{D-d}}
\end{align}
\end{proof}

\begin{lemma}
Similarly, we show that $D_{right} < \frac{c_2\cdot k'}{\sqrt{S}}\cdot{{S}\choose{U-d}}$ for some constant $c_2$.
\end{lemma}
\begin{proof}
\begin{align}
D_{right} &= {{S}\choose{U-d}}-{{S}\choose{U-d-k'}}\\
&= {{S}\choose{U-d}}\cdot \left[1-\frac{(U-d)(U-d-1)\cdots(U-d-k'+1)}{(S-U-d+k')(S-U-d+k'-1)\cdots(S-U-d+1)}\right]\\
&\le {{S}\choose{U-d}}\cdot \left[1-\left(\frac{U-d-k'+1}{S-U+d+k'}\right)^{k'}\right]\\
&\le {{S}\choose{U-d}}\cdot \left[1-\left(\frac{2D-U-d-k+1}{2U-D+k+d}\right)^{k'}\right]\\
&\le {{S}\choose{U-d}}\cdot \left[1-\left(\frac{U+k+d - (2U-2D+2d+2k-1)}{U+k+d}\right)^{k'}\right]\\
&\le {{S}\choose{U-d}}\cdot \left[1-\left(\frac{U+k+d - \Bo(\sqrt U)}{U+k+d}\right)^{k'}\right]\\
&\le \frac{k'}{\Theta(\sqrt{S})}\cdot{{S}\choose{U-d}}
\end{align}
\end{proof}

Finally, we need to lower bound the value of $D_{tot}$.

\begin{lemma}
We claim that $D_{tot} < \frac{c_3\cdot k\cdot k'}{S}\cdot{{2S}\choose{2D}}$ for some constant $c_3$.
\end{lemma}
\begin{proof}
\begin{align}
D_{tot} &= {{2S}\choose{2D}}-{{2S}\choose{2D-k}}\\
&= {{2S}\choose{2D}}\cdot \left[1-\frac{(2D)(2D-1)\cdots(2D-k+1)}{(2S-2D+k)(2S-2D+k-1)\cdots(2S-2D+1)}\right]\\
&\ge {{2S}\choose{2D}}\cdot \left[1-\left(\frac{2D-k+1}{2S-2D+1}\right)^k\right]\\
&\ge {{2S}\choose{2D}}\cdot \left[1-\left(\frac{2U-(2U-2D+k-1)}{2U+1}\right)^k\right]\\
&\ge {{2S}\choose{2D}}\cdot \left[1-\left(\frac{(2U+1)-k'}{2U+1}\right)^k\right]\\
&\ge \frac{k\cdot k'}{\Theta(S)}\cdot{{2S}\choose{2D}}
\end{align}
\end{proof}

\begin{theorem}
We can now put these lemmas together to show that $p_d/q_d \le c = c_1\cdot c_2/c_3 = \Theta(1)$.
This satisfies all the conditions of Lemma~\ref{lem:huge} from \cite{huge}.
We simply need to set the accept probability less than $p_d/(c\cdot q_d)$.
\end{theorem}
