\subsubsection{The Simple Case}
The problem of sampling reduces to the binomial sampling case when $k > \mathcal{O}(\log n)\sqrt S$ for some constant $c$.
This is because with high probability, will never dip below the threshold.
In this case, the we can simply approximate the probability as
$$
\frac{{{S}\choose{D-d}}\cdot{{S}\choose{D+d}}}{{{2S}\choose{2D}}}
$$
This is because unconstrained random walks will not dip below the $1-k$ threshold with high probability.
This problem was solved in \cite{huge} using $\mathcal O(poly(\log n))$ resources.

\subsubsection{Path Segments Close to Zero}
The problem arises when we $k <\mathcal{O}(\log n)\sqrt{S}$. In this case we need to compute the actual probability,
Using the formula from \cite{trap}, we find that.
{\scriptsize
    \begin{align}
        D_{left} = {{S}\choose{D-d}}-{{S}\choose{D-d-k}}
        &&D_{right} = {{S}\choose{U-d}}-{{S}\choose{U-d-k'}}
        &&D_{tot} = {{2S}\choose{2D}}-{{2S}\choose{2D-k}}
    \end{align}
}
Here, $k' = k+2U-2D$, and so $k' = \Bo(\log n)\sqrt S$ (using Lemma~\ref{lem:dyck_var0}).

The final distribution we wish to sample from is given by $\{p_d\}_d$ where $p_d = \frac{D_{left}\cdot D_{right}}{D_{tot}}$.
To achieve this, we will use Lemma~\ref{lem:huge} from \cite{huge}.
An important point to note is that in order to apply this lemma, we must be able to compute the $p_d$ values.
For now, we will assume that we have access to an oracle that will compute the value for us.
Later, in Section~\ref{sec:}, we will see how to construct such an oracle.
\todo{Fix reference}

In this process, we will first disregard all values of $d$ where $|d|>\Theta(\log n\sqrt S)$.
The probability mass associated with these values can be shown to be negligible \todo{bound variance of path}.

Next, we will construct an appropriate $\{q_i\}$ and show that $p_d < \log^{\mathcal{O}(1)} n\cdot q_d$
for all $|d|<\Theta(\sqrt S)$ and some constant $c$.
We will use the following distribution
$$
q_d = \frac{{S\choose D-d}\cdot{S\choose D+d}}{{2S\choose 2D}} = \frac{{S\choose D-d}\cdot{S\choose U-d}}{{2S\choose 2D}}
$$
It is shown in \cite{huge} that this distribution is closely implementable.

First, we consider the case where $k\cdot k'\le 2U+1$.
In this case, we use loose bounds for $D_{left} < \binom{S}{D-d}$ and $D_{right} < \binom{S}{U-d}$.
We also use the following lemma (proven in Section~\ref{sec:dyck_appendix}).

\begin{restatable}{lemma}{DTotalFarBoundary}
\label{lem:DTotalFarBoundary}
When $kk' > 2U + 1$, $D_{tot} > \frac 12\cdot \binom{2S}{2D}$.
\end{restatable}

Combining the three bounds we obtain $p_d < \frac 12 q_d$.
Intuitively, in this case the dyck boundary is far away, and therefore the number of possible paths
is only a constant factor away from the number of unconstrained paths (no boundary).

The case where the boundaries are closer (i.e. $k\cdot k' \le 2U+1$) is trickier,
since the individual counts need not be close to the corresponding binomial counts.
However, in this case we can still ensure that the sampling probability is within
poly-logarithmic factors of the binomial sampling probability.
We use the following lemmas (proven in Section~\ref{sec:dyck_appendix}).

\begin{restatable}{lemma}{DLeftBound}
\label{lem:DLeftBound}
$D_{left} \le c_1 \frac{ k\cdot\log n}{\sqrt{S}}\cdot{{S}\choose{D-d}}$ for some constant $c_1$.
\end{restatable}

\begin{restatable}{lemma}{DRightBound}
\label{lem:DRightBound}
$D_{right} < c_2 \frac{k'\cdot log n}{\sqrt{S}}\cdot{{S}\choose{U-d}}$ for some constant $c_2$.
\end{restatable}

\begin{restatable}{lemma}{DTotalNearBoundary}
\label{lem:DTotalNearBoundary}
When $kk' \le 2U + 1$, $D_{tot} < c_3 \frac{k\cdot k'}{S}\cdot{{2S}\choose{2D}}$ for some constant $c_3$.
\end{restatable}

We can now put these lemmas together to show that $p_d/q_d \le \Theta(\log^2 n)$.
Now, we can apply Lemma~\ref{lem:huge} to sample the value of $d$,
which gives us the height of the Dyck path at the midpoint of the two given points.

\begin{theorem}
\label{thm:dyck_midpoint_sampling}
There is an algorithm that given two points at distance $a$ and $b$ (with $a < b$) along a Dyck path of length $2n$,
with the guarantee that no position between $a$ and $b$ has been sampled yet,
returns the height of the path halfway between $a$ and $b$.
Moreover, this algorithm only uses $\mathcal O(poly(\log n))$ resources.
\end{theorem}
\begin{proof}
If $b-a$ is even, we can set $S = (b-a)/2$.
Otherwise, we first sample a single step from $a$ to $a+1$, and then set $S = (b-a-1)/2$.
Since there are only two possibilities for a single step,
we can explicitly compute an approximation of the probabilities, and then sample accordingly.
Now, if $S > \Theta(\log^2 n)$ we can simply use the rejection sampling procedure described above
to obtain a $\mathcal O(poly(\log n))$ algorithm.
Otherwise, we sample each step induividually.
Since there are only $2S = \Theta(\log^2 n)$ steps, the sampling is still efficient.
\end{proof}

\begin{theorem}
\label{thm:dyck_sampling}
There is an algorithm that provides sample access to a Dyck path of length $2n$,
by answering queries of the form \func{Height}$(x)$ with the correctly sampled height of the Dyck path at position $x$
using only $\mathcal O(poly(\log n))$ resources per query.
\end{theorem}
\begin{proof}
The algorithm maintains a successor-predecessor data structure (e.g. Van Emde Boas tree)
to store all positions $x$ that have already been queried.
Each newly queried position is added to this structure.
Given a query \func{Height}$(x)$, the algorithm first finds the successor and predecessor
(say $a$ and $b$) of $x$ among the alredy queried positions.
This provides us the quarantee required to apply Theorem~\ref{thm:dyck_midpoint_sampling},
which allows us to query the height at the midpoint of $a$ and $b$.
We then binary search by updating either the successor or predecessor of $x$.
Once the interval length becomes less than $\Theta(\log^2 n)$,
we perform the full sampling (as in Theorem~\ref{thm:dyck_midpoint_sampling}) which provides us the height at position $x$.
\end{proof}

