\section{Introduction}

The problem of computing local information of huge random objects
was pioneered in \cite{huge_old,huge}. 
Further work of \cite{sparse} considers the generation of sparse random $G(n,p)$ graphs
from the Erd\"{o}s-R\'{e}nyi model \cite{er}, with $p = O(\poly(\log n)/n)$,
which answers $\poly(\log n)$ \func{all-neighbors} queries,
listing the neighbors of queried vertices.
While these generators use polylogarithmic resources over their entire execution, 
they generate graphs that are  only guaranteed to {\em appear random} to algorithms
that inspect a {\em limited portion} of the generated graph.

In \cite{reut}, the authors construct an oracle for the generation of recursive trees,
and BA preferential attachment graphs.
Unlike \cite{sparse}, their implementation allows for an arbitrary number of queries.
This result is particularly interesting --  
although the graphs in this model are generated via a sequential process,
the oracle is able to locally generate arbitrary portions of it
and answer queries in polylogarithmic time.
Though preferential attachment graphs are sparse,
they contain vertices of high degree,
thus \cite{reut} provides access to 
the adjacency list through \func{next-neighbor} queries.
%On a high level, this is due to the fact that the BA model is directed, and all edges are independent.

In this work, we begin by \emph{formalizing} a model of local-access generators
implicitly used in \cite{reut}.
We next construct oracles that allow queries to both the adjacency matrix
and adjacency list representation of a basic class of random
graph families, without generating the entire graph at the onset.
Our oracles
provide \func{vertex-pair}, \func{next-neighbor}, and \func{random-neighbor} queries\footnote{\func{vertex-pair}$(u,v)$ returns whether $u$ and $v$ are adjacent, \func{next-neighbor}$(v)$ returns a new neighbor of $v$ each time it is invoked (until none is left), and \func{random-neighbor}$(v)$ returns a uniform random neighbor of $v$ (if $v$ is not isolated).}
for graphs with {\em independent edge probabilities}, that is,
when each edge is chosen as an independent Bernoulli random variable. 
Using this framework, we construct the first \emph{efficient} local-access generators for undirected graph models, supporting all three types of queries
using $\mathcal{O}(\poly(\log n))$ time, space, and random bits
per query, under assumptions on the ability to compute certain values
pertaining to consecutive edge probabilities. In particular, our construction yields local-access generators for the Erd\"{o}s-R\'{e}nyi $G(n,p)$ model (for \emph{all} values of $p$), and the Stochastic Block model with random community assignment. 
As in \cite{reut} (and unlike the generators in \cite{huge_old,huge,sparse}), 
our techniques allow unlimited queries.%(i.e. the entire graph can be generated).

While \func{vertex-pair} and \func{next-neighbor} queries, as well as \func{all-neighbors} queries for sparse graphs, have been considered in the prior works of \cite{reut, huge_old, huge, sparse}, we provide the first implementation (to the best of our knowledge)
of \func{random-neighbor} queries, which do not follow trivially from the
\func{all-neighbor} queries in \emph{non-sparse graphs}.
Such queries are useful, for instance, for sub-linear algorithms that employ random walk processes.
\func{random-neighbor} queries
present particularly interesting challenges,  since as we note in 
Section~\ref{par:random_neighbor_queries},
(1) \func{random-neighbor} queries affect the conditional probabilities
of the remaining neighbors in a non-trivial manner, and
(2) our implementation does not resort to explicitly sampling the degree of any vertex in order to generate a random neighbor.
First, sampling the degree of the query vertex, we suspect, is not viable for \emph{sub-linear} generators,
because this quantity alone imposes dependence on the existence of \emph{all} of its potential incident edges.
Therefore, our generator needs to return a random neighbor, with probability reciprocal to the query vertex's degree,
without resorting to ``knowing'' its degree.
Second, even without committing to the degrees, answers to \func{random-neighbor} querie
affect the conditional probabilities of the remaining adjacencies in a global and non-trivial manner
-- that is, from the point of view of the \emph{agent} interacting with the generator.
The generator, however, must somehow maintain and leverage its additional \emph{internal knowledge}
of the partially-generated graph, to keep its computation tractable throughout the entire graph generation process.

We then consider local-access generators for directed graphs in Kleinberg's Small World model.
In this case, the probabilities are based on distances in a 2-dimensional grid.
Using a modified version of our previous sampling procedure,
we present such a generator supporting \func{all-neighbors} queries in 
$\mathcal{O}(\poly(\log n))$ time, space and random bits per query (since such graphs
are sparse, the other queries follow directly).

For additional related work, see Section~\ref{sec:related_work}.

%\subsection{Our results and techniques}

\anak{TODO: Is this subsection presented well? Are we mixing our results with related works too much here? Should we add a table (arxiv version)? Does it overlap too much with the above stuff? Should we highlight multivariate hypergeometric distribution?}

Our work provides local-access generators for the following
aforementioned three families of random graphs, 
where each query is processed using $\poly(\log n)$ 
time, random bits, and additional space, with no initialization overhead. 
Assuming constant computation time for each arithmetic operation with
$O(\log n)$-bit precision, each of our generators constructs a random graph
drawn from a distribution that is $\frac{1}{\poly(n)}$-close
to the desired distribution in the $L_1$-distance.
\footnote{The \emph{$L_1$-distance} between two probability distributions $\distr{p}$ 
and $\distr{q}$ over domain $D$ is defined as $\|\distr{p-q}\|_1 = 
\sum_{x \in D } |p(x)-q(x)|$.
We say that $\distr{p}$ and $\distr{q}$ are $\epsilon$-close if $\|\distr{p-q}\|_1 \leq \epsilon$. 
%Note that the \emph{total variation distance} is related to the $L_1$-distance as $d_{\mathrm{TV}}(\matr{p},\matr{q})=\frac{1}{2}\|\distr{p-q}\|_1$.
}
\anak{I saw \href{https://stat.mit.edu/calendar/optimal-lower-bounds-for-universal-relation-and-for-samplers-and-finding-duplicates-in-streams}{Jelani's 9/29 talk abstract} describing this ``$L_1$-close'' output as: with prob $1-\epsilon$ the generator outputs something close to the desired distribution; with prob $\epsilon$ can do anything, like fail or output something outside the support. That's an alternative to giving an explicit definition, I suppose; also a good sign we're consider a suitable notion.}

The main complication in our setup, as compared to \cite{reut},
arises from the fact that our graph is undirected.
Each next neighbor query, can affect the probabilties
of a large number of other vertices.
For instance, if $u_1$, and $u_2$ are sampled as two consecutive neighbors of $v$,
we have to maintain consistency in all subsequent steps,
by ensuring that none of the vertices in the range $(u_1,u_2)$
return $v$ as a neighbor.

We present both a deterministic, and a randomized strategy to deal with this.
The randomized algorithm, presented in Section~\ref{sec:ER-rand},
uses the result from Lemma~\ref{alg:oblivious-coin-toss}.
The deterministic algorithm is presented in Section~\ref{sec:ER-det}.

\paragraph{Erd\"{o}s-R\'{e}nyi model}
First, we design generators for constructing $G(n,p)$ graphs.
%Our final local-access implementation answers both adjacency matrix and adjacency list queries
%with $poly(\log n)$ overhead. It is worth noting that if we use this to generate the entire graph,
%we obtain an almost optimal \emph{global} generator that takes $\tilde{\Bo}(m)$ time.
%This matches the bound presented in \cite{er_gen}, upto $\Bo(\log n)$ factors.
We first provide a generator supporting \func{next-neighbor} queries using $\poly(\log n)$ resources per query \emph{in the worst case}:  
In particular, note that when the graph is sparse,
maintaining an adjacency matrix is impossible as a \func{next-neighbor} call may ``skip'' as many as $\Theta(n)$ vertices, and updating this matrix would take linear time. 
Our implementation allows access to the graph's adjacency list representation, enumerating all neighbors of each queried vertex in the lexicographical order. 
We remark that, while $\Omega(n+m)$ time is clearly necessary to generate a full random graph, our implementation supports the local-access via \func{next-neighbor} queries, and yet can generate a full graph within $\widetilde{O}(n+m)$ time, which is tight up to polylogarithmic factors.

We avoid making random choices for all edges by sampling for the index of the next neighbor directly using the geometric distribution (used in \cite{er_gen}).
The difficulty of this approach lies in succinctly and efficiently maintaining the state of the generated graph.
Finding the next neighbor of $v$ requires verifying whether each vertex $u$ has already been decided as a neighbor or a non-neighbor of $v$ during a previous \func{next-neighbor}$(u)$ call, or whether $u$ remains a potential neighbor.
Instead, we observe that it is sufficient to simply maintain the last returned neighbor and all known neighbors for each vertex to compute the status of $u$. We then design a data structure (Section~\ref{sec:ER-det}) that counts and samples the next neighbor while restricting to only the potential neighbors of $v$, and prove that given accurate samples from the geometric distribution one may achieve the desired guarantee in the $L_1$-distance. Additionally, we augment our implementation so that \func{vertex-pair} queries are also supported in $\poly(\log n)$ \emph{amortized} time.

We also provide an alternative implementation that does not require any complicated data structure (Section~\ref{sec:ER-rand}).
Instead, it samples for the next neighbor without excluding the known non-neighbors,
then retries the sampling process if it samples a non-neighbor.
While this approach may encounter many failures, especially if many vertices has been designated as non-neighbors of $v$,
we prove that such an event is extremely unlikely to take more than $O(\log n)$ tries.
This guarantee applies for \emph{arbitrary probabilities} and an
\emph{adversarial} series of queries,
even when the adversary knows the random bits used by the algorithm.
Our generator answers every query using $\poly(\log n)$ resources per query with high probability.
\footnote{An event happens \emph{with high probability} if it holds with probability $\ge 1-n^{-c}$ for any constant $c > 0$.}

\paragraph{Stochastic Block model}
We generalize the $G(n,p)$ construction to the Stochastic Block Model under random community assignment.
Our approach is similarly to sample for the next neighbor directly,
although it does not simply follow the geometric distribution,
as the probabilities for the potential edges now depend on the communities of endpoints.
To handle this issue, we observe that it is sufficient to count
the number of vertices of each community in any interested range of contiguous vertex indices.
We then design a data structure extending a construction of \cite{huge},
which maintain these counts for ranges of vertices, and ``sample'' the partition of their counts only on an as-needed basis.
This extension results in an efficient technique  to sample counts from the \emph{multivariate hypergeometric distribution}.
For $r$ communities, this yields an implementation with $r\,\poly(\log n)$ overhead in required resources for each operation.
This upholds all previous polylogarithmic guarantees when $r = \poly(\log n)$.

\paragraph{Small-World model} 
We design generators for the aforementioned case of the Small-World model, supporting each \func{all-neighbors} query, listing all neighbors from closest to furthest away from the queried vertex, using $\poly(\log n)$ resources per query. Providing local access for directed graphs is simpler because the out-neighbors of vertices may be chosen independently at each vertex, so the main challenge is to sample for the next (closest) neighbor, when the probabilities are a function of the Manhattan distance on the lattice. Rather than sampling for a neighbor directly, we sample the next smallest distance with a neighbor, employing the rejection sampling technique that allows efficient sampling through an approximate distribution that have closed-form description, then as a second step, sample for all neighbors for each chosen distance.

\paragraph{CDF Based Sampling}
It is worth noting that our techniques for implementing local-access for the ER and SBM graphs
can easily be extended to other similar models of random graphs.
The only requirement is that the CDF of the probability sequences can be efficiently computed as in Section~\ref{para:CDF}.


\section{Our Contributions and Techniques}

We begin by formalizing a model of {\em local-access generators}
(Section~\ref{sec:oracle_model}), implicitly used in \cite{reut}.
Our work provides local-access generators for various
basic classes of graphs described in the following, with 
\func{vertex-pair}, \func{next-neighbor}, and \func{random-neighbor}
queries.  In all of our results,
each query is processed using $\poly(\log n)$
time, random bits, and additional space, with \emph{no initialization overhead}.
These guarantees hold even in the case of adversarial queries.
Our bounds assume constant computation time for each arithmetic operation with
$O(\log n)$-bit precision. Each of our generators constructs a random graph
drawn from a distribution that is $1/\poly(n)$-close
to the desired distribution in the $L_1$-distance.\footnote{The \emph{$L_1$-distance} between two probability distributions $\distr{p}$
and $\distr{q}$ over domain $D$ is defined as $\|\distr{p-q}\|_1 = 
\sum_{x \in D } |p(x)-q(x)|$.
We say that $\distr{p}$ and $\distr{q}$ are $\epsilon$-close if $\|\distr{p-q}\|_1 \leq \epsilon$.
%Note that the \emph{total variation distance} is related to the $L_1$-distance as $d_{\mathrm{TV}}(\matr{p},\matr{q})=\frac{1}{2}\|\distr{p-q}\|_1$.
}
%One of our contributions is to formalize this model (Section~\ref{sec:oracle_model}).

\subsection{Undirected Graphs}
\label{sec:undirected_graphs}

In Section~\ref{sec:undirected} we construct local access generators for the generic
class of undirected graphs
with {\em independent edge probabilities} $\left\{ p_{u,v} \right\}_{u,v\in V}$,
where $p_{u,v}$ denote the probability that there is an edge between $u$ and $v$.
Throughout, we identify our vertices via their unique IDs from $1$ to $n$, namely $V = [n]$.
We assume that we can compute various values pertaining to consecutive
edge probabilities for the class of graphs, as detailed below.
We then show that such values can be computed for graphs
generated according to the Erd\"{o}s-R\'{e}nyi $G(n,p)$ model
and the Stochastic Block model.

\paragraph{\func{next-neighbor} Queries}
\label{par:next_neighbor_queries}
We note that the next neighbor of a vertex can be found trivially by generating consecutive
entries of the adjacency matrix, but for small edge probabilities $p_{u,v} = o(1)$
this implementation can be too slow.  In our algorithms, we achieve speed-up by sampling multiple 
neighbor values at once for a given vertex $u$; more specifically,  
we sample for the number of ``non-neighbors'' preceding
the next neighbor.
To do this, we assume that we have access to
an oracle which can estimate the ``skip'' probabilities 
$F(v,a,b)=\prod^{b}_{u=a} (1-p_{v,u})$,
where $F(v,a,b)$ is the probability that $v$ 
has no neighbors in the range $[a,b]$.
We later show that it is possible to compute this quantity efficiently
for the $G(n,p)$ and Stochastic block models.

A main difficulty in our setup, as compared to \cite{reut},
arises from the fact that our graph is undirected, and thus
we must design a data structure that ``informs'' all (potentially $\Theta(n)$) non-neighbors once we decide on the query vertex's next neighbor.
%Unlike in the case of directed graphs, each $\func{next-neighbor}$ query
%can affect the probabilities of a large number of other vertices.
More concretely, if $u'$ is sampled as the next neighbor of $v$ after its previous neighbor $u$,
we must maintain consistency in subsequent steps
by ensuring that none of the vertices in the range $(u,u')$
return $v$ as a neighbor. This update will become even more complicated as we later handle \func{random-neighbor} queries, where we may generate non-neighbors at random locations.

In Section~\ref{sec:ER-rand}, we present a very simple randomized generator
(Algorithm~\ref{alg:oblivious-coin-toss}) that supports \func{next-neighbor}
queries efficiently, albeit the analysis of its performance is rather complicated.
We remark that this approach may be extended to support \func{vertex-pair} queries with superior performance (given that we do not to support \func{random-neighbor} queries) and to provide deterministic resource usage guarantee -- the full analysis can be found in Section~\ref{sec:reroll-cont} and \ref{sec:ER-det}, respectively.

\paragraph{\func{random-neighbor} Queries}
\label{par:random_neighbor_queries}
We provide efficient \func{random-neighbor} queries (Section~\ref{sec:buckets}).
The ability to do so is surprising.  First, note that after performing a \func{random-neighbor} query
all other conditional probabilities will be affected in a non-trivial way.
\footnote{Consider a $G(n,p)$ graph with small $p$, say $p = 1/\sqrt n$,
such that vertices will have $\tilde{\mathcal{O}}(\sqrt n)$ neighbors with high probability.
After $\tilde{\mathcal{O}}(\sqrt n)$ \func{random-neighbor} queries,
we will have uncovered all the neighbors (w.h.p.),
so that the conditional probability of the remaining
$\Theta(n)$ edges should now be close to zero.}
This requires a way of implicitly keeping track of all the resulting changes.
Second, we can sample a \func{random-neighbor} with the correct probability $1/\deg(v)$,
even though we do not sample or know the degree of the vertex.

We formulate a {\em bucketing approach} (Section~\ref{sec:buckets})
which samples multiple consecutive edges at once, in such a way
that the conditional probabilities of the unsampled edges
remain independent and ``well-behaved'' during subsequent queries.
For each vertex $v$, we divide the vertex set (potential neighbors) or $v$ into consecutive ranges (buckets),
so that each bucket contains, in expectation, roughly the same number of neighbors
$\sum^{b}_{u=a} p_{v,u}$ (which we must be able to compute efficiently).
The subroutine of \func{next-neighbor} may be applied to sample the neighbors within a bucket in expected constant time.
Then, one may obtain a random neighbor of $v$ by picking a random neighbor from a random bucket;
probabilities of picking any neighbors may be normalized to the uniform distribution via rejection sampling,
while stilling yielding $\poly(\log n)$ complexities overall.
This bucketing approach also naturally leads to our data structure that requires
constant space for each bucket and for each edge, using $\Theta(n+m)$ overall memory requirement.
The \func{vertex-pair} queries are implemented by sampling the relevant bucket.

We now consider the application of our construction above to actual random graph models,
where we must realize the assumption that $\prod^{b}_{u=a} (1-p_{v,u})$
and $\sum^{b}_{u=a} p_{v,u}$ can be computed efficiently.
This holds trivially for the $G(n,p)$ model via closed-form formulas,
but requires an additional back-end data structure for the Stochastic Block models.

\paragraph{Erd\"{o}s-R\'{e}nyi}
\label{par:erdos_renyi}
In Section~\ref{sec:app_er}, we apply our construction to random $G(n,p)$ graphs for
arbitrary $p$, and obtain
$\func{vertex-pair}$, $\func{next-neighbor}$, and $\func{random-neighbor}$ queries,
using polylogarithmic resources (time, space and random bits) per query.
We remark that, while $\Omega(n+m) = \Omega(p n^2)$ time and space
is clearly necessary to generate and represent a full random graph,
our implementation supports local-access via all three types of queries, 
and yet can generate a full graph in $\widetilde{O}(n+m)$ time and space (Corollary~\ref{thm:er-optimal}),
which is tight up to polylogarithmic factors.

%{\color{red}
\paragraph{Stochastic Block Model}
\label{par:stochastic_block_model}
We generalize our construction to the Stochastic Block Model.
In this model, the vertex set is partitioned into $r$ \emph{communities}
$\left\{ C_1, \ldots, C_r \right\}$.
The probability that an edge exists depends on the communities of its endpoints:
if $u\in C_i$ and $v \in C_j$, then $\{u,v\}$ exists with probability $p_{i,j}$,
given in an $r\times r$ matrix $\matr{P}$.
As communities in the observed data are generally unknown a priori,
and significant research has been devoted to designing efficient algorithm
for community detection and recovery,
these studies generally consider the \emph{random community assignment} condition for the purpose of designing and analyzing algorithms (see e.g., \cite{mossel2015reconstruction}).
Thus, in this work, we aim to construct generators for this important case, where the community assignment of vertices are independently sampled from some given distribution $\distr{R}$.
%}

Our approach is, as before, to sample for the next neighbor or a random neighbor directly,
although our result does not simply follow closed-form formulas,
as the probabilities for the potential edges now depend
on the communities of endpoints.
To handle this issue, we observe that it is sufficient to efficiently count
the number of vertices of each community in any
range of contiguous vertex indices.
We then design a data structure extending a construction of \cite{huge},
which maintain these counts for ranges of vertices,
and ``sample'' the partition of their counts only on an as-needed basis.
This extension results in an efficient technique to sample counts
from the \emph{multivariate hypergeometric distribution} (Section~\ref{sec:partition}).
This sampling procedure may be of independent interest.
For $r$ communities, this yields an implementation with
$ \mathcal{O}(r\cdot \poly(\log n))$ overhead in required resources for each operation.
This upholds all previous polylogarithmic guarantees when $r = \poly(\log n)$.

\subsection{Directed Graphs}
\label{sec:directed_graphs}

Lastly, we consider Kleinberg's Small World model (\cite{kleinberg, klein}) in Section~\ref{sec:small_world}.
While Small-World models are proposed to capture 
properties of observed data such as small shortest-path 
distances and large clustering coefficients \cite{watts1998collective}, 
this important special case of Kleinberg's model, defined on two-dimensional grids, 
demonstrates underlying geographical structures of networks. 
The vertices are aligned on a $\sqrt{n}\times\sqrt{n}$ grid,
and the edge probabilities are a function of a two-dimensional distance metric.
Since the degree of each vertex in this model is $\Bo(\log n)$ with high probability,
we design generators supporting \func{all-neighbor} queries.
%In contrast to our previous cases, this model imposes an underlying
%two-dimensional structure of the vertex set, which governs
%the distance function as well as complicates the individual edge probabilities.


%\section{Additional related work}
\label{sec:additional_related_work}

\paragraph*{Random graph models}
The Erd\"{o}s-R\'{e}nyi model, given in \cite{er}, is one of the most simple theoretical random graph model,
yet more specialized models are required to capture properties of real-world data.
The Stochastic Block model (or the planted partition model) was proposed in \cite{holland} originally for modeling social networks;
nonetheless, it has proven to be an useful general statistical model in numerous fields,
including recommender systems \cite{rec0,rec1}, medicine \cite{med0}, social networks \cite{social0,social1},
molecular biology \cite{bio0,bio1}, genetics \cite{gene0,gene1,gene2}, and image segmentation \cite{img0}.
Canonical problems for this model are the community detection and community recovery problems:
some recent works include \cite{chin2015stochastic,mossel2015reconstruction,abbe2015community,abbe2016exact};
see e.g., \cite{abbe} for survey of recent results.
The study of Small-World networks is originated in \cite{watts1998collective} has frequently been observed,
and proven to be important for the modeling of many real world graphs such as social networks \cite{small0, small1},
brain neurons \cite{bassett2006small}, among many others.
Kleinberg's model on the simple lattice topology (as considered in this paper) imposes a geographical that allows navigations,
yielding important results such as routing algorithms (decentralized search) \cite{kleinberg, klein}.
See also e.g., \cite{newman2000models} and Chapter 20 of \cite{easley2010networks}.

\paragraph*{Generation of random graphs}
The problem of local-access implementation of random graphs has been considered in the aforementioned work \cite{huge_old,sparse,reut},
as well as in \cite{mansour} that locally generates out-going edges on bipartite graphs while minimizing the maximum in-degree.
The problem of generating full graph instances for random graph models have been frequently considered in many models of computations,
such as sequential algorithms \cite{milo2003uniform,er_gen,nobari2011fast,miller2011efficient},
and the parallel computation model \cite{alam2017parallel}.

\paragraph*{Query models}
In the study of sub-linear time graph algorithms where reading the entire input is infeasible,
it is necessary to specify how the algorithm may access the input graph,
normally by defining the type of queries that the algorithm may ask about the input graph;
the allowed types of queries 
can greatly affect the performance of the algorithms.
While \func{Next-Neighbor} query is only recently considered in \cite{reut},
there are other query models providing a neighbor of a vertex,
such as asking for an entry in the adjacency-list representation \cite{goldreich1997property},
or traversing to a random neighbor. On the other hand,
the
\func{Vertex-Pair} query is common in the study of dense graphs as accessing the adjacency matrix representation \cite{goldreich1998property}.
The \func{All-Neighbors} query has recently been explicitly considered in local algorithms \cite{feige2017probe}.

Other constructions of huge pseudorandom functions that are permutations or random hash functions were given in \cite{luby_rackoff, naor, mansour}.

