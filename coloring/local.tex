\subsection{Local Coloring Algorithm}%
\label{sub:local_coloring_algortihm}

\begin{algorithm}[H]
\caption{Generator}
\begin{algorithmic}[1]

\Procedure{Color}{$v, T$}
    \For{$i \gets [ T, T-1, T-2 \cdots 1 ]$}
        \If {\func{Accepted}($i, v$)}
            \State \Return $\mathcal C^{(i)}_v$
        \EndIf
    \EndFor
\EndProcedure

\Procedure{Accepted}{$v, t$}
    \State {$c\gets C^{(t)}_v$}
    \For{$w \gets \Gamma(v)$}
        \State {$flag\gets \ZERO$}
        \For{$t' \gets [t, t-1, t-2, \cdots, 1]$}
            \If {$t' \not= t$ \textbf{or} $\mathcal I^{(t)}_w < \mathcal I^{(t)}_v$}
                \If {$\mathcal C^{(t')}_w = c$ \textbf{and} \func{Accepted}($w, t'$)}
                    \State $flag\gets \ONE$
                    \While{$t' < t$}
                        \State $t'\gets t' + 1$
                        \If {\func{Accepted}($w, t'$)}
                            \State $flag\gets \ZERO$
                            \State \textbf{break}
                        \EndIf
                    \EndWhile
                    \If {$flag = \ONE$}
                        \State \Return $\ZERO$
                    \EndIf
                    \State \textbf{break}
               \EndIf
           \EndIf
        \EndFor
    \EndFor
    \State \Return $\ONE
    $\EndProcedure

\end{algorithmic}
\label{alg:coloring}
\end{algorithm}




Given a vertex $v$, the local-access generator has to output the color of $v$
after running $t = \mathcal O(1)$ epochs of \emph{Modified Glauber Dynamics}.
We will define the $i^{th}$ epoch as $\chi^{(i)}$ and the number of colors as $q = \tau\Delta$ where $\tau > 1$.

Each epoch can be implemented a local generator for a permutation of $V$.
Specifically, each epoch is a sequence of vertex and color samples:
\[
\chi^{(i)}
= \left\langle (v^{(i)}_1, c^{(i)}_1), (v^{(i)}_2, c^{(i)}_2), (v^{(i)}_3, c^{(i)}_3), \cdots, (v^{(i)}_T, c^{(i)}_T)\right\rangle
\thicksim_{\mathcal U} \left( V\times [q]\right)^T
\]
Here, $\left\{ v^{(i)}_1, v^{(i)}_2,\cdots, v^{(i)}_n\right\}$ is a permutation of all the vertices.
A position $j$ in the sequence is labeled \emph{``ACCEPT''} if at the $j^{th}$ step,
$v^{(i)}_j$ was recolored to $c^{(i)}_j$ (no conflicts with neighbors).
Otherwise, position $j$ is marked \emph{``REJECT''}.
We define $X^{(i)}_v = 1$ if the sample for $v$ in the $i^{th}$ epoch $(v, c^{(i)}_j)$ is accepted and $X^{(i)}_v = 0$ otherwise.

We also define $\mathcal C^{(i)}_v$ to be the color sampled for vertex $v$ in the $i^{th}$ epoch,
and $\mathcal I^{(i)}_v$ to be the corresponding index $j$.

Algorithm~\ref{alg:coloring} shows the procedure for querying the value of $X^{(i)}_v$.

\begin{theorem}
\label{thm:coloring_recurrence}
Coloring recurrence.
\end{theorem}
\begin{proof}
Here, $\mathcal B(p)$ refers to the Bernoulli random variable with parameter $p$.
Assuming $t = k\log n$.
\begin{align}
T_{t} &\le \left|\Gamma(v)\right| \cdot \mathlarger\sum\limits_{i=1}^{t} \mathcal B\left( \frac{1}{\tau\Delta}\right)\cdot
\left[ T_{i} + T_{i+1} + \mathcal B\left(\frac{1}{\tau}\right)\cdot T_{i+2} + \mathcal B\left(\frac{1}{\tau^2}\right)\cdot T_{i+3} + \cdots\right]\\
&\le 3\Delta \cdot \mathcal B\left(\frac{1}{\tau\Delta}\right)\cdot T_{t-1} + T_{t-1}\\
\implies \mathbb E[T_t] &\le \left( 1 + \frac{3}{\tau}\right)\cdot \mathbb E[T_{t-1}]\\
\implies \mathbb E[T_t] &\le \Delta\cdot \left( 1 + \frac{3}{\tau}\right)^t = \Delta\cdot \mathlarger n^{k\log{\left(1 + \frac{3}{\tau} \right)}}
= \Delta\cdot \mathlarger n^{\frac{3k}{\tau}}
\end{align}
\end{proof}
