\section{Random Coloring of a Graph}%
\label{sec:random_coloring_of_a_graph}

We wish to locally sample an uniformly random coloring of a graph.
A $q$-coloring of a graph $G = (V, E)$ is a function $\sigma : V\rightarrow [q]$,
such that for all $(u,v)\in E$, $\sigma_u \not= \sigma_v$.
We will consider only bounded degree graphs, i.e. graphs with max degree $\le \Delta$.
Otherwise, the coloring problem becomes NP-hard\todo{cite}.

Using the technique of path-coupling, Vigoda \todo{cite} showed that for $q > 2\Delta$,
one can sample an uniformly random coloring by using a MCMC algorithm.

\subsection{Glauber Dynamics}%
\label{sub:glauber_dynamics}

The Markov Chain proceeds in $T$ steps. The state of the chain at time $t$ is given by $\vec X^t\in [q]^{|V|}$.
Specifically, the color of vertex $v$ at step $t$ is $\vec X^t_v$.

In each step of the Markov process, a pair $(v, c)\in V\times [q]$ is sampled uniformly at random.
Subsequently, if the recoloring of vertex $v$ with color $c$ does not result in a conflict with $v$'s neighbors,
i.e. $c\not\in \left\{ X^t_u : u\in \Gamma(v)\right\}$, then the vertex is recolored i.e. $X_v^{t+1}\leftarrow c$.

After running the MC for $T = \mathcal{O}(n\log n)$ steps we reach the stationary distribution ($\epsilon$ close),
and the coloring is an uniformly random one.

\textbf{Exact Bound:}
$t_{mix}(\epsilon) \le \left( \frac{q-\Delta}{q-2\Delta}\right)n\left( \log n + \log(1/\epsilon)\right)$
\todo{cite book (Peres, Lyons)}

%\begin{algorithm}[H]
    \caption{Na\"{i}ve Generator}
    \begin{algorithmic}
        \Procedure{Color}{$v, t$}
            \For{$i \gets 1 \cdots $}
            \EndFor
            \State{$S \gets U + D$}\\
            \State{$d \sim \left\{\frac{\binom{S}{S-d}\binom{S}{D+d}}{\binom{2S}{2D}}\right\}_d$}\\\\
            \State{$k' \gets k + U - D$}\\
            \State{$p_d \gets
                \frac{{{S}\choose{D-d}}-{{S}\choose{D-d-k}}{{S}\choose{U-d}}
                -{{S}\choose{U-d-k'}}}{{{2S}\choose{2D}}-{{2S}\choose{2D-k}}}$}
            \State{$q_d \gets \frac{\binom{S}{S-d}\binom{S}{D+d}}{\binom{2S}{2D}}$}\\
            \If {$p_d < q_d$}
                \State \Return $d$
            \EndIf
            \State{\textbf{draw} $X \sim\distr{Bern}(p_d/q_d)$}
            \If {$X = 0$}
                \State \Return $d$
            \EndIf
            \Return \func{Split}($U, D, k, k'$)

        \EndProcedure
    \end{algorithmic}
    \label{alg:naive_coloring}
\end{algorithm}

\subsection{Modified Glauber Dynamics}%
\label{sub:modified_glauber_dynamics}

Now we define a modified Markov Chain, with each step called an epoch.
In the $i^{th}$ epoch, denoted by $\mathcal E_i$,
\begin{itemize}
    \item Pick a random permutation $\pi^{(i)}$ of the vertices $V$.
    \item Sample $n = |V|$ colors $ \langle c_1, c_2,\cdots, c_n \rangle$ from $[q]$.
    \item Perform the standard update using the pairs
          $\left\langle (\pi^{(i)}_1, c_1), (\pi^{(i)}_2, c_2), \cdots, (\pi^{(i)}_n, c_n)\right\rangle$.
\end{itemize}

\todo{Cite Path Coupling and prove?}
For the path coupling argument, we define the standard pre-metric on the space for all possible colorings (not necessarily valid ones).
Given two colorings $X$ and $Y$, we define $d(X,Y)$ as the number of vertices $v$ such that $X_v\not= Y_v$.

We define a coupling $(X,Y)\rightarrow(X',Y')$ where $X$ and $Y$ differ only at a single vertex $v$ such that $X_v = c_X$ and $Y_v = c_Y$.
Now, we pick a random permutation of the vertices along with uniformly sampled colors:
\[
\left\langle (v_1, c_1), (v_2, c_2), \cdots, (v_n, c_n)\right\rangle
= \left\langle (\pi_1, c_1), (\pi_2, c_2), \cdots, (\pi_n, c_n)\right\rangle
\]
Now, for each $(v_i, c_i)$ in order, we update the coloring of $X$ and $Y$ as follows:
\begin{itemize}
    \item If the current color of $v_i$ as well as $c_i$ is in $\{c_X,c_Y\}$,
    then the $X$ chain picks the color $c_i$ and the $Y$ chain picks the other color.
\end{itemize}

\begin{lemma}
\label{lem:single_epoch_distance}
If $\tau \ge 7\Delta$ and $d(X, Y) = 1$, then $\mathbb E[d(X',Y')] < \frac 12$
\end{lemma}

\begin{theorem}
\label{thm:modified_mixing}
After $\mathcal O(\log n)$ epochs, the Markov Chain is mixed.
\end{theorem}
\begin{proof}
Path Coupling
\end{proof}


\subsection{Local Coloring Algorithm}%
\label{sub:local_coloring_algortihm}

\begin{algorithm}[H]
\caption{Generator}
\begin{algorithmic}[1]

\Procedure{Color}{$v, T$}
    \For{$i \gets [ T, T-1, T-2 \cdots 1 ]$}
        \If {\func{Accepted}($i, v$)}
            \State \Return $\mathcal C^{(i)}_v$
        \EndIf
    \EndFor
\EndProcedure

\Procedure{Accepted}{$v, t$}
    \State {$c\gets C^{(t)}_v$}
    \For{$w \gets \Gamma(v)$}
        \State {$flag\gets \ZERO$}
        \For{$t' \gets [t, t-1, t-2, \cdots, 1]$}
            \If {$t' \not= t$ \textbf{or} $\mathcal I^{(t)}_w < \mathcal I^{(t)}_v$}
                \If {$\mathcal C^{(t')}_w = c$ \textbf{and} \func{Accepted}($w, t'$)}
                    \State $flag\gets \ONE$
                    \While{$t' < t$}
                        \State $t'\gets t' + 1$
                        \If {\func{Accepted}($w, t'$)}
                            \State $flag\gets \ZERO$
                            \State \textbf{break}
                        \EndIf
                    \EndWhile
                    \If {$flag = \ONE$}
                        \State \Return $\ZERO$
                    \EndIf
                    \State \textbf{break}
               \EndIf
           \EndIf
        \EndFor
    \EndFor
    \State \Return $\ONE
    $\EndProcedure

\end{algorithmic}
\label{alg:coloring}
\end{algorithm}




Given a vertex $v$, the local-access generator has to output the color of $v$
after running $t = \mathcal O(1)$ epochs of \emph{Modified Glauber Dynamics}.
We will define the $i^{th}$ epoch as $\chi^{(i)}$ and the number of colors as $q = \tau\Delta$ where $\tau > 1$.

Each epoch can be implemented a local generator for a permutation of $V$.
Specifically, each epoch is a sequence of vertex and color samples:
\[
\chi^{(i)}
= \left\langle (v^{(i)}_1, c^{(i)}_1), (v^{(i)}_2, c^{(i)}_2), (v^{(i)}_3, c^{(i)}_3), \cdots, (v^{(i)}_T, c^{(i)}_T)\right\rangle
\thicksim_{\mathcal U} \left( V\times [q]\right)^T
\]
Here, $\left\{ v^{(i)}_1, v^{(i)}_2,\cdots, v^{(i)}_n\right\}$ is a permutation of all the vertices.
A position $j$ in the sequence is labeled \emph{``ACCEPT''} if at the $j^{th}$ step,
$v^{(i)}_j$ was recolored to $c^{(i)}_j$ (no conflicts with neighbors).
Otherwise, position $j$ is marked \emph{``REJECT''}.
We define $X^{(i)}_v = 1$ if the sample for $v$ in the $i^{th}$ epoch $(v, c^{(i)}_j)$ is accepted and $X^{(i)}_v = 0$ otherwise.

We also define $\mathcal C^{(i)}_v$ to be the color sampled for vertex $v$ in the $i^{th}$ epoch,
and $\mathcal I^{(i)}_v$ to be the corresponding index $j$.

Algorithm~\ref{alg:coloring} shows the procedure for querying the value of $X^{(i)}_v$.

\begin{theorem}
\label{thm:coloring_recurrence}
Coloring recurrence.
\end{theorem}
\begin{proof}
Here, $\mathcal B(p)$ refers to the Bernoulli random variable with parameter $p$.
Assuming $t = k\log n$.
\begin{align}
T_{t} &\le \left|\Gamma(v)\right| \cdot \mathlarger\sum\limits_{i=1}^{t} \mathcal B\left( \frac{1}{\tau\Delta}\right)\cdot
\left[ T_{i} + T_{i+1} + \mathcal B\left(\frac{1}{\tau}\right)\cdot T_{i+2} + \mathcal B\left(\frac{1}{\tau^2}\right)\cdot T_{i+3} + \cdots\right]\\
&\le 3\Delta \cdot \mathcal B\left(\frac{1}{\tau\Delta}\right)\cdot T_{t-1} + T_{t-1}\\
\implies \mathbb E[T_t] &\le \left( 1 + \frac{3}{\tau}\right)\cdot \mathbb E[T_{t-1}]\\
\implies \mathbb E[T_t] &\le \Delta\cdot \left( 1 + \frac{3}{\tau}\right)^t = \Delta\cdot \mathlarger n^{k\log{\left(1 + \frac{3}{\tau} \right)}}
= \Delta\cdot \mathlarger n^{\frac{3k}{\tau}}
\end{align}
\end{proof}

