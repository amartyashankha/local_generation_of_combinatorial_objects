\subsubsection{Filling a bucket}
\label{sec:bucket_filling}

We consider buckets to be in two possible states -- \filled~or \unfilled.
Initially, all buckets are considered \unfilled.
In our algorithm we will maintain, for each bucket $B^{(i)}_v$, the set $P^{(i)}_v$ of known neighbors of $u$ in bucket $B^{(i)}_v$; this is a refinement of the set $P_v$ in Section~\ref{sec:ER-rand}. 
We define the behaviors of the procedure $\func{fill}(v,i)$ as follows. When invoked on an unfilled bucket $B_v^{(i)}$, $\func{fill}(v,i)$ performs the following tasks:
\begin{itemize}
\item decide whether each vertex $u \in B_v^{(i)}$ is a neighbor of $v$ (implicitly setting $\ADJ[v][u]$ to $\ONE$ or $\ZERO$) unless $X_{v,u}$ is already decided; in other words, update $P_v^{(i)}$ to $\Gamma^{(i)}(v)$
\item mark $B_v^{(i)}$ as \filled.
\end{itemize}
For the sake of presentation, we postpone our description of the implementation of $\func{fill}$ to Section~\ref{sec:fill_implement}. For now, let us use $\func{fill}$ as a black-box operation.
