\section{Sampling from the Multivariate Hypergeometric Distribution}
\label{sec:multivariate_hypergeometric_sampling}

Consider the following random experiment. Suppose that we have an urn containing $B \leq n$ marbles (representing vertices), each occupies one of the $r$ possible colors (representing communities) represented by an integer from $[r]$. The number of marbles of each color in the urn is known: there are $C_k$ indistinguishable marbles of color $k \in [r]$, where $C_1 + \cdots + C_r = B$. Consider the process of drawing $\ell \leq B$ marbles from this urn \emph{without replacement}. We would like to sample how many marbles of each color we draw.

More formally, let $\vec{C} = \langle c_1, \ldots, c_r \rangle$, then we would like to (approximately) sample a vector $\mathbf{S}^\vec{C}_\ell$ of $r$ non-negative integers such that
\[\Pr[\mathbf{S}^\vec{C}_\ell = \langle s_1, \ldots, s_r \rangle]
= \frac{{C_1\choose s_1}\cdot{C_2\choose s_2}\cdots{C_r\choose s_r}}{{B \choose C_1+C_2+ \cdots +C_r}}\]

where the distribution is supported by all vectors satisfying $s_k \in \{0, \ldots, C_k\}$ for all $k \in [r]$ and $\sum_{k=1}^{r} s_k = \ell$. This distribution is referred to as the \emph{multivariate hypergeometric distribution}.

The sample $\mathbf{S}^\vec{C}_\ell$ above may be generated easily by simulating the drawing process, but this may take $\Omega(\ell)$ iterations, which have linear dependency in $n$ in the worst case: $\ell = \Theta(B) = \Theta(n)$. Instead, we aim to generate such a sample in $O(r\,\poly(\log n))$ time with high probability. We first make use of the following procedure from \cite{huge}.

\begin{restatable}{lemma}{res:ggn-marble}\label{claim:ggn}
Suppose that there are $T$ marbles of color $1$ and $B-T$ marbles of color $2$ in an urn,
where $B \leq n$ is even. There exists an algorithm that samples $\langle s_1, s_2 \rangle$,
the number of marbles of each color appearing when drawing $B/2$ marbles from the urn without replacement,
in $O(\poly(\log n))$ time and random words.
Specifically, the probability of sampling a specific pair $\langle s_1, s_2 \rangle$ where $s_1 + s_2 = T$
is approximately ${B/2 \choose s_1}{B/2 \choose T-s_1}/{B \choose T}$ with error of at most $n^{-c}$ for any constant $c>0$.
\end{restatable}

In other words, the claim here only applies to the two-color case,
where we sample the number of marbles when drawing exactly half of the marbles from the entire urn ($r=2$ and $\ell = B/2$).
First we generalize this claim to handle any desired number of drawn marbles $\ell$ (while keeping $r=2$).

\begin{restatable}{lemma}{res:new-marble2}\label{thm:sampling_two_colors}
Given $C_1$ marbles of color $1$ and $C_2 = B-C_1$ marbles of color $2$,
there exists an algorithm that samples $\langle s_1, s_2 \rangle$,
the number of marbles of each color appearing when drawing $\ell$ marbles from the urn without replacement,
in $O(\poly(\log B))$ time and random words.
\end{restatable}
\begin{proof}
For the base case where $B=1$, we trivially have $\mathbf{S}^\vec{C}_1=\vec{C}_1$ and $\mathbf{S}^\vec{C}_0=\vec{C}_2$.
Otherwise, for even $B$, we apply the following procedure.
\begin{itemize}
\item If $\ell \leq B/2$, generate $\vec{C}'=\mathbf{S}^\vec{C}_{B/2}$ using Claim~\ref{claim:ggn}.
\begin{itemize}
\item If $\ell = B/2$ then we are done.
\item Else, for $\ell < B/2$ we recursively generate $\mathbf{S}^\vec{C'}_{\ell}$.
\end{itemize}
\item Else, for $\ell > B/2$, we generate $\mathbf{S}^\vec{C'}_{B-\ell}$ as above, then output $\vec{C}-\mathbf{S}^\vec{C'}_{B-\ell}$.
\end{itemize}
On the other hand, for odd $B$, we simply simulate drawing a single random marble
from the urn before applying the above procedure on the remaining $B-1$ marbles in the urn.
That is, this process halves the domain size $B$ in each step, requiring $\log B$ iterations to sample $\mathbf{S}^\vec{C}_\ell$.
\end{proof}

Lastly we generalize to support larger $r$.
\begin{restatable}{theorem}{res:new-marble}
\label{thm:sampling_many_colors}
Given $B$ marbles of $r$ different colors, such that there are $C_i$ marbles of color $i$,
there exists an algorithm that samples $\langle s_1, s_2,\cdots, s_r \rangle$,
the number of marbles of each color appearing when drawing $l$ marbles from the urn without replacement,
in $O(r\cdot\poly(\log B))$ time and random words.
\end{restatable}
\begin{proof}
Observe that we may reduce $r>2$ to the two-color case by sampling the number of marbles of the first color,
collapsing the rest of the colors together.
Namely, define a pair $\vec D=\langle C_1, C_2+\cdots+C_r \rangle$,
then generate $\mathbf{S}^{\vec D}_{\ell}=\langle s_1, s_2+\ldots+s_r\rangle$ via the above procedure.
At this point we have obtained the first entry $s_1$ of the desired $\mathbf{S}^{\vec{C}}_{\ell}$.
So it remains to generate the number of marbles of each color from the remaining $r-1$ colors in $\ell-s_1$ remaining draws.
In total, we may generate $\mathbf{S}^{\vec{C}}_{\ell}$ by performing $r$ iterations of the two-colored case.
The error in the $L_1$-distance may be established similarly to the proof of Lemma~\ref{lemma:transition}.
\end{proof}
\todo{faster sampling for first $k$ colors}
\begin{theorem}
\label{thm:sampling_many_colors_contiguous}
Given $B$ marbles of $r$ different colors in $[r]$, such that there are $C_i$ marbles of color $i$ and a parameter $k\le r$,
there exists an algorithm that samples $s_1 + s_2 +\cdots + s_k$,
the number of marbles among the first $k$ colors appearing when drawing $\ell$ marbles from the urn without replacement,
in $O(\poly(\log B))$ time and random words.
\end{theorem}
\begin{proof}
Since we don't have to find the individual counts, we can be more efficient by grouping half the colors together at each step.
Formally, we define a pair $\vec  D= \langle D_1,D_2 \rangle$ where $D_1=C_1 + C_2 +\cdots + C_{r/2}$ and $D_2=C_{r/2+1}+\cdots+C_{r-1}+C_r$.
We then generate $ \langle D_1', D_2'\rangle = \mathbf S^{\vec D}_\ell$.
\begin{itemize}
    \item If $k < r/2$, we recursively solve the problem with the first $r/2$ colors, $B\gets D_1'$, and the original value of $k$.
    \item If $k > r/2$, we recurse on the last $r/2$ colors, $B$ set to $D_2'$, and $k$ set to $k-r/2$.
    In this case, we add $D_1'$ to the returned value.
    \item Otherwise, $k=r/2$ and we can return $D_1'$.
\end{itemize}
The number of recursive calls is $\mathcal O(\log r) = \mathcal O(\log B)$ (since $r\le B$).
So, the overall runtime is $\mathcal O(\poly(\log B))$.
\end{proof}




\subsection{Data structure}
We now show that Theorem~\ref{thm:sampling_many_colors} may be used in order to create the following data structure.
Recall that $\mathsf{R}$ denote the given distribution over integers $[r]$ (namely, the random distribution of communities for each vertex).
Our data structure generates and maintains random variables $X_1, \ldots, X_n$,
each of which is drawn independently at random from $\mathsf{R}$: $X_i$ denotes the community of vertex $i$.
Then given a pair $(i, j)$, it returns the vector $\vec{C}(i, j) = \langle c_1, \ldots, c_r \rangle$
where $c_k$ counts the number of variables $X_i, \ldots, X_j$ that takes on the value $k$.
Note that we may also find out $X_i$ by querying for $(i, i)$ and take the corresponding index.

We maintain a complete binary tree whose leaves corresponds to indices from $[n]$.
Each node represents a range and stores the vector $\vec{C}$ for the corresponding range.
The root represents the entire range $[n]$, which is then halved in each level.
Initially the root samples $\vec{C}(1, n)$ from the multinomial distribution according to $\mathsf{R}$
(see e.g., Section 3.4.1 of \cite{knuth}).
Then, the children are generated on-the-fly using the lemma above.
Thus, each query can be processed within $O(r\,\poly(\log n))$ time, yielding Theorem~\ref{thm:sbm-data}.
Then, by embedding the information stored by the data structure into the state (as in the proof of Lemma~\ref{lemma:transition}),
we obtain the desired Corollary~\ref{cor:sbm-construct}.

