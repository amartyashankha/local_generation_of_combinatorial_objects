\section{Dyck Path Generator}%
\label{sec:dyck_appendix}

\subsection{Dyck Path Boundaries and Deviations}%
\label{sub:dyck_path_boundaries_and_deviations}
Consider a contiguous \emph{sub-path} of a simple Dyck path of length $2n$ that comprises of $U$ up-steps and $D$ down-steps
with $U + D = 2S$, such that the sum of any initial sub-string is not less than $1-k$ for some valid $U, D, k < n$.
This means that we start our Dyck path at a height of $k-1$,
and we are never allowed to cross below zero (Figure~\ref{fig:complex_dyck}).

\begin{lemma}
\label{lem:}
Both $|S-U|$ and $|S-D|$ are $\mathcal O(\log n\sqrt S)$ with high probability.
\end{lemma}
\begin{proof}

\end{proof}
\begin{lemma}
\label{lem:}
There exists a constant $c$ such that if $k > c\log n\sqrt S$, then the distribution of paths sampled without a boundary
(normal random walks) is statistically $1/n^2$-close in $L_1$ distance to the distribution of Dyck paths.
\end{lemma}
\begin{proof}

\end{proof}


\subsection{Computing Probabilities}%
\label{sub:computing_probabilities}
We start with Stirling's approximation which states that
\[
m! = \sqrt{2\pi m}\left( \frac me\right)^m\left( 1 + \mathcal O\left( \frac 1m\right)\right)\\
\]
We will also use the logarithm approximation when a better approximation is required:
\begin{align}
    \label{eq:log_factorial_approximation}
\log (m!) = m\log m -m + \frac 12 \log(2\pi m) + \frac{1}{12m} - \frac{1}{360m^3} + \frac{1}{1260m^5} - \cdots
\end{align}

Oracle for estimating probabilities:
\begin{lemma}
\label{lem:probability_approximation_oracle}
\todo{Point to section referencing the left right/ total.}

Given a probability expression of the form $p_d = \frac{D_{left}\cdot D_{right}}{D_{total}}$ and a parameter $n$,
there exists a $\poly(\log n)$ time oracle that returns a $\left( 1\pm 1/n^2\right)$
multiplicative approximation to $p_d$.% as long as $p_d > 1/n^2$.
\end{lemma}
\begin{proof}
We first compute a $1/n^2$ \emph{additive} approximation to $\log p_d$.
Note that this is possible because $\log p_d$ can be written as a sum of logarithms of factorials \todo{Not obvious how}.
So, we can use the series expansion from Equation~\ref{eq:log_factorial_approximation} up to $\mathcal O(\log n)$ terms.

Now, we can exponentiate the approximation to obtain
$p_d\cdot e^{\mathcal O(1/n^2)} \approx p_d\left( 1 + \mathcal O(1/n^2)\right)$
\end{proof}


\subsection{Approximating Close-to-Central Binomial Coefficients}%
\label{sub:approximating_close_to_central_binomial_coefficients}
This immediately gives us an asymptotic formula for the central binomial coefficient as:
\[
\binom{n}{n/2} = \sqrt{\frac{2}{\pi n}}2^n\left( 1 + \mathcal O\left( \frac 1n\right)\right)
\]

\todo{Cite Asymptopia}
Now, we consider a ``off-center'' Binomial coefficient $\binom{n}{k}$ where $k = \frac{n+c\sqrt n}{2}$.
We consider the ratio: $R = \binom{n}{k}/\binom{n}{n/2}$:
\begin{align}
R &= \frac{\binom{n}{k}}{\binom{n}{n/2}}
= \frac{(n/2)!(n/2)!}{k!(n-k)!} = \mathlarger\prod\limits_{i=1}^{c\sqrt n/2}\frac{n/2-i+1}{n/2+i}\\
\implies \log R &= \mathlarger\sum\limits_{i=1}^{c\sqrt n/2}\log\left(\frac{n/2-i+1}{n/2+i} \right)\\
&= \mathlarger\sum\limits_{i=1}^{c\sqrt n/2} - \frac{4i}{n} + \mathcal O\left( \frac{i^2}{n^2}\right)
= - \frac{c^2n}{2n} + \mathcal O\left( \frac{(c\sqrt n)^3}{n^2}\right)
= - \frac{c^2}{2} + \mathcal O\left( \frac{c^3}{\sqrt n}\right)\\
\implies \binom{n}{k} &= \binom{n}{n/2} e^{-c^2/2}exp\left( \mathcal O(c^3/\sqrt n)\right)
\end{align}


\subsection{Sampling the Height}%
\label{sub:sampling_the_height}


\todo{fix}
\begin{itemize}
    \item $d < c\cdot \sqrt S \log n$
    \item $k < c\cdot \sqrt S \log n \implies U-D < c\cdot \sqrt S \log n$
    \item $k' < c\cdot \sqrt S \log n$
    \item $S > \log^2 n \implies \sqrt S \log n < S$
\end{itemize}

\begin{lemma}
\label{lem:taylor_bound}
For $x < 1$ and $k\ge 1$,
\[
1-kx < (1-x)^k < 1 - kx + \frac{k(k-1)}{2}x^2.
\]
\end{lemma}


\DLeftBound*
\begin{proof}
This involves some simple manipulations.
\begin{align}
D_{left} &= {{S}\choose{D-d}}-{{S}\choose{D-d-k}}\\
&= {{S}\choose{D-d}}\cdot \left[1-\frac{(D-d)(D-d-1)\cdots(D-d-k+1)}{(S-D-d+k)(S-D-d+k-1)\cdots(S-D-d+1)}\right]\\
&\le {{S}\choose{D-d}}\cdot \left[1-\left(\frac{D-d-k+1}{S-D+d+k}\right)^k\right]\\
&\le {{S}\choose{D-d}}\cdot \left[1-\left(\frac{U+d+k-(U-D+d+k-1)}{U+d+k}\right)^k\right]\\
&\le {{S}\choose{D-d}}\cdot \left[1-\left(\frac{U+d+k-\Bo(\log n\sqrt{S})}{U+d+k}\right)^k\right]\\
&\le \Theta \left( \frac{k\log n}{\sqrt{S}} \right) \cdot{{S}\choose{D-d}}
\end{align}
\end{proof}

\DRightBound*
\begin{proof}
\begin{align}
D_{right} &= {{S}\choose{U-d}}-{{S}\choose{U-d-k'}}\\
&= {{S}\choose{U-d}}\cdot \left[1-\frac{(U-d)(U-d-1)\cdots(U-d-k'+1)}{(S-U-d+k')(S-U-d+k'-1)\cdots(S-U-d+1)}\right]\\
&\le {{S}\choose{U-d}}\cdot \left[1-\left(\frac{U-d-k'+1}{S-U+d+k'}\right)^{k'}\right]\\
&\le {{S}\choose{U-d}}\cdot \left[1-\left(\frac{2D-U-d-k+1}{2U-D+k+d}\right)^{k'}\right]\\
&\le {{S}\choose{U-d}}\cdot \left[1-\left(\frac{U+k+d - (2U-2D+2d+2k-1)}{U+k+d}\right)^{k'}\right]\\
&\le {{S}\choose{U-d}}\cdot \left[1-\left(\frac{U+k+d - \Bo(\log n\sqrt S)}{U+k+d}\right)^{k'}\right]\\
&\le \Theta\left(\frac{k'\log n}{\sqrt{S}}\right)\cdot{{S}\choose{U-d}}
\end{align}
\end{proof}

\begin{restatable}{lemma}{DTotalBasicBound}
\label{lem:DTotalBasicBound}
\todo{change statement}
$D_{tot} \ge {{2S}\choose{2D}}\cdot \left[1-\left(1 - \frac{k'}{2U+1}\right)^k\right]$.
\end{restatable}
\begin{proof}
\begin{align}
D_{tot} &= {{2S}\choose{2D}}-{{2S}\choose{2D-k}}\\
&= {{2S}\choose{2D}}\cdot \left[1-\frac{(2D)(2D-1)\cdots(2D-k+1)}{(2S-2D+k)(2S-2D+k-1)\cdots(2S-2D+1)}\right]\\
&\ge {{2S}\choose{2D}}\cdot \left[1-\left(\frac{2D-k+1}{2S-2D+1}\right)^k\right]\\
&\ge {{2S}\choose{2D}}\cdot \left[1-\left(\frac{2U-(2U-2D+k-1)}{2U+1}\right)^k\right]\\
&\ge {{2S}\choose{2D}}\cdot \left[1-\left(\frac{(2U+1)-k'}{2U+1}\right)^k\right]\\
&\ge {{2S}\choose{2D}}\cdot \left[1-\left(1 - \frac{k'}{2U+1}\right)^k\right]\\
\end{align}
\end{proof}

\todo{Reference previous lemma}

\DTotalFarBoundary*
\begin{proof}
When $kk' > 2U + 1 \implies k > \frac{2U+1}{k'}$,
we will show that the above expression is greater than $\frac 12 \binom{2S}{2D}$.
Defining $\nu = \frac{2U+1}{k'} > 1$, we see that $(1-\frac 1\nu)^k \le (1-\frac 1\nu)^\nu$.
Since this is an increasing function of $\nu$ and since the limit of this function is $\frac 1e$,
we conclude that
\[
1-\left(1 - \frac{k'}{2U+1}\right)^k > \frac 12
\]
\end{proof}

\DTotalNearBoundary*
\begin{proof}
Now we bound the term $1-\left(1-\frac{k'}{2U+1}\right)^k$, given that $kk'\le 2U+1\implies\frac{kk'}{2U+1} \le 1$.
Using Taylor expansion, we see that
\begin{align}
&1 - \left(1 - \frac{k'}{2U+1}\right)^k\\
&\le \frac{kk'}{2U+1} - \frac{k(k-1)}{2}\cdot\frac{k'^2}{(2U+1)^2}\\
&\le \frac{kk'}{2U+1} - \frac{k^2k'^2}{2(2U+1)^2}\\
&\le \frac{kk'}{2U+1} \left(1 - \frac{kk'}{2(2U+1)}\right)\\
&\le \frac{kk'}{2(2U+1)} \le \frac{kk'}{\Theta(S)}\\
\end{align}
\end{proof}

\subsection{First Return Sampling}%
\label{sub:first_return_sampling}

\ReturnDLeftBound*
\begin{proof}
In what follows, we will drop constant factors:
\end{proof}


\ReturnDRightBound*
\begin{proof}

\end{proof}


