\section{Omitted Proofs for the Dyck Path Implementation}%
\label{sec:dyck_appendix}

\begin{theorem}
\label{thm:number_of_dyck_paths}
There are $\frac{1}{n+1}\binom{2n}{n}$ Dyck paths for length $2n$ (construction from \cite{catalan_book}).
\end{theorem}
\begin{proof}[Proof from \cite{catalan_book}]
Consider all possible sequences containing $n+1$ up-steps and $n$ down-steps with the restriction that the first step is an up-step.
We say that two sequences belong to the same \emph{class} if they are cyclic shifts of each other.
Because of the restriction, the total number of sequences is $\binom{2n}{n}$ and each class is of size $n+1$.
Now, within each class, exactly one of the sequences is such that the prefix sums are \emph{strictly greater} than zero.
From such a sequence, we can obtain a Dyck sequence by deleting the first up-step.
Similarly, we can start with a Dyck sequence, add an initial up-step and consider all $n+1$ cyclic shifts to obtain a \emph{class}.
This bijection shows that the number of Dyck paths is $ \frac{1}{n+1} \binom{2n}{n}$.
\end{proof}



\subsection{Approximating Close-to-Central Binomial Coefficients}%
\label{sec:approximating_close_to_central_binomial_coefficients}
We start with Stirling's approximation which states that
\[
m! = \sqrt{2\pi m}\left( \frac me\right)^m\left( 1 + \mathcal O\left( \frac 1m\right)\right)\\
\]
We will also use the logarithm approximation when a better approximation is required:
\begin{align}
    \label{eq:log_factorial_approximation}
\log (m!) = m\log m -m + \frac 12 \log(2\pi m) + \frac{1}{12m} - \frac{1}{360m^3} + \frac{1}{1260m^5} - \cdots
\end{align}

This immediately gives us an asymptotic formula for the central binomial coefficient as:
\begin{lemma}
\label{lem:central_binomial_coefficient}
The central binomial coefficient can be approximated as:
\[
\binom{n}{n/2} = \sqrt{\frac{2}{\pi n}}2^n\left( 1 + \mathcal O\left( \frac 1n\right)\right)
\]
\end{lemma}

Now, we consider a ``off-center'' Binomial coefficient $\binom{n}{k}$ where $k = \frac{n+c\sqrt n}{2}$.
\begin{lemma}
\label{lem:close_to_central_binomial_coefficient}
\[
\binom{n}{k} = \binom{n}{n/2} e^{-c^2/2}exp\left( \mathcal O(c^3/\sqrt n)\right)
\]
\end{lemma}
\begin{proof}[Proof from \cite{asymptopia}]
We consider the ratio: $R = \binom{n}{k}/\binom{n}{n/2}$:
\begin{align}
R &= \frac{\binom{n}{k}}{\binom{n}{n/2}}
= \frac{(n/2)!(n/2)!}{k!(n-k)!} = \mathlarger\prod\limits_{i=1}^{c\sqrt n/2}\frac{n/2-i+1}{n/2+i}\\
\implies \log R &= \mathlarger\sum\limits_{i=1}^{c\sqrt n/2}\log\left(\frac{n/2-i+1}{n/2+i} \right)\\
&= \mathlarger\sum\limits_{i=1}^{c\sqrt n/2} - \frac{4i}{n} + \mathcal O\left( \frac{i^2}{n^2}\right)
= - \frac{c^2n}{2n} + \mathcal O\left( \frac{(c\sqrt n)^3}{n^2}\right)
= - \frac{c^2}{2} + \mathcal O\left( \frac{c^3}{\sqrt n}\right)\\
\implies \binom{n}{k} &= \binom{n}{n/2} e^{-c^2/2}exp\left( \mathcal O(c^3/\sqrt n)\right)
\end{align}
\end{proof}



\subsection{Dyck Path Boundaries and Deviations}%
\label{sec:dyck_path_boundaries_and_deviations}
\begin{lemma}
\label{lem:random_walk_deviation_bound}
Given a random walk of length $2n$ with exactly $n$ up and down steps,
consider a contiguous \emph{sub-path} of length $2B$ that comprises of $U$ up-steps and $D$ down-steps i.e. $U + D = 2B$.
Both $|B-U|$ and $|B-D|$ are $\mathcal O(\sqrt{B\log n})$ with probability at least $1-1/n^4$.
\end{lemma}
\begin{proof}
We consider the random walk as a sequence of unbiased random variables $\{X_i\}_{i=1}^{2n}\in \{0,1\}^{2n}$
with the constraint $\sum\limits_{i=1}^{2n}X_i = n$.
Here, $1$ corresponds to an up-step and $0$ corresponds to a down step.
Because of the constraint, $X_i, X_j$ are negatively correlated for $i \not= j$ which allows us to apply Chernoff bounds.
Now we consider a sub-path of length $2B$ and let $U$ denote the sum of the $X_i$s associated with this subpath.
Using Chernoff bound with $\mathbb E[X] = B$, we get:
\[
\mathbb P\left[ |U-B| < 3\sqrt{B \log n}\right]
= \mathbb P\left[ |U-B| < 3\frac{\sqrt{\log n}}{\sqrt B}B\right] < e^{\frac{9\log n}{3}} \approx \frac{1}{n^3}
\]
Since $U$ and $D$ are symmetric, the same argument applies.
\end{proof}

\begin{corollary}
\label{cor:random_walk_deviation_bound}
With high probability, every contiguous sub-path of length $2B$ (with $U$ up and $D$ down steps such that $U+D=2B$) in the random walk
satisfies the property that $|B-U|$ and $|B-D|$ are upper bounded by $c\sqrt{B\log n}$ w.h.p. $1-1/n^2$ (for some constant $c$).
\end{corollary}
\begin{proof}
We can simply apply Lemma~\ref{lem:random_walk_deviation_bound} and union bound over all $n^2$ possible contiguous sub-paths.
\end{proof}

\DyckPathDeviationBound*
\begin{proof}
As a consequence of Theorem~\ref{thm:number_of_dyck_paths}, we can sample a Dyck path
by first sampling a \emph{balanced} random walk with $n$ up steps and $n$ down steps and adding an initial up step.
We can then find the corresponding Dyck path by taking the unique cyclic shift that satisfies the Dyck constraint (after removing the initial up-step).
Any interval in a cyclic shift is the union of at most two intervals in the original sequence.
This affects the bound only by a constant factor.
So, we can simply use Corollary~\ref{cor:random_walk_deviation_bound} to finish the proof.
Notice that since $|U-D| \le |B-U|+|B-D|$, $|U-D| = \mathcal O(\sqrt{B\log n})$ comes for free.
\end{proof}

\DyckPathIrrelevantBoundary*
\begin{proof}
We use $\mathcal D$ and $\mathcal R$ to denote the set of all valid Dyck paths and all random sequences respectively.
Clearly, $\mathcal D\subseteq \mathcal R$.
Since the random walk/sequence distribution is uniform on $\mathcal R$, and by Corollary~\ref{cor:random_walk_deviation_bound}
we see that at least $1-1/n^2$ fraction of the elements of $\mathcal R$ do not violate the boundary constraint.
Therefore, $|\mathcal D|\ge (1-1/n^2)|\mathcal R|$,
and so the $L_1$ distance between the uniform distributions on $\mathcal D$ and $\mathcal R$ is $\mathcal O(1/n^2)$.
\end{proof}



\subsection{Estimating the Sampling Probabilities}%
\label{sec:computing_probabilities}
\begin{lemma}
\label{lem:probability_approximation_oracle}
Given a Dych sub-path problem within a global Dyck path of size $2n$ and a probability expression of the form
$p_d = \frac{S_{left}\cdot S_{right}}{S_{total}}$, there exists a $\poly(\log n)$ time oracle that returns a
$\left( 1\pm 1/n^2\right)$ multiplicative approximation to $p_d$ if $p_d = \Omega(1/n^2)$ and returns $0$ otherwise.
\end{lemma}
\begin{proof}
We first compute a $1+1/n^3$ multiplicative approximation to $\ln p_d$.
Using $\mathcal O(\log n)$ terms of the series in Equation~\ref{eq:log_factorial_approximation},
it is possible to estimate the logarithm of a factorial up to $1/n^c$ additive error.
So, we can use the series expansion from Equation~\ref{eq:log_factorial_approximation} up to $\mathcal O(\log n)$ terms.
The additive error can also be cast as multiplicative since factorials are large positive integers.

The probability $p_d$ can be written as an arithmetic expression involving sums and products of a constant number of factorial terms.
Given a $1\pm1/n^c$ multiplicative approximation to $l_a = \ln a$ and $l_a = \ln b$, we wish to approximate $\ln(ab)$ and $\ln(a+b)$.
The former is trivial since $\ln(ab) = l_a + l_b$.
For the latter, we assume $a>b$ and use the identity $\ln(a+b) = \ln a + \ln(1+b/a)$ to note that it suffices to approximate $\ln(1+b/a)$.
We define $\widetilde l_a = l_a\cdot(1\pm \mathcal O(1/n^c))$ and $\widetilde l_b = l_b\cdot(1\pm \mathcal O(1/n^c))$.
In case $\widetilde l_b-\widetilde l_a < -c\ln n\implies b/a < 1/n^c$,
we approximate $\ln(a+b)$ by $\ln a$ since $\ln(1+b/a) = \mathcal O(1/n^c)$ in this case.

Otherwise, we notice that $\max(a,b) = \mathcal O\left((n!)^2\right)\implies \max(l_a, l_b) = o(n^3)\implies l_a-l_b = o(n^3)$.
This is true because if we write out the expression for $p_d = S_{left}\cdot S_{right}/S_{total}$ in terms of sums and products of factorials,
the largest possible value will be a product of at most two terms that are present in the numerator/denominator of a binomial term
(see the expression for generalized Catalan numbers in Equation~\ref{eq:catalan_trapezoid}).
Hence, the claim $\max(a,b) = \mathcal O\left((n!)^2\right)$ follows from the fact that
the numerators/denominators of the relevant binomial coefficient in Equation~\ref{eq:catalan_trapezoid} are at most $n!$.
Using this fact, we obtain the following:
\[
1+e^{\widetilde l_b-\widetilde l_a} = 1+\frac ba\cdot \mathlarger e^{\mathcal O\left( \frac{l_b-l_a}{n^c}\right)}
= 1+\frac ba\cdot\left( 1 \pm \mathcal O\left( \frac{1}{n^{c-3}} \right)\right)
= \left( 1+\frac ba\right)\cdot\left( 1 \pm \mathcal O\left( \frac{1}{n^{c-3}} \right)\right)
\]
In other words, the value of $c$ decreases every time we have a sum operation.
Since there are only a constant number of such arithmetic operations in the expression for $p_d$,
we can set $c$ to be a high enough constant when approximating the factorials,
cewhich allows us to obtain the desired $1\pm1/n^3$ multiplicative approximation to $\ln p_d$.
If $\ln p_d < -3\ln n$, we approximate $p_d = 0$.
Otherwise, we can exponentiate the approximation to obtain $\widetilde p_d=p_d\cdot e^{-\mathcal O(\ln n/n^3)} = p_d\left( 1\pm\mathcal O(1/n^2)\right)$.
\end{proof}



\subsection{Omitted Proofs from Section~\ref{sec:implementing_height_queries}: Sampling the Height}%
\label{sec:appendix_implementing_height_queries}
\begin{lemma}
\label{lem:taylor_bound}
For $x < 1$ and $k\ge 1$, we claim that $1-kx < (1-x)^k < 1 - kx + \frac{k(k-1)}{2}x^2$
\end{lemma}


\DLeftBound*
\begin{proof}
This involves some simple manipulations.
\begin{align}
S_{left} &= {{B}\choose{D-d}}-{{B}\choose{D-d-k}}\\
&= {{B}\choose{D-d}}\cdot \left[1-\frac{(D-d)(D-d-1)\cdots(D-d-k+1)}{(B-D-d+k)(B-D-d+k-1)\cdots(B-D-d+1)}\right]\\
&\le {{B}\choose{D-d}}\cdot \left[1-\left(\frac{D-d-k+1}{B-D+d+k}\right)^k\right]\\
&\le {{B}\choose{D-d}}\cdot \left[1-\left(\frac{U+d+k-(U-D+d+k-1)}{U+d+k}\right)^k\right]\\
&\le {{B}\choose{D-d}}\cdot \left[1-\left(\frac{U+d+k-\Bo(\sqrt{B\log n})}{U+d+k}\right)^k\right]\\
&\le \Theta \left( \frac{k\sqrt{\log n}}{\sqrt{B}} \right) \cdot{{B}\choose{D-d}}
\end{align}
\end{proof}

\DRightBound*
\begin{proof}
\begin{align}
S_{right} &= {{B}\choose{U-d}}-{{B}\choose{U-d-k'}}\\
&= {{B}\choose{U-d}}\cdot \left[1-\frac{(U-d)(U-d-1)\cdots(U-d-k'+1)}{(B-U-d+k')(B-U-d+k'-1)\cdots(B-U-d+1)}\right]\\
&\le {{B}\choose{U-d}}\cdot \left[1-\left(\frac{U-d-k'+1}{B-U+d+k'}\right)^{k'}\right]\\
&\le {{B}\choose{U-d}}\cdot \left[1-\left(\frac{2D-U-d-k+1}{2U-D+k+d}\right)^{k'}\right]\\
&\le {{B}\choose{U-d}}\cdot \left[1-\left(\frac{U+k+d - (2U-2D+2d+2k-1)}{U+k+d}\right)^{k'}\right]\\
&\le {{B}\choose{U-d}}\cdot \left[1-\left(\frac{U+k+d - \Bo(\sqrt{B\log n})}{U+k+d}\right)^{k'}\right]\\
&\le \Theta\left(\frac{k'\sqrt{\log n}}{\sqrt{B}}\right)\cdot{{B}\choose{U-d}}
\end{align}
\end{proof}

\begin{restatable}{lemma}{DTotalBasicBound}
\label{lem:DTotalBasicBound}
$S_{tot} \ge {{2B}\choose{2D}}\cdot \left[1-\left(1 - \frac{k'}{2U+1}\right)^k\right]$.
\end{restatable}
\begin{proof}
\begin{align}
S_{tot} &= {{2B}\choose{2D}}-{{2B}\choose{2D-k}}\\
&= {{2B}\choose{2D}}\cdot \left[1-\frac{(2D)(2D-1)\cdots(2D-k+1)}{(2B-2D+k)(2B-2D+k-1)\cdots(2B-2D+1)}\right]\\
&\ge {{2B}\choose{2D}}\cdot \left[1-\left(\frac{2D-k+1}{2B-2D+1}\right)^k\right]\\
&\ge {{2B}\choose{2D}}\cdot \left[1-\left(\frac{2U-(2U-2D+k-1)}{2U+1}\right)^k\right]\\
&\ge {{2B}\choose{2D}}\cdot \left[1-\left(\frac{(2U+1)-k'}{2U+1}\right)^k\right]\\
&\ge {{2B}\choose{2D}}\cdot \left[1-\left(1 - \frac{k'}{2U+1}\right)^k\right]\\
\end{align}
\end{proof}

\DTotalFarBoundary*
\begin{proof}
When $kk' > 2U + 1 \implies k > \frac{2U+1}{k'}$,
we will show that the above expression is greater than $\frac 12 \binom{2B}{2D}$.
Defining $\nu = \frac{2U+1}{k'} > 1$, we see that $(1-\frac 1\nu)^k \le (1-\frac 1\nu)^\nu$.
Since this is an increasing function of $\nu$ and since the limit of this function is $\frac 1e$,
we conclude that
\[
1-\left(1 - \frac{k'}{2U+1}\right)^k > \frac 12
\]
\end{proof}

\DTotalNearBoundary*
\begin{proof}
Now we bound the term $1-\left(1-\frac{k'}{2U+1}\right)^k$, given that $kk'\le 2U+1\implies\frac{kk'}{2U+1} \le 1$.
Using Bernoulli's inequality, we know that $(1 - x)^n \le 1/(1+nx)$ for $x\in [0, 1]$ and $n\in \mathbb N$.
Since the term $k'/(2U+1)$ is positive and $\le 1$ (since $kk' \le 2U+1$), we can apply this as follows:
\begin{align}
&1 - \left(1 - \frac{k'}{2U+1}\right)^k\\
&\ge 1 - \frac{1}{1 + \frac{kk'}{2U+1} } = \frac{\frac{kk'}{2U+1}}{1 + \frac{kk'}{2U+1} } = \frac{kk'}{2U+1}\cdot \frac{1}{1 + \frac{kk'}{2U+1}} \\
&\ge \frac{kk'}{2U+1}\cdot \frac{1}{2} &\left( \textrm{Since }\frac{kk'}{2U+1}\le 1\right) \\
&\ge \frac{kk'}{\Theta(B)}
\end{align}
\end{proof}



\subsection{Omitted Proofs from Section~\ref{sec:supporting_first_return_queries}: Supporting \func{First-Return} Queries}
\label{sec:omitted_supporting_first_return_queries}
%\ReturnProbabilityBoundNotNormalized*
%\begin{proof}
%This follows from the fact that both $r_{left}(d)$ and $r_{right}(d)$ are $\mathcal O(\log^2 n)$.
%\end{proof}

\ReturnDLeftBound*
\begin{proof}
In what follows, we will drop constant factors:
Refer to Figure~\ref{fig:dyck_mandatory_boundary_sampling} for the setup.
The left section of the path reaches one unit above the boundary (the next step would make it touch the boundary).
The number of up-steps on the left side is $d$ and therefore the number of down steps must be $d + k - 2$.
This inclues $d$ down steps to cancel out the upwards movement, and $k-2$ more to get to one unit above the boundary.
The boundary for this section is $k' = k-1$. This gives us:
\begin{align}
S_{left}(d) &= \binom{2d+k-2}{d} - \binom{2d+k-2}{d-1}\\
&= \binom{2d+k-2}{d}\left[ 1-\frac{d}{d+k-1}\right] = \binom{2d+k-2}{d}\frac{k-1}{d+k-1}
\end{align}
Now, letting $z = 2d+k-2$,  we can write $d = \frac{z-(k-2)}{2} = \frac{z-\frac{k-2}{\sqrt z}\sqrt z}{2}$.
Using Lemma~\ref{lem:DyckPathDeviationBound}, we see that $\frac{k-2}{\sqrt z}$ should be $\mathcal O(\sqrt{\log n})$.
If this is not the case, we can simply return $0$ because the probability associated with this value of $d$ is negligible.
Since $z > \log^4 n$, we can apply Lemma~\ref{lem:close_to_central_binomial_coefficient} to get:
\[
S_{left}(d) = \Theta\left( \binom{z}{z/2} e^{\frac{(k-2)^2}{2z}} \frac{k-1}{d+k-1} \right)
= \Theta\left( \frac{2^{2d+k}}{\sqrt d} e^{\frac{(k-2)^2}{2(2d+k-2)}} \frac{k-1}{d+k-1} \right)
\]
\end{proof}


\ReturnDRightBound*
\begin{proof}
The right section of the path starts from the original boundary.
Consequently, the boundary for this section is at $k' = 1$.
The number of up-steps on the right side is $U-d$ and the number of down steps is $D-d-k+1$.
This gives us:
\begin{align}
S_{right}(d) &= \binom{U+D-2d-k+1}{U-d} - \binom{U+D-2d-k+1}{U-d+1}\\
&= \binom{U+D-2d-k+1}{U-d}\left[ 1-\frac{D-d-k-1}{U-d+1}\right]\\
&= \binom{U+D-2d-k+1}{U-d}\frac{U-D+k}{U-d+1}
\end{align}
Now, letting $z = U+D-2d-k+1$,  we can write $U-d = \frac{z+(U-D+k-1)}{2} = \frac{z+\frac{U-D+k-1}{\sqrt z}\sqrt z}{2}$.
Using Lemma~\ref{lem:DyckPathDeviationBound}, we see that $\frac{k-2}{\sqrt z}$ should be $\mathcal O(\sqrt{\log n})$.
If this is not the case, we can simply return $0$ because the probability associated with this value of $d$ is negligible.
Since $z > \log^4 n$, we can apply Lemma~\ref{lem:close_to_central_binomial_coefficient} to get:
\begin{align}
S_{right}(d) &= \Theta\left( \binom{z}{z/2} e^{\frac{(U-D+k-1)^2}{2z}} \frac{U-D+k}{U-d+1} \right)\\
&= \Theta\left( \frac{2^{U+D-2d-k}}{\sqrt{U+D-2d-k}} e^{\frac{(U-D+k-1)^2}{2(U+D-2d-k+1)}} \frac{U-D+k}{U-d+1} \right)
\end{align}
\end{proof}
