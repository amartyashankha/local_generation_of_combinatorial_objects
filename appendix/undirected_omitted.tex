\section{Omitted Details from Section~\ref{sec:undirected}: Undirected Random Graph Implementations}
\label{sec:undirected_omitted}


\subsection{Removing the Perfect-Precision Arithmetic Assumption}
\label{sec:remove-perfect}

In this section we remove the prefect-precision arithmetic assumption. Instead, we only assume that it is possible to compute $\prod_{u=a}^b (1-p_{vu})$ and $\sum_{u=a}^b p_{vu}$ to $N$-bit precision, as well as drawing a random $N$-bit word, using polylogarithmic resources. Here we will focus on proving that the family of the random graph we generate via our procedures is statistically close to that of the desired distribution. The main technicality of this lemma arises from the fact that, not only the generator is randomized, but the agent interacting with the generator may choose his queries arbitrarily (or adversarially): our proof must handle any sequence of random choices the generator makes, and any sequence of queries the agent may make.

Observe that the distribution of the graphs constructed by our generator is governed entirely by the samples $u$ drawn from $\mathsf{F}(v,a,b)$ in Algorithm~\ref{alg:fill}. By our assumption, the CDF of any $\mathsf{F}(v,a,b)$ can be efficiently computed from $\prod_{u=a}^{u'} (1-p_{vu})$, and thus sampling with $\frac{1}{\poly(n)}$ error in the $L_1$-distance requires a random $N$-bit word and a binary-search in $\bo(\log (b-a+1)) = \bo(\log n)$ iterations. Using this crucial fact, we prove our lemma that removes the perfect-precision arithmetic assumption.

%Note that throughout the proof, we refer to the pair $\mathsf{F}(v,a,b), \mathsf{F}'(v,a,b)$ generically 

\begin{restatable}{lemma}{transition}\label{lemma:transition}
If Algorithm~\ref{alg:fill} (the \func{Fill} operation) is repeatedly invoked to construct a graph $G$ by drawing the value $u$ for at most $S$ times in total, each of which comes from some distribution $\mathsf{F}'(v,a,b)$ that is $\epsilon$-close in $L_1$-distance to the correct distribution $\mathsf{F}(v,a,b)$ that perfectly generates the desired distribution $\mathsf{G}$ over all graphs, then the distribution $\mathsf{G}'$ of the generated graph $G$ is $(\epsilon S)$-close to $\mathsf{G}$ in the $L_1$-distance.
\end{restatable}
\begin{proof}
\label{proof:transition}
For simplicity, assume that the algorithm generates the graph to completion according to a sequence of up to $n^2$ distinct blocks $\mathcal{B} = \langle B^{(u_1)}_{v_1}, B^{(u_2)}_{v_2}, \ldots \rangle$, where each $B^{(u_i)}_{v_i}$ specifies the \unfilled~block in which any query instigates a \func{Fill} function call. Define an \emph{internal state} of our generator as the triplet $s = (k, u, \ADJ)$, representing that the algorithm is currently processing the $k^\textrm{th}$ \func{Fill}, in the iteration (the \textbf{repeat} loop of Algorithm~\ref{alg:fill}) with value $u$, and have generated $\ADJ$ so far. Let $t_{\ADJ}$ denote the \emph{terminal state} after processing all queries and having generated the graph $G_\ADJ$ represented by $\ADJ$. We note that $\ADJ$ is used here in the analysis but not explicitly maintained; further, it reflects the changes in every iteration: as $u$ is updated during each iteration of \func{Fill}, the cells $\ADJ[v][u'] = \PHI$ for $u' < u$ (within that block) that has been skipped are also updated to $\ZERO$.

Let $\mathcal{S}$ denote the set of all (internal and terminal) states. For each state $s$, the generator samples $u$ from the corresponding $\mathsf{F}'(v,a,b)$ where $\|\mathsf{F}(v,a,b)-\mathsf{F}'(v,a,b)\|_1 \leq \epsilon = \frac{1}{\poly(n)}$, then moves to a new state according to $u$. In other words, there is an induced pair of collection of distributions over the states: $(\mathcal{T},\mathcal{T}')$ where $\mathcal{T}=\{\mathsf{T}_s\}_{s\in\mathcal{S}}, \mathcal{T}'=\{\mathsf{T}'_s\}_{s\in\mathcal{S}}$, such that $\mathsf{T}_s(s')$ and $\mathsf{T}'_s(s')$ denote the probability that the algorithm advances from $s$ to $s'$ by using a sample from the correct $\mathsf{F}(v,a,b)$ and from the approximated $\mathsf{F}'(v,a,b)$, respectively. Consequently, $\|\mathsf{T}_s-\mathsf{T}'_s\|_1 \leq \epsilon$ for every $s\in\mathcal{S}$.

The generator begins with the initial (internal) state $s_0 = (1, 0, \ADJ_\PHI)$ where all cells of $\ADJ_\PHI$ are $\PHI$'s, goes through at most $S=O(n^3)$ other states (as there are up to $n^2$ values of $k$ and $O(n)$ values of $u$), and reach some terminal state $t_\ADJ$, generating the entire graph in the process. Let $\pi = \langle s^\pi_0 = s_0, s^\pi_1, \ldots, s^\pi_{\ell(\pi)} = t_\ADJ \rangle$ for some $\ADJ$ denote a sequence (``path'') of up to $S+1$ states the algorithm proceeds through, where $\ell(\pi)$ denote the number of transitions it undergoes. For simplicity, let $T_{t_\ADJ}(t_\ADJ)=1$, and $T_{t_\ADJ}(s)=0$ for all state $s \neq t_\ADJ$, so that the terminal state can be repeated and we may assume $\ell(\pi) = S$ for every $\pi$. Then, for the correct transition probabilities described as $\mathcal{T}$, each $\pi$ occurs with probability $q(\pi) = \prod_{i=1}^{S} \mathsf{T}_{s_{i-1}}(s_i)$, and thus $\mathsf{G}(G_\ADJ) = \sum_{\pi:s^\pi_{S} = t_\ADJ} q(\pi)$.

Let $\mathcal{T}^{\min}=\{\mathsf{T}^{\min}_s\}_{s\in\mathcal{S}}$ where $\mathsf{T}^{\min}_s(s') = \min\{\mathsf{T}_s(s'),\mathsf{T}'_s(s')\}$, and note that each $\mathsf{T}^{\min}_s$ is not necessarily a probability distribution. Then, $\sum_{s'} \mathsf{T}^{\min}_s(s') = 1 - \|\mathsf{T}_s-\mathsf{T}'_s\|_1 \geq 1-\epsilon$. Define $q', q^{\min}, \mathsf{G}'(G_\ADJ),\mathsf{G}^{\min}(G_\ADJ)$ analogously, and observe that $q^{\min}(\pi) \leq \min\{q(\pi), q'(\pi)\}$ for every $\pi$, so $\mathsf{G}^{\min}(G_\ADJ) \leq \min\{\mathsf{G}(G_\ADJ),\mathsf{G}'(G_\ADJ)\}$ for every $G_\ADJ$ as well. In other words, $q^{\min}(\pi)$ lower bounds the probability that the algorithm, drawing samples from the correct distributions or the approximated distributions, proceeds through states of $\pi$; consequently, $\mathsf{G}^{\min}(G_\ADJ)$ lower bounds the probability that the algorithm generates the graph $G_\ADJ$.

Next, consider the probability that the algorithm proceeds through the prefix $\pi_i = \langle s^\pi_0, \ldots, s^\pi_{i}\rangle$ of $\pi$. Observe that for $i \geq 1$,
\begin{align*}\sum_{\pi} q^{\min}(\pi_i) &=\sum_{\pi} q^{\min}(\pi_{i-1})\cdot \mathsf{T}^{\min}_{s^\pi_{i-1}}(s^\pi_{i}) 
= \sum_{s,s'} \sum_{\pi:s^\pi_{i-1} = s,s^\pi_{i} = s'} q^{\min}(\pi_{i-1})\cdot \mathsf{T}^{\min}_{s}(s') \\
&= \sum_{s'} \mathsf{T}^{\min}_s(s')\cdot\sum_{s} \sum_{\pi:s^\pi_{i-1} = s} q^{\min}(\pi_{i-1})
\geq (1-\epsilon) \sum_{\pi} q^{\min}(\pi_{i-1}).\end{align*}
Roughly speaking, at least a factor of $1-\epsilon$ of the ``agreement'' between the distributions over states according to $\mathcal{T}$ and $\mathcal{T}'$ is necessarily conserved after a single sampling process. As $\sum_{\pi} q^{\min}(\pi_0)=1$ because the algorithm begins with $s_0 = (1, 0, \ADJ_\PHI)$, by an inductive argument we have $\sum_{\pi} q^{\min}(\pi)=\sum_{\pi} q^{\min}(\pi_S) \geq (1-\epsilon)^S \geq 1-\epsilon S$. Hence, $\sum_{G_\ADJ} \min\{\mathsf{G}(G_\ADJ),\mathsf{G}'(G_\ADJ)\} \geq \sum_{G_\ADJ} \mathsf{G}^{\min}(G_\ADJ) \geq 1-\epsilon S$, implying that $\|\mathsf{G}-\mathsf{G}'\|_1 \leq \epsilon S$, as desired. In particular,  by substituting $\epsilon = \frac{1}{\poly(n)}$ and $S = O(n^3)$, we have shown that Algorithm~\ref{alg:fill} only creates a $\frac{1}{\poly(n)}$ error in the $L_1$-distance. 
\end{proof}

We remark that \func{Random-Neighbor} queries also require that the returned edge is drawn from a distribution that is close to a uniform one, but this requirement applies only \emph{per query} rather then over the entire execution of the generator. Hence, the error due to the selection of a random neighbor may be handled separately from the error for generating the random graph; its guarantee follows straightforwardly from a similar analysis.





\subsection{Bounding Block Sizes}\label{sec:bounding_block_sizes}
\MaxBlockSize*
\begin{proof}
Fix a block $B_v^{(i)}$, and consider the Bernoulli RVs $\left\{ X_{vu}\right\}_{u\in B_v^{(i)}}$.
The expected number of neighbors in this block is
$ \textstyle\mathbb{E} \left[ |\Gamma^{(i)}(v)| \right] =\mathbb{E} \left[ \sum_{u\in B_v^{(i)}} X_{vu} \right] < L+1$.
Via the Chernoff bound,
\[
\mathbb{P} \left[ |\Gamma^{(i)}(v)|> (1+3c\log n)\cdot L \right]
\le e^{-\frac{3c\log n\cdot L}{3}} = n^{-\Theta(c)}
\]
for any constant $c > 0$.
\end{proof}

\EmptyBlock*
\begin{proof}
For $i < |B_v|$, since $ \mathbb{E} \left[ |\Gamma^{(i)}(v)| \right] =\mathbb{E} \left[ \sum_{u\in B_v^{(i)}} X_{vu} \right] > L-1$, we bound the probability that $B_v^{(i)}$ is empty:
\[
\mathbb P[B_v^{(i)}\textrm{ is empty}] = \prod_{u\in B_v^{(i)}} (1-p_{vu}) \leq e^{-\sum_{u\in B_v^{(i)}} p_{vu}} \leq e^{1-L}=c
\]
for any arbitrary small constant $c$ given sufficienty large constant $L$. Let $T_{i}$ be the indicator for the event that $B_v^{(i)}$ is \emph{not} empty, so $\mathbb E 1-c$. By the Chernoff bound, the probability that less than $|B_v|/3$ blocks are non-empty is 
\[
\textstyle
\mathbb P\left[\sum_{i\in[|B_v|]} T_i<\frac{|B_v|}{3}\right]<\mathbb P\left[\sum_{i\in[|B_v|-1]} T_i<\frac{|B_v-1|}{2}\right]\leq e^{-\Theta(|B_v|-1)} = n^{-\Omega(1)}
\] as $|B_v| = \Omega(\log n)$ by assumption.
\end{proof}
