\subsection{First Return Sampling}%
\label{sub:first_return_sampling}

\ReturnDLeftBound*
\begin{proof}
In what follows, we will drop constant factors:
Refer to Figure~\ref{fig:dyck_return_sampling} for the setup.
The left section of the path reaches one unit above the boundary (the next step would make it touch the boundary).
The number of up-steps on the left side is $d$ and therefore the number of down steps must be $d + k - 2$.
This inclues $d$ down steps to cancel out the upwards movement, and $k-2$ more to get to one unit above the boundary.
The boundary for this section is $k' = k-1$. This gives us:
\begin{align}
D_{left}(d) &= \binom{2d+k-2}{d} - \binom{2d+k-2}{d-1}\\
&= \binom{2d+k-2}{d}\left[ 1-\frac{d}{d+k-1}\right] = \binom{2d+k-2}{d}\frac{k-1}{d+k-1}
\end{align}
Now, letting $z = 2d+k-2$,  we can write $d = \frac{z-(k-2)}{2} = \frac{z-\frac{k-2}{\sqrt z}\sqrt z}{2}$.
Using Lemma~\ref{lem:dyck_path_deviation_bound}, we see that $\frac{k-2}{\sqrt z}$ should be $\mathcal O(\log n)$.
If this is not the case, we can simply return $0$ because the probability associated with this value of $d$ is negligible.
Since $z > \log^7 n$, we can apply Lemma~\ref{lem:close_to_central_binomial_coefficient} to get:
\[
D_{left}(d) = \Theta\left( \binom{z}{z/2} e^{\frac{(k-2)^2}{2z}} \frac{k-1}{d+k-1} \right)
= \Theta\left( \frac{2^{2d+k}}{\sqrt d} e^{\frac{(k-2)^2}{2(2d+k-2)}} \frac{k-1}{d+k-1} \right)
\]
\end{proof}


\ReturnDRightBound*
\begin{proof}
The right section of the path starts from the original boundary.
Consequently, the boundary for this section is at $k' = 1$.
The number of up-steps on the right side is $U-d$ and the number of down steps is $D-d-k+1$.
This gives us:
\begin{align}
D_{right}(d) &= \binom{U+D-2d-k+1}{U-d} - \binom{U+D-2d-k+1}{U-d+1}\\
&= \binom{U+D-2d-k+1}{U-d}\left[ 1-\frac{D-d-k-1}{U-d+1}\right]\\
&= \binom{U+D-2d-k+1}{U-d}\frac{U-D+k}{U-d+1}
\end{align}
Now, letting $z = U+D-2d-k+1$,  we can write $U-d = \frac{z+(U-D+k-1)}{2} = \frac{z+\frac{U-D+k-1}{\sqrt z}\sqrt z}{2}$.
Using Lemma~\ref{lem:dyck_path_deviation_bound}, we see that $\frac{k-2}{\sqrt z}$ should be $\mathcal O(\log n)$.
If this is not the case, we can simply return $0$ because the probability associated with this value of $d$ is negligible.
Since $z > \log^7 n$, we can apply Lemma~\ref{lem:close_to_central_binomial_coefficient} to get:
\begin{align}
D_{right}(d) &= \Theta\left( \binom{z}{z/2} e^{\frac{(U-D+k-1)^2}{2z}} \frac{U-D+k}{U-d+1} \right)\\
&= \Theta\left( \frac{2^{U+D-2d-k}}{\sqrt{U+D-2d-k}} e^{\frac{(U-D+k-1)^2}{2(U+D-2d-k+1)}} \frac{U-D+k}{U-d+1} \right)
\end{align}
\end{proof}

