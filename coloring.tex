\section{Random Coloring of a Graph}%
\label{sec:random_coloring_of_a_graph}

We wish to locally sample an uniformly random coloring of a graph.
A $q$-coloring of a graph $G = (V, E)$ is a function $\sigma : V\rightarrow [q]$,
such that for all $(u,v)\in E$, $\sigma_u \not= \sigma_v$.
We will consider only bounded degree graphs, i.e. graphs with max degree $\le \Delta$.
Otherwise, the coloring problem becomes NP-hard\todo{cite}.

Using the technique of path-coupling, Vigoda \todo{cite} showed that for $q > 2\Delta$,
one can sample an uniformly random coloring by using a MCMC algorithm.

The Markov Chain proceeds in $T$ steps. The state of the chain at time $t$ is given by $\vec X^t\in [q]^{|V|}$.
Specifically, the color of vertex $v$ at step $t$ is $\vec X^t_v$.

In each step of the Markov process, a pair $(v, c)\in V\times [q]$ is sampled uniformly at random.
Subsequently, if the recoloring of vertex $v$ with color $c$ does not result in a conflict with $v$'s neighbors,
i.e. $c\not\in \left\{ X^t_u : u\in \Gamma(v)\right\}$, then the vertex is recolored i.e. $X_v^{t+1}\leftarrow c$.

After running the MC for $T = \mathcal{O}(n\log n)$ steps we reach the stationary distribution ($\epsilon$ close),
and the coloring is an uniformly random one.

\textbf{Exact Bound:}
$t_{mix}(\epsilon) \le \left( \frac{q-\Delta}{q-2\Delta}\right)n\left( \log n + \log(1/\epsilon)\right)$
\todo{cite book (Peres, Lyons)}



\subsection{Modified Glauber Dynamics}%
\label{sub:modified_glauber_dynamics}

Now we define a modified Markov Chain that proceeds in epochs.
We denote the initial coloring of the graph by $\vec X^0$ and the state of the coloring after the $k^{th}$ epoch by $\vec X^k$.
In the $k^{th}$ epoch $\mathcal E_k$:
\begin{itemize}
    \item Sample $n = |V|$ colors $ \langle c_1, c_2,\cdots, c_n \rangle$ from $[q]$, where $c_v$ is the proposed color for vertex $v$.
    \item For each vertex $v$, we set $\vec X^k_v$ to $c_v$ if for all neighbors $w$ of $v$, $\vec X^k_w\not=c_v$ and $\vec X^{k-1}_w\not=c_v$.
\end{itemize}
%\begin{itemize}
    %\item Pick a random permutation $\pi^{(i)}$ of the vertices $V$.
    %\item Sample $n = |V|$ colors $ \langle c_1, c_2,\cdots, c_n \rangle$ from $[q]$.
    %\item Perform the standard update using the pairs $\left\langle (\pi^{(i)}_1, c_1), (\pi^{(i)}_2, c_2), \cdots, (\pi^{(i)}_n, c_n)\right\rangle$.
%\end{itemize}

This procedure is a special case of the \emph{Local Glauber Dynamics} presented in \cite{mohsen}.
The goal in \cite{mohsen} is to find a simultaneous update rule that causes few conflicts among neighbors (and converges to the correct distribution).
Notice that we can have adjacent nodes update in the same epoch, and this can also be implemented locally using \todo{what?}.
However for the sake of succinctness and because this would only improve a constant factor in the exponent,
we use their update rule and avoid a tedious path coupling argument.
\todo{Wording: a vertex needs to avoid $\le 2\Delta$ colors in order to be  accepted}

\todo{Cite Path Coupling}
For the path coupling argument, we define the standard pre-metric on the space for all possible colorings (not necessarily valid ones).
Given two colorings $X$ and $Y$, we define $d(X,Y)$ as the number of vertices $v$ such that $X_v\not= Y_v$.

We define a coupling $(X,Y)\rightarrow(X',Y')$ where $X$ and $Y$ differ only at a single vertex $v$ such that $X_v = c_X$ and $Y_v = c_Y$.
Now, we pick a random permutation of the vertices along with uniformly sampled colors:
\[
\left\langle (v_1, c_1), (v_2, c_2), \cdots, (v_n, c_n)\right\rangle
= \left\langle (\pi_1, c_1), (\pi_2, c_2), \cdots, (\pi_n, c_n)\right\rangle
\]
Now, for each $(v_i, c_i)$ in order, we update the coloring of $X$ and $Y$ as follows:
\begin{itemize}
    \item If the current color of $v_i$ as well as $c_i$ is in $\{c_X,c_Y\}$,
    then the $X$ chain picks the color $c_i$ and the $Y$ chain picks the other color.
\end{itemize}

\begin{lemma}
\label{lem:mohsen_single_epoch_distance}
If $q = \alpha\Delta$ and $d(X, Y) = 1$, then $\mathbb E[d(X',Y')] \le 1-\left( 1-\frac1\alpha\right)e^{-6/\alpha} + \frac{1/\alpha}{1-2/\alpha}$
\end{lemma}
\begin{corollary}
\label{cor:single_epoch_distansce}
If $q \ge 9\Delta$ and $d(X, Y) = 1$, then $\mathbb E[d(X',Y')] < \frac1{\sqrt{2}}$
\end{corollary}

\begin{theorem}
\label{thm:modified_mixing_time}
If $q\ge 9\Delta$, then the Markov Chain is mixed after $2\log n$ epochs.
\end{theorem}
\begin{proof}
\todo{Write proof}
\end{proof}




\subsection{Local Coloring Algorithm}%
\label{sub:local_coloring_algortihm}
Given a vertex $v$, the local-access generator has to output the color of $v$ after running $t \le 2\log n$ epochs of \emph{Modified Glauber Dynamics}.
We will define the number of colors as $q = \alpha\Delta$ where $\alpha > 1$.

Each epoch can be implemented a local generator for a permutation of $V$.
\todo{No need for this any more}
%Specifically, each epoch is a sequence of vertex and color samples:
%\[
%\chi^{(i)}
%= \left\langle (v^{(i)}_1, c^{(i)}_1), (v^{(i)}_2, c^{(i)}_2), (v^{(i)}_3, c^{(i)}_3), \cdots, (v^{(i)}_T, c^{(i)}_T)\right\rangle
%\thicksim_{\mathcal U} \left( V\times [q]\right)^T
%\]
%Here, $\left\{ v^{(i)}_1, v^{(i)}_2,\cdots, v^{(i)}_n\right\}$ is a permutation of all the vertices.
%A position $j$ in the sequence is labeled \emph{``ACCEPT''} if at the $j^{th}$ step,
%$v^{(i)}_j$ was recolored to $c^{(i)}_j$ (no conflicts with neighbors).
%Otherwise, position $j$ is marked \emph{``REJECT''}.
%We define $X^{(i)}_v = 1$ if the sample for $v$ in the $i^{th}$ epoch $(v, c^{(i)}_j)$ is accepted and $X^{(i)}_v = 0$ otherwise.

%We also define $\mathcal C^{(i)}_v$ to be the color sampled for vertex $v$ in the $i^{th}$ epoch,
%and $\mathcal I^{(i)}_v$ to be the corresponding index $j$.

Each epoch is a vector of color samples $\vec C^{i} \thicksim_{\mathcal U} [q]^T$.
We also use $\vec X^i$ to denote the final vector of vertex colors at the end of the $i^{th}$ epoch.
Finally, we define indicator variables $\bm \chi^i_v$ to denote if the color for vertex $v$ was accepted at the $i^{th}$ epoch;
$\bm \chi^i_v = 1$ if and only if for all neighbors $w\in \Gamma(v)$,
we satisfy the condition $\vec  C^i_v\not= \vec X^{i-1}_w$ and $\vec C^i_v\not= \vec C^i_w$.
So, the color of a vertex $v$ after the $t^{th}$ epoch $X^t_v$ is set to be $C^i_v$
where $i\le t$ is the largest index such that $\chi^i_v=1$.
Algorithm~\ref{alg:coloring} shows the procedure for querying the value of $X^t_v$.

\begin{algorithm}[H]
\caption{Generator}
\begin{algorithmic}[1]

\Procedure{Color}{$v, t$}
    \For{$i \gets [ t, t-1, t-2 \cdots 1 ]$}
        \If {\func{Accepted}($i, v$)}
            \State \Return $C^i_v$
        \EndIf
    \EndFor
\EndProcedure

\Procedure{Accepted}{$v, t$}
    \State {$c\gets C^t_v$}
    \For{$w \gets \Gamma(v)$}
        \State {$flag\gets \ZERO$}
        \For{$t' \gets [t, t-1, t-2, \cdots, 1]$}
            \If {$t' \not= t$ \textbf{or} $\mathcal I^{(t)}_w < \mathcal I^{(t)}_v$}
                \If {$\mathcal C^{(t')}_w = c$ \textbf{and} \func{Accepted}($w, t'$)}
                    \State $flag\gets \ONE$
                    \While{$t' < t$}
                        \State $t'\gets t' + 1$
                        \If {\func{Accepted}($w, t'$)}
                            \State $flag\gets \ZERO$
                            \State \textbf{break}
                        \EndIf
                    \EndWhile
                    \If {$flag = \ONE$}
                        \State \Return $\ZERO$
                    \EndIf
                    \State \textbf{break}
               \EndIf
           \EndIf
        \EndFor
    \EndFor
    \State \Return $\ONE$
\EndProcedure

\end{algorithmic}
\label{alg:coloring}
\end{algorithm}

\begin{theorem}
\label{thm:coloring_recurrence}
Coloring recurrence.
\end{theorem}
\begin{proof}
Here, $\mathcal B(p)$ refers to the Bernoulli random variable with parameter $p$.
Assuming $t = k\log n$.
\begin{align}
T_{t} &\le \left|\Gamma(v)\right| \cdot \mathlarger\sum\limits_{i=1}^{t} \mathcal B\left( \frac{1}{\alpha\Delta}\right)\cdot
\left[ T_{i} + T_{i+1} + \mathcal B\left(\frac{1}{\alpha}\right)\cdot T_{i+2} + \mathcal B\left(\frac{1}{\alpha^2}\right)\cdot T_{i+3} + \cdots\right]\\
&\le 3\Delta \cdot \mathcal B\left(\frac{1}{\alpha\Delta}\right)\cdot T_{t-1} + T_{t-1}\\
\implies \mathbb E[T_t] &\le \left( 1 + \frac{3}{\alpha}\right)\cdot \mathbb E[T_{t-1}]\\
\implies \mathbb E[T_t] &\le \Delta\cdot \left( 1 + \frac{3}{\alpha}\right)^t = \Delta\cdot \mathlarger n^{k\log{\left(1 + \frac{3}{\alpha} \right)}}
= \Delta\cdot \mathlarger n^{\frac{3k}{\alpha}}
\end{align}
\end{proof}
