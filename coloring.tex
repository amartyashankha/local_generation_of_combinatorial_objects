\section{Random Coloring of a Graph}%
\label{sec:random_coloring_of_a_graph}

\todo{Query access}
We wish to locally sample an uniformly random coloring of a graph.
A $q$-coloring of a graph $G = (V, E)$ is a function $\sigma : V\rightarrow [q]$,
such that for all $(u,v)\in E$, $\sigma_u \not= \sigma_v$.
We will consider only bounded degree graphs, i.e. graphs with max degree $\le \Delta$.
Otherwise, the coloring problem becomes NP-hard\todo{cite}.

Using the technique of path-coupling, Vigoda \todo{cite} showed that for $q > 2\Delta$,
one can sample an uniformly random coloring by using a MCMC algorithm.

The Markov Chain proceeds in $T$ steps. The state of the chain at time $t$ is given by $\vec X^t\in [q]^{|V|}$.
Specifically, the color of vertex $v$ at step $t$ is $\vec X^t_v$.

In each step of the Markov process, a pair $(v, c)\in V\times [q]$ is sampled uniformly at random.
Subsequently, if the recoloring of vertex $v$ with color $c$ does not result in a conflict with $v$'s neighbors,
i.e. $c\not\in \left\{ X^t_u : u\in \Gamma(v)\right\}$, then the vertex is recolored i.e. $X_v^{t+1}\leftarrow c$.

After running the MC for $T = \mathcal{O}(n\log n)$ steps we reach the stationary distribution ($\epsilon$ close),
and the coloring is an uniformly random one.

\textbf{Exact Bound:}
$t_{mix}(\epsilon) \le \left( \frac{q-\Delta}{q-2\Delta}\right)n\left( \log n + \log(1/\epsilon)\right)$
\todo{cite book (Peres, Lyons)}



\subsection{Modified Glauber Dynamics}%
\label{sub:modified_glauber_dynamics}

Now we define a modified Markov Chain that proceeds in epochs.
We denote the initial coloring of the graph by $\vec X^0$ and the state of the coloring after the $k^{th}$ epoch by $\vec X^k$.
In the $k^{th}$ epoch $\mathcal E_k$:
\begin{itemize}
    \item Sample $n = |V|$ colors $ \langle c_1, c_2,\cdots, c_n \rangle$ from $[q]$, where $c_v$ is the proposed color for vertex $v$.
    \item For each vertex $v$, we set $\vec X^k_v$ to $c_v$ if for all neighbors $w$ of $v$, $\vec X^k_w\not=c_v$ and $\vec X^{k-1}_w\not=c_v$.
\end{itemize}
%\begin{itemize}
    %\item Pick a random permutation $\pi^{(i)}$ of the vertices $V$.
    %\item Sample $n = |V|$ colors $ \langle c_1, c_2,\cdots, c_n \rangle$ from $[q]$.
    %\item Perform the standard update using the pairs $\left\langle (\pi^{(i)}_1, c_1), (\pi^{(i)}_2, c_2), \cdots, (\pi^{(i)}_n, c_n)\right\rangle$.
%\end{itemize}

This procedure is a special case of the \emph{Local Glauber Dynamics} presented in \cite{mohsen}.
The goal in \cite{mohsen} is to find a simultaneous update rule that causes few conflicts among neighbors (and converges to the correct distribution).
Notice that we can have adjacent nodes update in the same epoch, and this can also be implemented locally using \todo{what?}.
However for the sake of succinctness and because this would only improve a constant factor in the exponent,
we use their update rule and avoid a tedious path coupling argument.
\todo{Wording: a vertex needs to avoid $\le 2\Delta$ colors in order to be  accepted}

\todo{Cite Path Coupling}
For the path coupling argument, we define the standard pre-metric on the space for all possible colorings (not necessarily valid ones).
Given two colorings $X$ and $Y$, we define $d(X,Y)$ as the number of vertices $v$ such that $X_v\not= Y_v$.

We define a coupling $(X,Y)\rightarrow(X',Y')$ where $X$ and $Y$ differ only at a single vertex $v$ such that $X_v = c_X$ and $Y_v = c_Y$.
Now, we pick a random permutation of the vertices along with uniformly sampled colors:
\[
\left\langle (v_1, c_1), (v_2, c_2), \cdots, (v_n, c_n)\right\rangle
= \left\langle (\pi_1, c_1), (\pi_2, c_2), \cdots, (\pi_n, c_n)\right\rangle
\]
Now, for each $(v_i, c_i)$ in order, we update the coloring of $X$ and $Y$ as follows:
\begin{itemize}
    \item If the current color of $v_i$ as well as $c_i$ is in $\{c_X,c_Y\}$,
    then the $X$ chain picks the color $c_i$ and the $Y$ chain picks the other color.
\end{itemize}

\begin{lemma}
\label{lem:mohsen_single_epoch_distance}
If $q = 2\alpha\Delta$ and $d(X, Y) = 1$, then $\mathbb E[d(X',Y')] \le 1-\left( 1-\frac1{2\alpha}\right)e^{-3/\alpha} + \frac{1/2\alpha}{1-1/\alpha}$
\end{lemma}
\begin{corollary}
\label{cor:single_epoch_distansce}
If $q \ge 9\Delta$ and $d(X, Y) = 1$, then $\mathbb E[d(X',Y')] < \frac1{e^{1/3}}$
\end{corollary}

\begin{theorem}
\label{thm:modified_mixing_time}
If $q\ge 9\Delta$, then the Markov Chain is mixed after $\tau_{mix}(\epsilon) = 3\ln n + 3\ln(1/\epsilon)$ epochs.
\end{theorem}
\begin{proof}
\todo{Write proof}
\end{proof}




\subsection{Local Coloring Algorithm}%
\label{sub:local_coloring_algortihm}
Given a vertex $v$, the local-access generator has to output the color of $v$ after running $t \le 2\ln n$ epochs of \emph{Modified Glauber Dynamics}.
We will define the number of colors as $q = \alpha\Delta$ where $\alpha > 1$.

%Each epoch can be implemented a local generator for a permutation of $V$.
%Specifically, each epoch is a sequence of vertex and color samples:
%\[
%\chi^{(i)}
%= \left\langle (v^{(i)}_1, c^{(i)}_1), (v^{(i)}_2, c^{(i)}_2), (v^{(i)}_3, c^{(i)}_3), \cdots, (v^{(i)}_T, c^{(i)}_T)\right\rangle
%\thicksim_{\mathcal U} \left( V\times [q]\right)^T
%\]
%Here, $\left\{ v^{(i)}_1, v^{(i)}_2,\cdots, v^{(i)}_n\right\}$ is a permutation of all the vertices.
%A position $j$ in the sequence is labeled \emph{``ACCEPT''} if at the $j^{th}$ step,
%$v^{(i)}_j$ was recolored to $c^{(i)}_j$ (no conflicts with neighbors).
%Otherwise, position $j$ is marked \emph{``REJECT''}.
%We define $X^{(i)}_v = 1$ if the sample for $v$ in the $i^{th}$ epoch $(v, c^{(i)}_j)$ is accepted and $X^{(i)}_v = 0$ otherwise.

%We also define $\mathcal C^{(i)}_v$ to be the color sampled for vertex $v$ in the $i^{th}$ epoch,
%and $\mathcal I^{(i)}_v$ to be the corresponding index $j$.

Each epoch is a vector of color samples $\vec C^{i} \thicksim_{\mathcal U} [q]^n$.
Note that these values are fully independent and as such any $\vec C^i_v$ can be sampled trivially.
We also use $\vec X^i$ to denote the final vector of vertex colors at the end of the $i^{th}$ epoch.
Finally, we define indicator variables $\bm \chi^i_v$ to denote if the color for vertex $v$ was accepted at the $i^{th}$ epoch;
$\bm \chi^i_v = 1$ if and only if for all neighbors $w\in \Gamma(v)$,
we satisfy the condition $\vec  C^i_v\not= \vec X^{i-1}_w$ and $\vec C^i_v\not= \vec C^i_w$.
So, the color of a vertex $v$ after the $t^{th}$ epoch $\vec X^t_v$ is set to be $\vec C^i_v$
where $i\le t$ is the largest index such that $\bm \chi^i_v=1$.
Algorithm~\ref{alg:coloring} shows the procedure for querying the value of $\vec X^t_v$.

\begin{algorithm}[H]
\caption{Generator}
\begin{algorithmic}[1]

\Procedure{Color}{$v, t$}
    \For{$i \gets [ t, t-1, t-2 \cdots 1 ]$}
        \If {\func{Accepted}($i, v$)}
            \State \Return $C^i_v$
        \EndIf
    \EndFor
\EndProcedure

\Procedure{Accepted}{$v, t$}
    \State {$c\gets C^t_v$}
    \For{$w \gets \Gamma(v)$}
        \State {$flag\gets \ZERO$}
        \For{$t' \gets [t, t-1, t-2, \cdots, 1]$}
            \If {$t' \not= t$ \textbf{or} $\mathcal I^{(t)}_w < \mathcal I^{(t)}_v$}
                \If {$\mathcal C^{(t')}_w = c$ \textbf{and} \func{Accepted}($w, t'$)}
                    \State $flag\gets \ONE$
                    \While{$t' < t$}
                        \State $t'\gets t' + 1$
                        \If {\func{Accepted}($w, t'$)}
                            \State $flag\gets \ZERO$
                            \State \textbf{break}
                        \EndIf
                    \EndWhile
                    \If {$flag = \ONE$}
                        \State \Return $\ZERO$
                    \EndIf
                    \State \textbf{break}
               \EndIf
           \EndIf
        \EndFor
    \EndFor
    \State \Return $\ONE$
\EndProcedure

\end{algorithmic}
\label{alg:coloring}
\end{algorithm}

When the algorithm is asked for the final color of $v$, it finds the last epoch in which $v$ was accepted.
Since there are only $\mathcal O(\log n)$ epochs, we focus our attention on the subroutine $\func{Accepted}$ that samples the value of $\bm\chi^t_v$.
The algorithm iterates through the neighbors $w$ of $v$, and check for conflicts with the proposed color $c=\vec C^t_v$.
The condition $c\not= \vec C^t_w$ is can easily be checked by sampling $\vec C^t_w$ in the current epoch.
If no conflict is seen, the next step is to check whether $c\not= \vec X^{t-1}_w$.

We first iterate through all the epochs in reverse order to check whether the color $c$ was ever proposed for vertex $w$.
If not, we can ignore $w$, and otherwise let's say that the last proposal for $c$ was at epoch $t'$ i.e. $\vec C^{t'}_w = c$.
Now, we need to recursively check if this proposal was $\func{Accepted}$.
If it was, we move to epoch $t'+1$ to see if $w$'s color was replaced.
If not, we check epoch $t'+2$ and so on until we reach epoch $t-1$.
At this point we have seen that $\bm\chi^{t'}_w = 1$ (color $c$ was accepted) and every subsequent proposal until the current epoch was rejected
i.e. $\vec X^{t-1}_w = c$ and this leads to a conflict with $v$'s current proposal for color $c$ and hence $\bm\chi^t_v = 0$.
If at any of the iterations, we see that a different proposal was accepted, then $w$ does not cause a conflict and we can move on to the next neighbor.
If we exhaust all the neighbors and don't find any conflicts then $\bm\chi^t_v = 1$.

\begin{lemma}
\label{lem:color_reject_probability}
The probability that any given proposal is rejected $\mathbb P[\bm\chi^t_v=0]$ is at most $1/\alpha$.
Moreover, this upper bound holds even if we condition on all the values in $\vec C$ except $\vec C^t_v$.
\end{lemma}
\begin{proof}
A rejection can occur due to a conflict with at most $2\Delta$ possible values in $\{ C^t_w, X^{t-1}_w | w\in\Gamma(v)\}$.
Since there are $2\alpha\Delta$ colors, the rejection probability is at most $1/\alpha$.
\end{proof}

So, the number of probes required to check whether a color $c$ (assigned at epoch $t'$) was overwritten at some epoch before $t$ is:
\begin{align}
\label{eq:color_overwrite}
\Biggl[T_{t'+1} + \mathcal B\left(\frac{1}{\alpha}\right)\cdot T_{t'+2}
+ \mathcal B\left(\frac{1}{\alpha^2}\right)\cdot T_{t'+3} + \cdots
+ \mathcal B\left(\frac{1}{\alpha^{t-t'-2}}\right)\cdot T_{t-1} \Biggr]
\end{align}

\begin{lemma}
\label{lem:coloring_recurrence}
For $\alpha>4$, the expected time to sample a single $\bm\chi^t_v$ is $\mathbb E[T_t] = \mathcal{O}\left(t\Delta e^{2t/\alpha}\right)$.
\end{lemma}
\begin{proof}
We formulate a recurrence for the expected number of probes to $\{\bm\chi^{t'}\}_{t'\in[t]}$ used by the algorithm.
We will use $\mathcal B(p)$ to refer to the Bernoulli random variable with bias $p$.
When checking a single neighbor $w$, the algorithm iterates through all the epochs $t'$ such that $\vec C^{t'}_w = c$
(technically, only the last occurence matters, but we are looking for an upper bound).
If such a $t'$ is found (this happens with probability $1/q$ independently for each trial), there is one recursive call to $T_{t'}$.
Regardless of what happens, let's say the algorithm queries $T_{t'+1}, T_{t'+2}, \cdots, T_{t-1}$ until an $\func{Accepted}$ proposal is found.
Adding an extra $T_{t'}$ term to Equation~\ref{eq:color_overwrite} and summing up over all neighbors and epochs we get the following:
\begin{align}
T_{t} &\le \Delta \cdot \mathlarger\sum\limits_{t'=1}^{t} \mathbb P[C^{t'}_w = c]\cdot
\Biggl[ T_{t'} + T_{t'+1} + \mathcal B\left(\frac{1}{\alpha}\right)\cdot T_{t'+2}
+ \mathcal B\left(\frac{1}{\alpha^2}\right)\cdot T_{t'+3} + \cdots\\
&\hspace{23em}
\cdots + \mathcal B\left(\frac{1}{\alpha^{t-t'-2}}\right)\cdot T_{t-1} \Biggr]\\
&\le \Delta\cdot\mathcal B\left( \frac{1}{q}\right) \Biggl[
\mathlarger\sum\limits_{t'=1}^{t-1} T_{t'} +
\mathlarger\sum\limits_{t'=1}^{t-1} T_{t'}\cdot \left(1 + \mathcal B\left(\frac1\alpha\right) + \mathcal B\left(\frac1{\alpha^2}\right) + \cdots\right)
\Biggr]
\end{align}
In the second step, we just group all the terms from the same epoch together.
Using Lemma~\ref{lem:color_reject_probability} and the fact that $\mathbb P[C^{t'}_w = c]$ is independent of all other events,
we can write a recurrence for the expected number of probes.
\begin{align}
\mathbb E[T_t]\le \Delta\cdot\frac{1}{\alpha\Delta}
\left[
\mathlarger\sum\limits_{t'=1}^{t-1} T_{t'} + \mathlarger\sum\limits_{t'=1}^{t-1} T_{t'}\cdot
\left(1 + \frac1\alpha + \frac1{\alpha^2} + \cdots\right)
\right]
\end{align}
Now, we make the assumption that $\mathbb E[T_{t'}]\le e^{2t/\alpha}$ and show that this satisfies the expectation recurrence.
First, we sum the geometric series:
\[
\mathlarger\sum\limits_{t'=1}^{t-1} \mathbb E[T_{t'}] = \mathlarger\sum\limits_{t'=1}^{t-1} e^{2t'/\alpha}
< \frac{e^{2t/\alpha}-1}{e^{2/\alpha}-1} < \frac{e^{2t/\alpha}}{e^{2/\alpha}-1}
\]
The expectation recurrence to be satisfied then becomes:
\[
\mathbb E[T_t]\le e^{2t/\alpha}\cdot \frac 1{\alpha}\cdot \frac{1}{e^{2/\alpha}-1}\cdot \left[ 1+ \frac{\alpha}{\alpha-1} \right]
= e^{2t/\alpha}\cdot \frac{2\alpha-1}{\alpha(\alpha-1)(e^{2/\alpha}-1)} = e^{2t/\alpha}\cdot f(\alpha)
\]
We notice that $f(\alpha)$ is is upper bounded by $1$ in the domain $\alpha> 4$ (in fact $\lim\limits_{\alpha\to\infty}f(\alpha) = 1$).
This can easily be verified by taking the derivatives and using L'H\^{o}spital's rule.
Thus, our recurrence is satisfied.
Finally, we note that each probe potentially takes time $\mathcal O(t\Delta)$ to iterate through all the neighbors in all epochs
resulting in a total runtime of $\mathcal O(t\Delta e^{2t/\alpha})$.
\end{proof}

\begin{corollary}
\label{cor:coloring_improved_probes}
Instead of looking through all the epochs in order, we can use the coloring generator \todo{where?} to find the locations directly.
\end{corollary}

\begin{theorem}
\label{thm:coloring_generator_main}
Given adjacency list query access to a graph with $n$ nodes, maximum degree $\Delta$, and $q=2\alpha\Delta$ colors,
we can sample the color of any given node in an ($1/n$-approximate) uniformly random coloring of the graph in a consistent manner
using only $\mathcal O(n^{12/\alpha}\Delta\log n)$ time space and random bits.
This is sublinear for $\alpha>12$ and the sampled coloring is $1/n$-close to the uniform distribution in $L_1$ distance.
\end{theorem}
\begin{proof}
We use the mixing time from Theorem~\ref{thm:modified_mixing_time} $\tau_{mix}(1/n) = 6\ln n$ in conjunction with Lemma~\ref{lem:coloring_recurrence}.
So, the overall runtime becomes $\mathcal O(n^{12/\alpha}\Delta\log n)$.
\end{proof}

