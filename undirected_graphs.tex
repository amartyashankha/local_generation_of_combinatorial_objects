\section{Local-Access Implementations for Random Undirected Graphs}
\label{sec:undirected}

In this section, we provide an efficient local access implementations for random undirected graphs
when the probabilities $p_{uv}=\mathbb P[(u,v)\in E]$ are given, and we can efficiently approximate the following quantities:
(1) the probability that there is no edge between a vertex $u$ and a range of consecutive vertices $[a,b]$, namely $\prod_{u=a}^b (1-p_{v,u})$, and
(2) the sum of the edge probabilities (i.e., the expected number of edges) between $u$ and vertices from $[a,b]$, namely $\sum_{u=a}^b p_{v,u}$.
In Section~\ref{sec:applications}, we provide subroutines for computing these values for the Erd\"{o}s-R\'{e}nyi model and the Stochastic Block model.
We also begin by assuming perfect-precision arithmetic, which we relax in Section~\ref{sec:remove-perfect}.

First, we propose a simple implementation of our generator in Section~\ref{sec:ER-naive} that sequentially fills out the adjacency matrix;
while we do not focus on its efficiency, we establish some basic concepts for further analysis in this section.
Next, we improve our subroutine for \func{Next-Neighbor} queries in Section~\ref{sec:ER-rand}:
this algorithm samples for the next candidate of the next neighbor in a more direct manner to speed-up the process.
Extending this construction, we obtain our main algorithm in Section~\ref{sec:buckets} via the bucketing technique:
partition the vertex set into contiguous ranges to normalize the expected number of neighbors in each bucket,
allowing an efficient \func{Random-Neighbor} implementation by picking a random neighbor from a random bucket.
The subroutine that samples for neighbors within a bucket, along with the remaining analysis of the algorithm,
is given later in Section~\ref{sec:fill_implement}.

\iffalse
{\color{blue}
Alternatively, we provide an implementation for these two models with a deterministic performance guarantee in Section~\ref{sec:ER-det}.
In this setting, introducing the \func{Vertex-Pair} queries results in an amortized guarantee on the run-time.
The deterministic guarantee comes at the cost of more complicated data-structures
(we use a two-level nested interval tree and binary search tree).
%We then give two modifications to this implementation; a simplified generator that provides a randomized guarantee in Section~\ref{sec:ER-rand}, and an augmented generator that also implements \func{Vertex-Pair} queries in Section~\ref{sec:ER-pair}.
}
\fi




\subsection{Na\"{i}ve Generator with an Explicit Adjacency Matrix}\label{sec:ER-naive}
\begin{figure}
    \caption{Na\"{i}ve Generator}
    \label{alg:naive}
    \begin{algorithmic}[1]
        \Procedure{Vertex-Pair}{$u,v$}
            \If {$\ADJ[u][v] = \PHI$}
                 \State{\textbf{draw} $X_{u,v}\sim\mathsf{Bern}(p_{u,v})$}
                 \State{$\ADJ[v][u], \ADJ[u][v] \gets X_{u,v}$}
            \EndIf
            \State \Return $\ADJ[u][v]$
        \EndProcedure
        \Procedure{Next-Neighbor}{$v$}
            \For {$u \gets \LAST[v]+1$ to $n$}
                \If {$\func{Vertex-Pair}(v,u) = \ONE$}
                        \State{$\LAST[v] \gets u$}
                        \State \Return $u$
                \EndIf
            \EndFor
            \State{$\LAST[v] \gets n+1$}
            \State \Return $n+1$
        \EndProcedure
        \Procedure{Random-Neighbor}{$v$}
            \State{$R \gets V$}
            \While{$R\not=\emptyset$}
                \State{\textbf{sample} $u \in R$ u.a.r.}
                \If {$\func{Vertex-Pair}(v,u) = \ONE$}
                        \State \Return $u$
                \Else \State{$R \gets R\setminus\{u\}$}
                \EndIf
            \EndWhile
            \State \Return $\bot$
        \EndProcedure
    \end{algorithmic}
\end{figure}

First, consider a naive implemention that fills out the cells of the adjacency matrix $\ADJ$ of $G$ one-by-one as required by each query.
Each entry $\ADJ[u][v]$ can exist in three possible states:
$\ADJ[u][v] = \ONE$ or $\ZERO$ if the generator has determined that $\{u,v\}\in E$ or $\{u,v\} \notin E$ respectively,
and $\ADJ[u][v] = \PHI$ if the marginal probability of $\mathbb P[(u,v)\in E]$ conditioned on all prior samples is $p_{uv}$.
Our generator also maintains the vector $\LAST$ (used in the same sense as \cite{reut}),
where $\LAST[v]$ records the neighbor of $v$ returned in the last call \func{Next-Neighbor}$(v)$, or $\LAST[v]=0$ if no such call has been invoked.
All cells of $\ADJ$ and $\LAST$ are initialized to $\PHI$ and $0$, respectively.

\paragraph*{Characterizing random choices via $X_{u,v}$'s}
Algorithm~\ref{alg:naive} updates the cell $\ADJ[u][v] = \PHI$ to the value of the Bernoulli random variable (RV) $X_{u,v} \sim \mathsf{Bern}(p_{u,v})$
(i.e., flip a coin with bias $p_{u,v}$) only when it needs to decide whether $\{u,v\}\in E$.
%: we say that $X_{u,v}$ is \emph{decided} if $\ADJ[u][v] \neq \PHI$, and call it \emph{undecided} otherwise.
For the sake of analysis, we will frequently consider the \emph{entire} table of RVs $X_{u,v}$ being sampled \emph{up-front} (i.e., flip all coins),
and the algorithm simply ``uncovers'' these variables instead of making coin-flips.
Thus, every cell $\ADJ[u][v]$ is originally $\PHI$, but will eventually take the value $X_{u,v}$ once the graph generation is complete.
An example application of this view of $X_{u,v}$ is the following analysis.

\paragraph*{Sampling from $\Gamma(v)$ uniformly without knowing $\deg(v)$}
Consider a \func{Random-Neighbor}$(v)$ query.
We create a \emph{pool} $R$ of vertices, draw from this pool one-by-one, until we find a neighbor of $u$.
Then, for any fixed table $X_{u,v}$, the probability that a vertex $u \in \Gamma(v)$ is returned is simply the probability that,
in the sequence of vertices drawn from the pool $R$, $u$ appears first among all neighbors in $\Gamma(v)$.
Hence, we sample each $u \in \Gamma(v)$ with probability $1/\deg(v)$, even without \emph{knowing} the specific value of $\deg(v)$.

\paragraph*{Capturing the state of the partially-generated graph with $\ADJ$}
Under the presence of \func{Random-Neighbor} queries, the probability distribution of the random graphs conditioned on the past queries and answers
can be very complex: for instance,
the number of repeated returned neighbors of $v$ reveals information about $\deg(v) = \sum_{u\in V}X_{u,v}$,
which imposes dependencies on as many as $\Theta(n)$ variables.
Our generator, on the other hand, records the neighbors and also \emph{non-neighbors} not revealed by its answers,
yet surprisingly this internal information fully captures the state of the partially-generated graph.
This suggests that we should design generators that maintain $\ADJ$ as done in Algorithm~\ref{alg:naive},
but in a more implicit and efficient fashion in order to achieve the desired complexities.
Another benefit of this approach is that any analysis can be performed on the simple representation $\ADJ$
rather than any complicated data structure we may employ.

\paragraph*{Obstacles for maintaining $\ADJ$}
There are two problems in the current approach.
Firstly, the algorithm only finds a neighbor, for a \func{Random-Neighbor} or \func{Next-Neighbor} query, with probability $p_{u,v}$,
which requires too many iterations: for $G(n,p)$ this requires $1/p$ iterations, which is already infeasible for $p = o(1/\poly(\log n))$.
Secondly, the algorithm may generate a large number of non-neighbors in the process, possibly in random or arbitrary locations.

\iffalse
\paragraph*{The table of Bernoulli random variables $X_{u,v}$'s} To process a \func{Vertex-Pair}$(u,v)$ query, we first check if $\ADJ[u][v]=\PHI$; if so, draw $X_{u,v}$ from a Bernoulli trial $\mathsf{Bern}(p_{u,v})$ (i.e., flip a coin with bias $p_{u,v}$) to choose whether $\{u,v\} \in E$, and update the cells corresponding to this edge accordingly. Then we simply answer with $\ADJ[u][v]$. That is, each $\ADJ[u][v]$ is originally $\PHI$, but eventually takes the value of the random variable (RV) $X_{u,v}$ (which is the same RV as $X_{v,u}$, only written differently) once the graph generation process is complete. For the sake of analysis, we will frequently consider the \emph{entire} table of Bernoulli RVs $X_{u,v}$ being sampled \emph{up-front}, and the algorithm inspects these variables instead of making coin-flips.

More concretely, consider a \func{Random-Neighbor}$(v)$ query, where the subroutine samples a uniform random neighbor of $v$ \emph{without} knowing $\deg(v)$. We create a pool $R$ of vertices, draw from this pool one-by-one, until we find a neighbor of $u$. Then, for any fixed table $X_{u,v}$, the probability that a vertex $u \in \Gamma(v)$ is returned is simply the probability that, in the sequence of vertices drawn from the pool $R$, $u$ appears first among all neighbors in $\Gamma(v)$. Hence, $u$ is clearly sampled with probability $1/\deg(v)$, as desired.

As for \func{Next-Neighbor}$(v)$, we straightforwardly consider each potential neighbor $u > \LAST[v]$: if $\{v,u\}\in E$, we update $\LAST[v]\leftarrow u$, return $u$ and terminate. We return $n+1$ if no such neighbor $u$ exists.
\fi


%Algorithm~\ref{alg:naive} clearly generates a graph drawn from the correct distribution $X_{u,v}\sim\mathsf{Bern}(p_{u,v})$ if the coin-flips are performed with no error. Otherwise, using random $N$-bit words, we may sample $X_{u,v}$ from a distribution that is $\frac{1}{\poly(n)}$-close to $\mathsf{Bern}(p_{u,v})$. Since all $O(n^2)$ RVs $X_{u,v}$'s are drawn independently, the distribution from which $G$ is generated is represented via the product distribution for these variables. Thus its $L_1$-distance to the desired distribution is increased by at most a factor of $O(n^2)$, which is still $\frac{1}{\poly(n)}$. %Unfortunately, Algorithm~\ref{alg:naive} may take up to $\Theta(n)$ iterations to answer a single \func{Next-Neighbor} query.




\subsection{Improved \func{Next-Neighbor} Queries via Run-of-$\ZERO$'s Sampling}\label{sec:ER-rand}

We now speed-up our \func{Next-Neighbor}$(v)$ procedure by attempting to sample for the first index $u > \LAST[v]$ of $X_{v,u}=\ONE$, from a sequence of Bernoulli RVs $\{X_{v, u}\}_{u > \LAST[v]}$, in a direct fashion. To do so, we sample a consecutive ``run'' of $\ZERO$'s with probability $\prod_{u=\LAST[v]+1}^{u'} (1-p_{v,u})$: this is the probability that there is no edge between a vertex $v$ and any $u \in (\LAST[v],u']$, which can be computed efficiently by our assumption. The problem is that, some entries $\ADJ[v][u]$'s in this run may have already been determined (to be $\ONE$ or $\ZERO$) by queries $\func{Next-Neighbor}(u)$ for $u > \LAST[v]$. To this end, we give a succinct data structure that determines the value of $\ADJ[v][u]$ for $u >\LAST[v]$ and, more generally, captures the state $\ADJ$, in Section~\ref{sec:nn-ds}. Using this data structure, we ensure that our sampled run does not skip over any $\ONE$. Next, for the sampled index $u$ of the first occurrence of $\ONE$, we check against this data structure to see if $\ADJ[v][u]$ is already assigned to $\ZERO$, in which case we re-sample for a new candidate $u' > u$. Section~\ref{sec:nn-correctness} discusses the subtlety of this issue.

We note that we do not yet try to handle other types of queries here yet. We also do not formally bound the number of re-sampling iterations of this approach here, because the argument is not needed by our final algorithm. Yet, we remark that $O(\log n)$ iterations suffice with high probability, even if the queries are adversarial. This method can be extended to support \func{Vertex-Pair} queries (but unfortunately not \func{Random-Neighbor} queries). See Section~\ref{sec:reroll-cont} for full details.

\subsubsection{Data structure}\label{sec:nn-ds}

From the definition of $X_{u,v}$, \func{Next-Neighbor}$(v)$ is given by $\min\{u > \LAST[v]: X_{v,u} = \ONE\}$ (or $n+1$ if no satisfying $u$ exists). Let
%\begin{align*}
%P_v &= \{u \in (\LAST[v], n]: \ADJ[v][u]=1\},\\
%w_v &= \min P_v \textrm{, or $n+1$ if $P_v = \emptyset$}. 
%\end{align*}
$P_v = \{u: \ADJ[v][u]=1\}$ be the set of known neighbors of $v$, and $w_v = \min \{(P_v \cap (\LAST[v], n]) \cup \{n+1\}\}$ be its first known neighbor not yet reported by a $\func{Next-Neighbor}(v)$ query, or equivalently, the next occurrence of $\ONE$ in $v$'s row on $\ADJ$ after $\LAST[v]$. Note that $w_v = n+1$ denotes that there is no known neighbor of $v$ after $\LAST[v]$.
Consequently, $\ADJ[v][u] \in \{\PHI,\ZERO\}$ for all $u \in (\LAST[v], w_v)$, so \func{Next-Neighbor}$(v)$ is
either the index $u$ of the first occurrence of $X_{v,u} = \ONE$ in this range, or $w_v$ if no such index exists.

We keep track of $\LAST[v]$ in a dictionary, where the key-value pair $(v,\LAST[v])$ is stored only when $\LAST[v] \neq 0$: this removes any initialization overhead.
Each $P_v$ is maintained as an ordered set, which is also only instantiated when it becomes non-empty.
We maintain $P_v$ simply by adding $u$ to $v$ if a call \func{Next-Neighbor}$(v)$ returns $u$, and vice versa. Clearly, $\ADJ[v][u]=1$ if and only if $u \in P_v$ by construction.

As discussed in the previous section, we cannot maintain $\ADJ$ explicitly, as updating it requires replacing up to $\Theta(n)$ $\PHI$'s to $\ZERO$'s for a single \func{Next-Neighbor} query in the worst case. Instead, we argue that $\LAST$ and $P_v$'s provide a succinct representation of $\ADJ$ via the following observation. For simplicity, we say that $X_{u,v}$ is \emph{decided} if $\ADJ[u][v] \neq \PHI$, and call it \emph{undecided} otherwise.
%For $\ADJ[v][u] \neq \ONE$, we observe that our algorithm upholds the following condition: $\ADJ[v][u] = \ZERO$ if $u < \LAST[v]$ or $v < \LAST[u]$; otherwise $\ADJ[v][u]=\PHI$. To see why this is true, $\ADJ[v][u]$ changes its value from $\PHI$ to $\ZERO$ only when a new neighbor $u' > u$ has been found, and only in this occasion that 

\iffalse
\begin{restatable}{lemma}{cond}\label{lem:cond-0}
Assuming that only \func{Next-Neighbor} queries are allowed, consider $u > \LAST[v]$. Then, $\ADJ[v][u] = \ZERO$ if and only if $\LAST[u] > v$ and $u \notin P_v$.
\end{restatable}
\begin{proof}
\label{proof:cond-0}
If $\ADJ[v][u] = \ZERO$ while $\LAST[v] < u$, then the coin-flip for $X_{v,u}$ must have been performed during some call \func{Next-Neighbor}$(u)$ returning a value $v' > v$: $\LAST[u]$ is increased to $v' > v$ right before the return. Further, $v$ could not have been returned from \func{Next-Neighbor}$(u)$ at any time because $X_{v,u}=\ZERO$, and thus $u$ is never added to $P_v$.

Similarly, if $\LAST[u] > v$ then $X_{v,u}$ is already determined. Further, suppose to the contrary that $X_{v,u} = \ONE$, then $v$ must have been returned in a \func{Next-Neighbor}$(u)$ call. As we still have $u > \LAST[v]$ and $\LAST[v]$ can never decrease, $u$ must have been added to $P_v$ during that call and not removed yet, contradicting $u \notin P_v$.
\end{proof}
Combined with the fact that all entries $\ADJ[v][u]$ for $u < \LAST[v]$ has been revealed (by as a result of repeatedly invoking \func{Next-Neighbor}$(v)$), we conclude that $\LAST$ and $P_v$'s fully represent $\ADJ$.
\fi
\begin{restatable}{lemma}{cond}\label{lem:cond-0}
The data structures $\LAST$ and $P_v$'s together provide a succinct representation of $\ADJ$ when only \func{Next-Neighbor} queries are allowed. In particular, $\ADJ[v][u]=\ONE$ if and only if $u \in P_v$. Otherwise, $\ADJ[v][u]=\ZERO$ when $u <\LAST[v]$ or $v < \LAST[u]$. In all remaining cases, $\ADJ[v][u]=\PHI$.
\end{restatable}
\begin{proof}
The condition for $\ADJ[v][u]=\ONE$ clearly holds by constuction. Otherwise, observe that $\ADJ[v][u]$ becomes decided (that is, its value is changed from $\PHI$ to $\ZERO$) precisely during the first call of \func{Next-Neighbor}$(v)$ that returns a value $u' > u$ which thereby sets $\LAST[v]$ to $u'$ yielding $u<\LAST[v]$, or vice versa.
\end{proof}

\subsubsection{Queries and Updates}\label{sec:nn-correctness}
\begin{wrapfigure}[15]{R}{0.5\textwidth}
\vspace{-2em}
\begin{framed}
    \renewcommand\figurename{Algorithm}
    \caption{Sampling \func{Next-Neighbor}}
    \label{alg:oblivious-coin-toss}
    \begin{algorithmic}[1]
        \Procedure{Next-Neighbor}{$v$}
            \State{$u \gets \LAST[v]$}
            \State{$w_v \gets \min \{(P_v \cap (u, n]) \cup \{n+1\}\}$}
            \While{$u = w_v$ or $\LAST[u] < v$}
                \State{\textbf{sample} $F\sim\mathsf{F}(v,u,w_v)$}
                \State{$u \gets F$}
            \EndWhile
            \If{$u \neq w_v$}
                \State{$P_v \gets P_v \cap \{u\}$}
                \State{$P_u \gets P_u \cap \{v\}$}
            \EndIf
            \State{$\LAST[v] \gets u$}
            \State \Return $u$
        \EndProcedure
    \end{algorithmic}
\end{framed}
\end{wrapfigure}

We now provide our generator (Algorithm~\ref{alg:oblivious-coin-toss}), and discuss the correctness of its sampling process. The argument here is rather subtle and relies on viewing the random process as an ``uncovering'' process on the table of RVs $X_{u,v}$'s as previously introduced in Section~\ref{sec:ER-naive}.
Algorithm~\ref{alg:oblivious-coin-toss}, considers the following experiment for sampling the next neighbor of $v$ in the range $(\LAST[v],w_v)$.
Suppose that we generate a sequence of $w_v-\LAST[v]-1$ independent coin-tosses,
where the $i^\textrm{th}$ coin $C_{v,u}$ corresponding to $u = \LAST[v]+i$ has bias $p_{v,u}$,
regardless of whether $X_{v,u}$'s are decided or not.
Then, we use the sequence $\langle C_{v,u} \rangle$ to assign values to \emph{undecided} random variable $X_{v,u}$.
The crucial observation here is that, the \emph{decided} random variables $X_{v,u} = \ZERO$ do not need coin-flips, and the corresponding coin result $C_{v,u}$ can simply be discarded.
Thus, we need to generate coin-flips up until we encounter some $u$ satisfying
both (i) $C_{v,u} = \ONE$, and (ii) $\ADJ[v][u] = \PHI$.

%\anak{Due to Shankha's change, $\mathsf{F}$ here and $\mathsf{ExactF}$ in the deterministic construction have rather different meaning; consider using a different name altogether in the latter.}

Let $\mathsf{F}(v,a,b)$ denote the probability distribution of the occurrence $u$ of the first coin-flip $C_{v,u} = \ONE$ among the neighbors in $(a, b)$.
More specifically, $F\sim\mathsf{F}(v,a,b)$ represents the event that $C_{v,a+1}=\cdots=C_{v,F-1}=\ZERO$ and $C_{v,F}=\ONE$,
which happens with probability $\mathbb P[F=f]=\prod_{u=a+1}^{f-1} (1-p_{v,u}) \cdot p_{v,f}$.
For convenience, let $F = b$ denote the event where all $C_{v,u}=\ZERO$. Our algorithm samples $F_1\sim\mathsf{F}(v,\LAST[v],w_v)$ to find the first occurrence of $C_{v,F_1} = \ONE$, then samples $F_2\sim\mathsf{F}(v,F_1,w_v)$ to find the second occurrence $C_{v,F_2} = \ONE$, and so on.
These values $\{F_i\}$ are iterated as $u$ in Algorithm~\ref{alg:oblivious-coin-toss}.
As this process generates $u$ satisfying (i) in the increasing order, we repeat until we find one that also satisfies (ii).
Note that once the process terminates at some $u$, we make no implications on the results of any uninspected coin-flips after $C_{v,u}$.

\paragraph*{Obstacles for extending beyond \func{Next-Neighbor} queries}
There are two main issues that prevent this method from supporting \func{Random-Neighbor} queries. Firstly, while one might consider applying \func{Next-Neighbor} from some random location $u$ to find the minimum $u' \geq u$ where $\ADJ[v][u']=\ONE$, the probability of choosing $u'$ will depend on the probabilities $p_{v,u}$'s, and is generally not uniform. While a rejection sampling method may be applied to balance out the probabilities of choosing neighbors, these arbitrary $p_{v,u}$'s may distribute the neighbors rather unevenly: some small contiguous locations may contain so many neighbors that the rejection sampling approach requires too many iterations to obtain a single uniform neighbor.

 %{\color{red}
 Secondly, in developing Algorithm~\ref{alg:oblivious-coin-toss}, we observe that $\LAST[v]$ and $P_v$ together provide a succinct representation of $\ADJ[v][u] = \ZERO$ only for contiguous cells $\ADJ[v][u]$ where $u \leq \LAST[v]$ or $v \leq \LAST[u]$: they cannot handle $\ZERO$ anywhere else. Unfortunately, in order to extend our construction to support \func{Random-Neighbor} queries using the idea suggested in Algorithm~\ref{alg:naive}, we must unavoidably assign $\ADJ[v][u]$ to $\ZERO$ in random locations beyond $\LAST[v]$ or $\LAST[u]$, which cannot be captured by the current data structure. Furthermore, unlike $\ONE$'s, we cannot record $\ZERO$'s using a data structure similarly to that of $P_v$. More specifically, to speed-up the sampling process for small $p_{v,u}$'s, we must generate many random non-neighbors at once as suggested in Algorithm~\ref{alg:oblivious-coin-toss}, but we cannot afford to spend time linear in the number of created $\ZERO$'s to update our data structure. We remedy these issues via the following bucketing approach.
 %}

\iffalse
\subsubsection{Correctness}

The CDF of the distribution $\mathsf{F}(v,a,b)$ is given by $\mathbb P[F \leq f] = 1-\prod_{i=a+1}^{f} (1-p_{v,i})$ for $F<b$
(i.e., $C_{v,u} = \ONE$ for some $u \in (a,F]$).
Since $b=w_v$ is already a neighbor of $v$, $\mathbb P[F \leq b] = 1$.
By our assumption, this CDF can be efficiently computed, and thus sampling with $\frac{1}{\poly(n)}$ error in the $L_1$-distance requires a random $N$-bit word and a binary-search in $O(\log (b-a+1)) = O(\log n)$ iterations.
However, unlike the case of Algorithm~\ref{alg:naive}, a single sample of our Algorithm~\ref{alg:oblivious-coin-toss} determines the values of multiple $X_{v,u}$'s, and we cannot represent the distribution of our generated graph simply via a product distributions of $X_{v,u}$'s.
Instead, we prove a more general statement given below.
Note that $\mathsf{F}, \mathsf{F}'$ need not be the same pair of distributions for every sample.
\begin{restatable}{lemma}{transition}\label{lemma:transition}
If Algorithm~\ref{alg:oblivious-coin-toss} constructs a graph $G$ by making at most $S$ random choices (samples), each of which comes from some distribution $\mathsf{F}'(v,a,b)$ that is $\epsilon$-close in $L_1$-distance to the correct distribution $\mathsf{F}(v,a,b)$ that perfectly generates the desired distribution $\mathsf{G}$ over all graphs, then the distribution $\mathsf{G}'$ of the generated graph $G$ is $(\epsilon S)$-close to $\mathsf{G}$ in the $L_1$-distance.
\end{restatable}
\begin{proof}
\label{proof:transition}
For simplicity, assume that the algorithm generates the graph to completion according to a sequence of $n^2$ \func{Next-Neighbor} queries $Q = \langle q_1, q_2, \ldots q_{n^2}\rangle$, where each $q_i \in V$ specifies the vertex in which the query \func{Next-Neighbor} is invoked, and there are $n$ occurrences of each vertex $Q$. Define an \emph{internal state} of our generator as the triplet $s = (k, u, \ADJ)$, representing that the algorithm is currently processing the $k^\textrm{th}$ \func{Next-Neighbor} query in the iteration (the \textbf{repeat} loop) with value $u$, and have generated $\ADJ$ so far. Let $t_{\ADJ}$ denote the \emph{terminal state} after processing all queries and having generated the graph $G_\ADJ$ represented by $\ADJ$. We note that $\ADJ$ is used here in the analysis but not explicitly maintained; further, it reflects the changes in every iteration: as $u$ is updated during each iteration of \func{Next-Neighbor}, the cells $\ADJ[v][u'] = \PHI$ for $u' \in (\LAST[v],u)$ are also updated to $\ZERO$.

Let $\mathcal{S}$ denote the set of all (internal and terminal) states. For each state $s$, the generator samples $F$ from the corresponding $\mathsf{F}'(v,a,b)$ where $\|\mathsf{F}(v,a,b)-\mathsf{F}'(v,a,b)\|_1 \leq \epsilon = \frac{1}{\poly(n)}$, then moves to a new state according to $F$. In other words, there is an induced pair of collection of distributions over the states: $(\mathcal{T},\mathcal{T}')$ where $\mathcal{T}=\{\mathsf{T}_s\}_{s\in\mathcal{S}}, \mathcal{T}'=\{\mathsf{T}'_s\}_{s\in\mathcal{S}}$, such that $\mathsf{T}_s(s')$ and $\mathsf{T}'_s(s')$ denote the probability that the algorithm advances from $s$ to $s'$ by using a sample from the correct $\mathsf{F}(v,a,b)$ and from the approximated $\mathsf{F}'(v,a,b)$, respectively. Consequently, $\|\mathsf{T}_s-\mathsf{T}'_s\|_1 \leq \epsilon$ for every $s\in\mathcal{S}$.

The generator begins with the initial (internal) state $s_0 = (1, 0, \ADJ_\PHI)$ where all cells of $\ADJ_\PHI$ are $\PHI$'s, goes through at most $S=O(n^3)$ other states as there are up to $n^2$ values of $k$ and $O(n)$ values of $u$, and reach some terminal state $t_\ADJ$. Let $P = \langle s^P_0 = s_0, s^P_1, \ldots, s^P_{\ell(P)} = t_\ADJ \rangle$ for some $\ADJ$ denote a sequence of up to $S+1$ states the algorithm proceeds through. For simplicity, let $T_{t_\ADJ}(t_\ADJ)=1$ and $T_{t_\ADJ}(s)=0$ for all state $s \neq t_\ADJ$, so that the terminal state can be repeated, and we may assume $\ell(P) = S$ for every $P$. Then, for the correct transition probabilities described as $\mathcal{T}$, each $P$ occurs with probability $q(P) = \prod_{i=1}^{S} \mathsf{T}_{s_{i-1}}(s_i)$, and thus $\mathsf{G}(G_\ADJ) = \sum_{P:s^P_{S} = t_\ADJ} q(P)$.

Let $\mathcal{T}^{\min}=\{\mathsf{T}^{\min}_s\}_{s\in\mathcal{S}}$ where $\mathsf{T}^{\min}_s(s') = \min\{\mathsf{T}_s(s'),\mathsf{T}'_s(s')\}$, and note that each $\mathsf{T}^{\min}_s$ is not necessarily a probability distribution. Then, $\sum_{s'} \mathsf{T}^{\min}_s(s') = 1 - \|\mathsf{T}_s-\mathsf{T}'_s\|_1 \geq 1-\epsilon$. Define $q', q^{\min}, \mathsf{G}'(G_\ADJ),\mathsf{G}^{\min}(G_\ADJ)$ analogously, and observe that $q^{\min}(P) \leq \min\{q(P), q'(p)\}$ for every $P$, so $\mathsf{G}^{\min}(G_\ADJ) \leq \min\{\mathsf{G}(G_\ADJ),\mathsf{G}'(G_\ADJ)\}$ for every $G_\ADJ$ as well. In other words, $q^{\min}(P)$ lower bounds the probability that the algorithm, drawing samples from the correct distributions or the approximated distributions, proceeds through states of $P$; consequently, $\mathsf{G}^{\min}(G_\ADJ)$ lower bounds the probability that the algorithm generates the graph $G_\ADJ$.

Next, consider the probability that the algorithm proceeds through the prefix $P_i = \langle s^P_0, \ldots, s^P_{l}\rangle$ of $P$. Observe that for $i \geq 1$,
\begin{align*}\sum_{P} q^{\min}(P_i) &=\sum_{P} q^{\min}(P_{l-1})\cdot \mathsf{T}^{\min}_{s^P_{i-1}}(s^P_{i}) 
= \sum_{s,s'} \sum_{P:s^P_{i-1} = s,s^P_{i} = s'} q^{\min}(P_{i-1})\cdot \mathsf{T}^{\min}_{s}(s') \\
&= \sum_{s'} \mathsf{T}^{\min}_s(s')\cdot\sum_{s} \sum_{P:s^P_{i-1} = s} q^{\min}(P_{i-1})
\geq (1-\epsilon) \sum_{P} q^{\min}(P_{i-1}).\end{align*}
Roughly speaking, at least a factor of $1-\epsilon$ of the ''agreement'' between the distributions over states according to $\mathcal{T}$ and $\mathcal{T}'$ is necessarily conserved after a single sampling process. As $\sum_{P} q^{\min}(P_0)=1$ because the algorithm begins with $s_0 = (1, 0, \ADJ_\PHI)$, by an inductive argument we have $\sum_{P} q^{\min}(P)=\sum_{P} q^{\min}(P_S) \geq (1-\epsilon)^S \geq 1-\epsilon S$. So, $\sum_{G_\ADJ} \min\{\mathsf{G}(G_\ADJ),\mathsf{G}'(G_\ADJ)\} \geq \sum_{G_\ADJ} \mathsf{G}^{\min}(G_\ADJ) \geq 1-\epsilon S$, implying that $\|\mathsf{G}-\mathsf{G}'\|_1 \leq \epsilon S$. In particular,  by substituting $\epsilon = \frac{1}{\poly(n)}$ and $S = O(n^3)$, we have shown that Algorithm~\ref{alg:oblivious-coin-toss} only creates a $\frac{1}{\poly(n)}$ error in the $L_1$-distance. 
\end{proof}


\anak{Ronitt says the proof above is known (e.g., from pseudorandomness) and will help find a reference.}
\iffalse
We defer its proof in Section~\ref{proof:transition}, and provide a high-level proof sketch here.
To prove Lemma~\ref{lemma:transition}, we consider an arbitrary sequence of up to $n^2$ queries  $Q = \langle q_i \rangle$, where $q_i \in V$ specifies the vertex in which the query \func{Next-Neighbor} is invoked. We represent a \emph{state} of the generator as triplet $s = (k, u, \ADJ)$, representing that the algorithm is currently processing the $k^\textrm{th}$ \func{Next-Neighbor} query in the iteration with value $u$, and have generated $\ADJ$ so far. Then, each random choice sampled from $\mathsf{F}$ and $\mathsf{F}'$ maps each state $s$ to some new state $s'$; in other words, they describe the transition probability that the generator moves to its next state. The distributions of the new states according to each $\mathsf{F}$ and $\mathsf{F}'$ only differ by $\epsilon = \frac{1}{\poly(n)}$ in the $L_1$-distance; that is, at least a factor of $1-\epsilon$ of the ''agreement'' between the distributions over states according to these two transitions is necessarily conserved after a single sampling process. 

Observe that a generator can only undergo at most $S = O(n^3)$ states before it generate the entire graph: for the specification of a state $s = (k, u, \ADJ)$, there are up to $n^2$ values of $k$ and $O(n)$ values of $u$. By an inductive argument, the distribution of $\ADJ$ generated by the ''correct'' sequence of distributions $\mathsf{F}$'s and the ``actual'' sequence of distributions $\mathsf{F}'$'s still agree on at least a fraction of $(1-\epsilon)^S \geq 1-\epsilon S = 1-\frac{1}{\poly(n)}$ of their probability masses, yielding the desired $L_1$-distance guarantee.
We remark that our proof above is rather robust: it is straightforward to embed more information into the state depending on the application, and the proof will still guarantee the desired correctness as long as $\epsilon S \leq \frac{1}{\poly(n)}$.
\fi

\subsubsection{Performance Guarantee}
We now show the following lemma that bounds the required resources per query of Algorithm~\ref{alg:oblivious-coin-toss}.
\begin{restatable}{lemma}{res:ER-rand-iterations}\label{lem:ER-rand-iterations}
The loop of Algorithm~\ref{alg:oblivious-coin-toss} requires at most $O(\log n)$ iterations with high probability.
\end{restatable}

\paragraph*{Specifying random choices} The performance of the algorithm depends on not only the random variables $X_{v,u}$'s, but also the unused coins $C_{v,u}$'s. We characterize the two collections of Bernoulli variables $\{X_{v,u}\}$ and $\{Y_{v,u}\}$ that cover all random choices made by Algorithm~\ref{alg:oblivious-coin-toss} as follows.

\begin{itemize}
\item Each $X_{v,u}$ (same as $X_{u,v}$) represents the result for the \emph{first} coin-toss corresponding to cells $\ADJ[v][u]$ and $\ADJ[u][v]$, which is the coin-toss obtained when $X_{v,u}$ becomes decided: either $C_{v,u}$ during a \func{Next-Neighbor}$(v)$ call when $\ADJ[v][u] = \PHI$, or $C_{v,u}$ during a \func{Next-Neighbor}$(u)$ call when $\ADJ[u][v] = \PHI$, whichever occurs first.
This description of $X_{v,u}$ respects our invariant that, if the generation process is executed to completion, we will have $\ADJ[v][u]=X_{v,u}$ in all entries.
\item Each $Y_{v,u}$ represents the result for the \emph{second} coin-toss corresponding to cell $\ADJ[v][u]$, which is the coin-toss $C_{v,u}$ obtained during a \func{Next-Neighbor}$(v)$ call when $X_{v,u}$ is already decided. In other words, $\{Y_{v,u}\}$'s are the coin-tosses that should have been skipped but still performed in Algorithm~\ref{alg:oblivious-coin-toss} (if they have indeed been generated). Unlike the previous case, $Y_{v,u}$ and $Y_{u,v}$ are two independent random variables: they may be generated during a \func{Next-Neighbor}$(v)$ call and a \func{Next-Neighbor}$(u)$ call, respectively.
\end{itemize}
As mentioned earlier, we allow any sequence of probabilities $p_{v,u}$ in our proof. The success probabilities of these indicators are therefore given by $\mathbb P[X_{v,u}=\ONE] = \mathbb P[Y_{v,u}=\ONE] = p_{v,u}$.

\anak{Consider improving paragraph below.}
\paragraph*{Characterizing iterations}
Suppose that we compute \func{Next-Neighbor}$(v)$ and obtain an answer $u$. Then $X_{v,\LAST[v]+1} = \cdots = X_{v, u-1} = \ZERO$ as none of $u' \in (\LAST[v], u)$ is a neighbor of $v$. The vertices considered in the loop of Algorithm~\ref{alg:oblivious-coin-toss} that do not result in the answer $u$, are $u' \in (\LAST[v], u)$ satisfying $\ADJ[v][u'] = \ZERO$ and $Y_{v,u'} = \ONE$; we call the iteration corresponding to such a $u'$ a \emph{failed iteration}. Observe that if $X_{v,u'} = \ZERO$ but is undecided ($\ADJ[v][u'] = \PHI$), then the iteration is not failed, even if $Y_{v,u'} = \ONE$ {\color{red}(in which case, $X_{v,u'}$ takes the value of $C_{v,u'}$ while $Y_{v,u'}$ is never used)}. Thus we assume the worst-case scenario where all $X_{v,u'}$ are revealed: $\ADJ[v][u']=X_{v,u'}=\ZERO$ for all $u'\in(\LAST[v], u)$. The number of failed iterations in this case stochastically dominates those in all other cases.\footnote{There exists an adversary who can enforce this worst case. Namely, an adversary that first makes \func{Next-Neighbor} queries to learn all neighbors of every vertex except for $v$, thereby filling out the whole $\ADJ$ in the process. The claimed worst case then occurs as this adversary now repeatedly makes \func{Next-Neighbor} queries on $v$. In particular, a committee of $n$ adversaries, each of which is tasked to perform this series of calls corresponding to each $v$, can always expose this worst case.}

Then, the upper bound on the number of failed iterations of a call \func{Next-Neighbor}$(v)$ is given by the maximum number of cells $Y_{v, u'} = 1$ of $u' \in(\LAST[v], u)$, over any $u \in(\LAST[v], n]$ satisfying $X_{v,\LAST[v]+1} = \cdots = X_{v,u} = \ZERO$. Informally, we are asking ''of all consecutive cells of $\ZERO$'s in a single row of $\{X_{v,u}\}$-table, what is the largest number of cells of $\ONE$'s in the corresponding cells of $\{Y_{v,u}\}$-table?''

\paragraph*{Bounding the number of iterations required for a fixed pair $(v, \LAST[v])$}
We now proceed to bounding the number of iterations required over a sampled pair of $\{X_{v,u}\}$ and $\{Y_{v,u}\}$, from any probability distribution. For simplicity we renumber our indices and drop the index $(v,\LAST[v])$ as follows. Let $p_1, \ldots, p_L \in [0, 1]$ denote the probabilities corresponding to the cells $\ADJ[v][\LAST[v]+1 \ldots n]$ (where $L = n-\LAST[v]$), then let $X_1, \ldots, X_L$ and $Y_1, \ldots, Y_L$ be the random variables corresponding to the same cells on $\ADJ$.

For $i=1, \ldots, L$, define the random variable $Z_i$ in terms of $X_i$ and $Y_i$ so that
\begin{itemize}
\item $Z_i = 2$ if $X_i = 0$ and $Y_i = 1$, which occurs with probability $p_i(1-p_i)$. \\
This represents the event where $i$ is not a neighbor, and the iteration fails.
\item $Z_i = 1$ if $X_i = Y_i = 0$, which occurs with probability $(1-p_i)^2$.\\
 This represents the event where $i$ is not a neighbor, and the iteration does not fail.
\item $Z_i = 0$ if $X_i = 1$, which occurs with probability $p_i$. \\
This represents the event where $i$ is a neighbor.
\end{itemize}

For $\ell \in [L]$, define the random variable $M_\ell := \prod_{i=1}^\ell Z_i$, and $M_0 = 1$ for convenience. If $X_i = 1$ for some $i \in [1, \ell]$, then $Z_i = 0$ and $M_\ell = 0$. Otherwise, $\log M_\ell$ counts the number of indices $i \in [\ell]$ with $Y_i = 1$, the number of failed iterations. Therefore, $\log(\max_{\ell \in \{0, \ldots, L\}} M_\ell)$ gives the number of failed iterations this \func{Next-Neighbor}$(v)$ call.

To bound $M_\ell$, observe that for any $\ell\in[L]$, $\mathbb{E}[Z_\ell] = 2p_\ell(1-p_\ell) + (1-p_\ell)^2 = 1 - p_\ell^2 \leq 1$ regardless of the probability $p_\ell \in [0, 1]$. Then, $\mathbb{E}[M_\ell] = \mathbb{E}[\prod_{i=1}^\ell Z_i] = \prod_{i=1}^\ell \mathbb{E}[Z_i] \leq 1$ because $Z_\ell$'s are all independent. By Markov's inequality, for any (integer) $r \geq 0$, $\Pr[\log M_\ell > r] = \Pr[M_\ell > 2^r] < 2^{-r}$. By the union bound, the probability that more than $r$ failed iterations are encountered is $\Pr[\log(\max_{\ell \in \{0, \ldots, L\}} M_\ell) > r] < L\cdot 2^{-r} \leq n\cdot 2^{-r}$.

\paragraph*{Establishing the overall performance guarantee}
So far we have deduced that, for each pair of a vertex $v$ and its $\LAST[v]$, the probability that the call \func{Next-Neighbor}$(v)$ encounters more than $r$ failed iterations is less that $n \cdot 2^{-r}$, which is at most $n^{-c-2}$ for any desired constant $c$ by choosing a sufficiently large $r = \Theta(\log n)$. As Algorithm~\ref{alg:oblivious-coin-toss} may need to support up to $\Theta(n^2)$ \func{Next-Neighbor} calls, one corresponding to each pair $(v, \LAST[v])$, the probability that it ever encounters more than $O(\log n)$  failed iterations to answer a single \func{Next-Neighbor} query is at most $n^{-c}$. That is, with high probability, $O(\log n)$ iterations are required per \func{Next-Neighbor} call, which concludes the proof of Lemma~\ref{lem:ER-rand-iterations}.
%For the $G(n,p)$ model, each iteration requires $O(\log n)$ time and random bits (for sampling), so this bound on the number of iterations implies that each \func{Next-Neighbor} call requires $O(\log^2 n)$ time, $O(1)$ additional space for the maintained data structure, and $O(\log^2 n)$ random bits with high probability. The former two offer an improvement over Algorithm~\ref{alg:exact-coin-toss}, but note also that the bounds of Algorithm~\ref{alg:exact-coin-toss} holds deterministically.

\subsubsection{Supporting \func{Vertex-Pair} Queries} \label{sec:ER-pair}
We consider extending our generator (Algorithm~\ref{alg:oblivious-coin-toss}) to support the \func{Vertex-Pair} queries: given a pair of vertices $(u, v)$, decide whether there exists an edge $\{u, v\}$ in the generated graph. To answer a \func{Vertex-Pair} query, we must first check whether the value $X_{u,v}$ for $\{u, v\}$ has already been assigned, in which case we answer accordingly. Otherwise, we must make a coin-flip with the corresponding bias $p_{u,v}$ to assign $X_{u,v}$, deciding whether $\{u, v\}$ exists in the generated graph. If we maintained the full $\ADJ$ as done in the na\"{i}ve Algorithm~\ref{alg:naive}, we would have been able to simply set $\ADJ[u][v]$ and $\ADJ[v][u]$ to this new value. However, our more efficient Algorithm~\ref{alg:oblivious-coin-toss} that represents $\ADJ$ compactly via $\LAST$ and $P_v$'s cannot record these arbitrary modifications to $\ADJ$. In our following analysis, we do not modify $\ADJ$ during any \func{Vertex-Pair} query: $\LAST$ and $P_v$'s would still together capture the correct state of $\ADJ$. Instead, we create data structures that explicitly record the modifications made by \func{Vertex-Pair} queries.

More specifically, we define the following additional set data structures:
\begin{align*}
P &= \{(u,v): X_{u,v}\textrm{ is assigned to }\ONE \textrm{ (during any query)}\},\\
Q &= \{(u,v): X_{u,v}\textrm{ is assigned to }\ZERO \textrm{ during a \func{Vertex-Pair} query}\}.
\end{align*}
Maintaining these data structures is trivial: we simply add $(u,v)$ to $P$ or $Q$ before we finish processing the query if their respective conditions hold. Employing hashing, each insertion requires constant time and additional space. We remark that $P_v = \{u \in (\LAST[v], n]: \ADJ[v][u]=1\}$ by definition, so if we assign $X_{u,v} = 1$ during a \func{Vertex-Pair} query, we must also add $u$ to $P_v$ or add $v$ to $P_u$, should they satisfy their respective conditions. Note that updating $P_v$ takes $O(\log n)$ time and $O(1)$ additional space.

To handle a \func{Vertex-Pair}$(u,v)$ query, we first decide whether $X_{u,v}$ has already been assigned or not. If $\LAST[v] \geq u$ or $\LAST[u] \geq v$, then $X_{u,v}$ is already assigned (as $X_{u,v}=\ADJ[u][v] \in \{\ZERO,\ONE\}$). In this case, we simply check whether $(u,v)\in P$ and answer \YES~or \NO~accordingly. Otherwise, we have $u > \LAST[v]$ and $v > \LAST[u]$. This condition implies that $\ADJ[u][v] \neq \ZERO$ by Lemma~\ref{lem:cond-0}. It also implies that $\ADJ[u][v] \neq \ONE$, because otherwise, we would have either increased $\LAST[u]$ to $u$ or vice versa when $\ADJ[u][v]$ was set to $\ONE$. Thus $\ADJ[u][v] = \PHI$, implying that $X_{u,v}$ could not have possibly been assigned a value during a \func{Next-Neighbor} query.

In other words, if $X_{u,v}$ is assigned, it must have happened during a \func{Vertex-Pair} query. Since all assignments done by \func{Vertex-Pair} queries are recorded in $P$ and $Q$, we answer \YES~if $(u,v)\in P$, answer \NO~if $(u,v)\in Q$, or safely conclude that $X_{u,v}$ has not been assigned otherwise. In the last case, we perform a coin-toss to assign $X_{u,v}$, and update the data structures as outlined above. This process takes $O(\log n)$ time and $O(1)$ additional space per \func{Vertex-Pair} call in the worst case.

We now explain other necessary changes to Algorithm~\ref{alg:oblivious-coin-toss}. In the implementation of \func{Next-Neighbor}, an iteration is not failed when the chosen $X_{v,u}$ is still undecided: $\ADJ[v][u]$ must still be $\phi$. Since $X_{v,u}$ may also be assigned to $\ZERO$ via a \func{Vertex-Pair}$(v,u)$ query, we must also consider an iteration where $(v,u) \in Q$ failed. That is, we now require one additional condition $(v,u) \notin Q$ for termination (which only takes $O(1)$ time to verify per iteration). As for the analysis, aside from handling the fact that $X_{v,u}$ may also be assigned during a \func{Vertex-Pair} call, and allowing the states of the algorithm to support \func{Vertex-Pair} queries, all of the remaining analysis for correctness and performance guarantee still holds. 


Therefore, we have established that our augmentation to Algorithm~\ref{alg:oblivious-coin-toss} still maintains all of its (asymptotic) performance guarantees for \func{Next-Neighbor} queries, and supports \func{Vertex-Pair} queries with complexities as specified above, concluding the following theorem.
\begin{restatable}{theorem}{res:oblivious-thm}
Let $t(n)$ and $w(n)$ denote the time and random $N$-bit words required to sample from the distribution $\mathsf{F}(v,a,b)$ with error $\frac{1}{\poly(n)}$ in the $L_1$-distance, respectively. Then Algorithm~\ref{alg:oblivious-coin-toss} is a random access generator that answers each \func{Next-Neighbor} or \func{Vertex-Pair} query using $O(t(n) \log n)$ time, $O(w(n) \log n)$ random $N$-bit words, and $O(\log n)$ additional space, with high probability.
\end{restatable} 
\fi
